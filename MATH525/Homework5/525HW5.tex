\documentclass{article}
\usepackage[top = 0.9in, bottom = 0.9in, left =1in, right = 1in]{geometry}
\usepackage[utf8]{inputenc}
\usepackage{hyperref}
\usepackage{listings}
\usepackage{multimedia} % to embed movies in the PDF file
\usepackage{graphicx}
\usepackage{comment}
\usepackage[english]{babel}
\usepackage{amsmath}
\usepackage{amssymb}
\usepackage{amsfonts}
\usepackage{wrapfig}
\usepackage{multirow}
\usepackage{verbatim}
\usepackage{float}
\usepackage{cancel}
\usepackage{caption}
\usepackage{subcaption}
\usepackage{mathdots}
\usepackage{/home/cade/Homework/latex-defs}


\title{MATH 525 Homework 5}
\author{Cade Ballew \#2120804}
\date{February 9, 2024}

\begin{document}
	
\maketitle
	
\section{Problem 1}
Let $X$ be a locally convex vector space with seminorms $\{p_\alpha\}_{\alpha\in A}$. If the topology is equivalent to one defined by a norm $p$, then for all $\alpha\in A$ and some $C>0$,
\[
p_\alpha(x)\leq C_\alpha p(x).
\]
Let $E=B_{1,p}$, the one-ball in the norm. Then, for all $\alpha\in A$,
\[
\sup_{x\in E}p_\alpha(x)\leq\sup_{x\in E}C_\alpha p(x)\leq C_\alpha,
\]
so each seminorm defining the topology is bounded on $E$. As the unit ball, $E$ is open, so $E$ is an open bounded set.

Conversely, let $X$ contain an open bounded set $E$. Then, because $E$ is open and $X$ is locally convex, there exist some $\alpha_1,\ldots,\alpha_n\in A$ and $\epsilon_1,\ldots,\epsilon_n>0$ such that
\[
E\supset \bigcap_{j=1}^nB_{\epsilon_j,\alpha_j}=\{x:p_{\alpha_j}(x)<\epsilon_j,~j=1,\ldots,n\}.
\]
Define 
\[
p(x)=\sum_{j=1}^np_{\alpha_j}(x),\quad \epsilon=\sum_{j=1}^n\epsilon_j.
\]
Then, $p$ is clearly a seminorm because nonnegativity, homogeneity, and the triangle inequality all immediately follow from its definition as the finite sum of seminorms. Furthermore, 
\[
E\supset\{x:p_{\alpha_j}(x)<\epsilon_j,~j=1,\ldots,n\}\supset B_{\epsilon, p}.
\]
Since $E$ is bounded, for each $\alpha\in A$, there exists some $C_\alpha>0$ such that $p_\alpha(x)<C_\alpha$ for all $x\in E$. In particular, this implies that on a neighborhood of the origin $F=B_{\epsilon, p}$, $p(x)<\epsilon$ and $p_\alpha(x)<C_\alpha$ for all $x\in F$. Then, for $x\in F$ such that $p(x)\neq0$, $\frac{\epsilon}{p(x)}x\in F$, so
\[
\frac{\epsilon}{p(x)}p_\alpha(x)=p_\alpha\left(\frac{\epsilon}{p(x)}x\right)<C_\alpha,
\]
and $p_\alpha(x)<\frac{C_\alpha}{\epsilon}p(x)$. If instead $p(x)=0$, then $p(cx)=0$ for all $c\in\compl$, so $x\in F$, but $p_\alpha(cx)=|c|p_\alpha(x)<C_\alpha$ for all $c\in\compl$. Thus, $p_\alpha(x)=0<\frac{C_\alpha}{\epsilon}p(x)$, and the inequality is satisfied for all $x\in F$. Because $F$ is absorbing, for any $x\in X$, $x=\lambda y$ for some $\lambda\geq0$, $y\in F$, so by homeogeneity, this implies that for all $x\in X$, $p_\alpha(x)<\frac{C_\alpha}{\epsilon}p(x)$, since a factor $1/\lambda$ divides through both sides. Since $p(x)=\sum_{j=1}^np_{\alpha_j}(x)$ for all $x\in X$ by construction, $p$ generates the same topology as $\{p_\alpha\}_{\alpha\in A}$. Finally, assuming that $X$ is Hausdorff, for every $x\neq 0$, there exists some $\alpha\in A$ for which $p_\alpha(x)>0$. For this $\alpha$,
\[
0<\frac{\epsilon}{C_\alpha}p_\alpha(x)<p(x),
\]
so $p(x)\neq 0$ for all $x\neq 0$. Thus, $p$ is nondegenerate and therefore a norm, so the locally convex topology generated by the seminorms $\{p_\alpha\}_{\alpha\in A}$ is equivalent to one generated by the norm $p$. 

\section{Problem 2}
Let $M$ be a vector subspace of a normed vector space $X$. If $M=X$, then it is closed in any topology, so it is trivially closed in the norm topology if and only if it is weakly closed. Assume that $M\subsetneq X$ and let $M$ be closed in the norm topology. Then, for any $x\in M^c$, Theorem 5.8a gives that there exists some $f\in X^*$ such that $f\big|_M=0$ and $f(x)=\delta>0$. Consider the open ball with respect to the seminorm $|f|$ given by
\[
B_{\delta,f}(x)=\{y\in X:|f(x-y)|<\delta\}=\{y\in X:0<f(y)<2\delta\}.
\]
Since $f\big|_M=0$, we have that $B_{\delta,f}(x)\subset M^c$. Since we can find such an open ball in some seminorm for any $x\in M^c$, $M^c$ must be open in the seminorm topology, meaning that $M$ is weakly closed.

Conversely, let $M$ be weakly closed and let $x\in\overline{M}$ with $\overline{M}$ defined in the norm topology. That is, there exists some sequence $\{x_n\}_{n=1}^\infty$ such that $\lim_{n\to\infty}\|x_n-x\|=0$. Then, for any $f\in X^*$, 
\[
\lim_{n\to\infty}|f(x_n)-f(x)|\leq\lim_{n\to\infty}\|f\|_{X^*}\|x_n-x\|=0,
\]
since $\|f\|_{X^*}$ is finite. Thus, $\{x_n\}_{n=1}^\infty\to x$ in the weak topology, so $x\in M$ since $M$ is closed under this topology. This means that in the norm topology, $M=\overline{M}$, so $M$ is closed in the norm topology as well. 

\section{Problem 3}
Let $X$ be a normed space and $f_j\in X^*$ converge weak* to $f$. That is, $f_j(x)\to f(x)$ for all $x\in X$. Then, for all $x\in X$,
\[
|f(x)|=\lim_{j\to\infty}|f_j(x)|=\liminf_{j\to\infty}|f_j(x)|\leq\liminf_{j\to\infty}\|f_j\|_{X^*}\|x\|=\|x\|\liminf_{j\to\infty}\|f_j\|_{X^*}.
\]
Thus,
\[
\|f\|_{X^*}=\sup_{x\in X}\frac{|f(x)|}{\|x\|}\leq\liminf_{j\to\infty}\|f_j\|_{X^*}.
\]
As an example for which this inequality is strict, let $X=\ell^1(\mathbb{N})$ and for any $x=\{x_j\}_{j\in\mathbb{N}}\in X$, define the functionals
\[
f_n(x)=\sum_{j=n+1}^\infty x_j.
\]
Because we have the correspondence $\ell^1(\mathbb{N})^*\equiv\ell^\infty(\mathbb{N})$, each $f_n$ corresponds to a sequence in $\ell^\infty(\mathbb{N})$ defined by
\[
(f_n)_j=f_n(e_j)=\begin{cases}
	1,\quad j\geq n+1,\\
	0,\quad j<n+1.
\end{cases}
\]
Define $f$ to be the zero functional on $X=\ell^1(\mathbb{N})$. Then, for any $x\in X$,
\[
\lim_{n\to\infty}|f_n(x)-f(x)|\leq\lim_{n\to\infty}\sum_{j=n+1}^\infty|x_j|=0,
\]
so $\{f_n\}$ converges weak* to $f$. However, $\|f\|_{X^*}=0$, but for any $n\in\mathbb{N}$,
\[
\|f_n\|_{X^*}=\sup_{j\in\mathbb{N}}(f_n)_j=1,
\]
so $\liminf_{j\to\infty}\|f_j\|_{X^*}=1$, and the inequality is sharp.

\section{Problem 4 (Folland Problem 38)}
Let $X$ and $Y$ be Banach spaces and $\{T_n\}\subset\cL(X,Y)$ such that $\lim_{n\to\infty} T_nx$ exists for every $x\in X$. Define $T$ by $Tx=\lim_{n\to\infty} T_nx$. To show that $T\in\cL(X,Y)$, we first verify that $T$ is linear.
\begin{itemize}
	\item If $x,y\in X$, then
	\[
	T(x+y)=\lim_{n\to\infty} T_n(x+y)=\lim_{n\to\infty} (T_nx+T_ny)=\lim_{n\to\infty} T_nx+\lim_{n\to\infty} T_ny=Tx+Ty.
	\]
	\item If $x\in X$ and $\lambda\in\compl$, then
	\[
	T(\lambda x)=\lim_{n\to\infty} T_n(\lambda x)=\lim_{n\to\infty} \lambda T_nx=\lambda\lim_{n\to\infty} T_nx=\lambda T_nx.
	\]
\end{itemize}
To see that $T$ is bounded, we note that because $\lim_{n\to\infty} T_nx$ exists for every $x\in X$, we must have that
\[
\sup_{n\in\mathbb{N}}\|T_nx\|<\infty,
\]
for all $x\in X$. Thus, an application of the uniform boundedness principle yields that $\sup_{n\in\mathbb{N}}\|T_n\|=C<\infty$ for some constant $C$. Then, for any $x\in X$, 
\[
\|Tx\|=\lim_{n\to\infty}\|T_nx\|\leq\lim_{n\to\infty}\|T_n\|\|x\|\leq\sup_{n\in\mathbb{N}}\|T_n\|\|x\|=C\|x\|.
\]
Thus, $\|T\|\leq C$, so $T$ is bounded and $T\in\cL(X,Y)$.

\section{Problem 5 (Folland Problem 48)}
Let $X$ be a Banach space.
\subsection{Part a}
Consider the norm-closed unit ball $B=\{x\in X:\|x\|\leq1\}$. To see that $B$ is also weakly closed, let $\{x_n\}_{n=1}^\infty\subset B$ such that $f(x_n)\to f(x)$ for all $f\in X^*$. To show that $B$ is weakly closed, we need to show that $x\in B$. For any $f\in X^*$,
\[
|\hat x(f)|=|f(x)|=\lim_{n\to\infty}|f(x_n)|\leq\lim_{n\to\infty}\|f\|\|x_n\|=\|f\|,
\]
so 
\[
\|\hat x\|=\sup_{f\in X^*}\frac{|\hat x(f)|}{\|f\|}\leq 1.
\]
By Theorem 5.8d, the map $x\mapsto\hat x$ is norm-preserving, so $\|x\|=\|\hat x\|\leq1$, so $x\in B$, and $B$ is weakly closed.

\subsection{Part b}
Let $E\subset X$ be bounded in the norm topology. Then $E\subset B(r,0)$ for some $r>0$, so $r^{-1}E\subset B$ using the notation of part a. Since $B$ is closed in the weak topology, $\overline{r^{-1}E}\subset B$ with the closure taken in the weak topology. Since this is just a dilation, $r^{-1}\overline E\subset B$, and $\overline{E}\subset rB$. Thus, for any $\epsilon>0$, $\overline{E}\subset B(r+\epsilon,0)$, so the weak closure of $E$ is bounded in the norm topology.

\subsection{Part c}
Let $F\subset X^*$ be bounded in the norm topology. Then, as before, $F\subset B(r,0)$ for some $r>0$, so $r^{-1}E\subset B^*$ where $B^*=\{f\in X^*:\|f\|\leq 1\}$. By Alaoglu's theorem, $B^*$ is compact in the weak* topology and therefore closed. Thus, by the same argument as in part b, $\overline{r^{-1}F}\subset B^*$, $r^{-1}\overline F\subset B^*$, $\overline{F}\subset rB^*$, and for any $\epsilon>0$, $\overline{F}\subset B(r+\epsilon,0)$, so the weak* closure of $F$ is bounded in the norm topology.

\subsection{Part d}
Let $\{f_n\}_{n=1}^\infty$ be a weak* Cauchy sequence in $X^*$. That is, for a given $x\in X$, 
\[
\lim_{m,n\to\infty}|f_n(x)-f_m(x)|=0.
\]
This means that for any given $x\in X$, $\{f_n(x)\}_{n=1}^\infty$ is a Cauchy sequence in the usual metric in $\compl$. Since $\compl$ is complete, $\{f_n(x)\}_{n=1}^\infty$ is a convergent sequence. Define the functional $f$ by
\[
f(x)=\lim_{n\to\infty}f_n(x),
\]
for all $x\in X$. Because $\{f_n(x)\}_{n=1}^\infty$ is convergent, $\lim_{n\to\infty}f_n(x)$ exists for all $x\in X$. $X$ and $\compl$ are both Banach spaces, so Problem 4 implies that $f\in\cL(X,\compl)$, i.e., $f\in X^*$. Since $f_n(x)\to f(x)$ for all $x\in X$, $\{f_n\}_{n=1}^\infty$ converges to $f$ in the weak topology. Thus, any weak* Cauchy sequence in $X^*$ converges.

\end{document}
