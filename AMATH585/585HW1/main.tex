\documentclass{article}
\usepackage[utf8]{inputenc}
\usepackage{listings}
\usepackage{multimedia} % to embed movies in the PDF file
\usepackage{graphicx}
\usepackage{comment}
\usepackage[english]{babel}
\usepackage{amsmath}
\usepackage{amsfonts}
\usepackage{subfigure}
\usepackage{wrapfig}
\usepackage{multirow}
\usepackage{tikz}
\usepackage{verbatim}
\usepackage{float}
%!TEX root = main.tex



\newcommand{\eref}[1]{\mbox{\rm(\ref{#1})}}
\newcommand{\tref}[1]{\mbox{\rm\ref{#1}}}
\newcommand{\set}[2]{\left\{ #1 \; : \; #2 \right\} }
\newcommand{\deq}{\raisebox{0pt}[1ex][0pt]{$\stackrel{\scriptscriptstyle{\rm def}}{{}={}}$}}

\newcommand {\DS} {\displaystyle}

\newcommand{\real}{\mathbb{R}}



\newcommand {\half} {\mbox{$\frac{1}{2}$}}
\newcommand{\force}{{\mathbf{f}}}
\newcommand{\strain}{{\boldsymbol{\varepsilon}}}
\newcommand{\stress}{{\boldsymbol{\sigma}}}
\renewcommand{\div}{{\boldsymbol{\nabla}}}

\newcommand {\cA} {{\cal A}}
\newcommand {\cB} {{\cal B}}
\newcommand {\cC} {{\cal C}}
\newcommand {\cD} {{\cal D}}
\newcommand {\cE} {{\cal E}}
\newcommand {\cK} {{\cal K}}
\newcommand {\cL} {{\cal L}}
\newcommand {\cP} {{\cal P}}
\newcommand {\cQ} {{\cal Q}}
\newcommand {\cR} {{\cal R}}
\newcommand {\cV} {{\cal V}}
\newcommand {\cW} {{\cal W}}
\newcommand {\CC} {{\cal C}}
\newcommand {\CD} {{\cal D}}
\newcommand {\CH} {{\cal H}}
\newcommand {\CS} {{\cal S}}
\newcommand {\CU} {{\cal U}}
\newcommand {\CY} {{\cal Y}}



\newcommand{\bzero}{\mathbf{0}}
\newcommand{\ba}{\mathbf{a}}
\newcommand{\bb}{\mathbf{b}}
\newcommand{\bc}{\mathbf{c}}
\newcommand{\bd}{\mathbf{d}}
\newcommand{\be}{\mathbf{e}}
\newcommand{\bg}{\mathbf{g}}
\newcommand{\bh}{\mathbf{h}}
\newcommand{\bl}{\mathbf{l}}
\newcommand{\bn}{\mathbf{n}}
\newcommand{\bp}{\mathbf{p}}
\newcommand{\bq}{\mathbf{q}}
\newcommand{\br}{\mathbf{r}}
\newcommand{\bs}{\mathbf{s}}
\newcommand{\bt}{\mathbf{t}}
\newcommand{\bu}{\mathbf{u}}
\newcommand{\bv}{\mathbf{v}}
\newcommand{\bw}{\mathbf{w}}
\newcommand{\bx}{\mathbf{x}}
\newcommand{\by}{\mathbf{y}}
\newcommand{\bz}{\mathbf{z}}
\newcommand{\bA}{{\mathbf A}}
\newcommand{\bB}{\mathbf{B}}
\newcommand{\bC}{\mathbf{C}}
\newcommand{\bD}{\mathbf{D}}
\newcommand{\bE}{\mathbf{E}}
\newcommand{\bF}{\mathbf{F}}
\newcommand{\bG}{\mathbf{G}}
\newcommand{\bH}{\mathbf{H}}
\newcommand{\bI}{\mathbf{I}}
\newcommand{\bJ}{\mathbf{J}}
\newcommand{\bK}{\mathbf{K}}
\newcommand{\bL}{\mathbf{L}}
\newcommand{\bM}{\mathbf{M}}
\newcommand{\bN}{\mathbf{N}}
\newcommand{\bO}{\mathbf{O}}
\newcommand{\bP}{\mathbf{P}}
\newcommand{\bQ}{\mathbf{Q}}
\newcommand{\bR}{\mathbf{R}}
\newcommand{\bS}{\mathbf{S}}
\newcommand{\bU}{\mathbf{U}}
\newcommand{\bV}{\mathbf{V}}
\newcommand{\bW}{\mathbf{W}}
\newcommand{\bX}{\mathbf{X}}
\newcommand{\bY}{\mathbf{Y}}
\newcommand{\bZ}{\mathbf{Z}}

\newcommand{\bgamma}{{\boldsymbol{\gamma}}}
\newcommand{\bmu}{{\boldsymbol{\mu}}}
\newcommand{\bkappa}{{\boldsymbol{\kappa}}}
\newcommand{\blambda}{{\boldsymbol{\lambda}}}
\newcommand{\bLambda}{{\boldsymbol{\Lambda}}}
\newcommand{\bpi}{{\boldsymbol{\pi}}}
\newcommand{\bPi}{{\boldsymbol{\Pi}}}
\newcommand{\btheta}{{\boldsymbol{\theta}}}
\newcommand{\bTheta}{{\boldsymbol{\Theta}}}
\newcommand{\bSigma}{{\boldsymbol{\Sigma}}}






\title{AMATH 585 Homework 1}
\author{Cade Ballew}
\date{January 14, 2022}

\begin{document}

\maketitle

\section{Problem 1}
We evaluate the second order 
accurate approximation 
\[
u'' (x) \approx \frac{u(x+h) + u(x-h) - 2 u(x)}{h^2}
\]
for $u(x) = \sin x$ and $x = \pi / 6$ for $h = 10^{-1} , 10^{-2} , \ldots ,
10^{-16}$ using the following MATLAB code. 
\begin{verbatim}
u=@(x) sin(x);
x = pi/6;
upptrue = -sin(x); %true u''(x) value
fprintf('  h       FD Quotient      Error\n')
for k=1:16
    h = 10^-k;
    upp = (u(x+h)+u(x-h)-2*u(x))/h^2;
    err = upp-upptrue; %error term
    fprintf('%.e   %e  %e\n',h, upp, err)
end
\end{verbatim}
The resulting output is displayed in the following table. 

\begin{table}[H]\centering
\begin{tabular}{|r|r|r|} \hline
{h} & {FD Approximation} & {Error} \\ \hline
1e-01   &-4.995835e-01&  4.165278e-04\\
1e-02  & -4.999958e-01&  4.166653e-06\\
1e-03 &  -5.000000e-01&  4.167450e-08\\
1e-04&   -5.000000e-01&  3.038735e-09\\
1e-05   &-5.000012e-01 & -1.151593e-06\\
1e-06  & -4.999334e-01  &6.657201e-05\\
1e-07 &  -4.996004e-01  &3.996389e-04\\
1e-08&   -1.110223e+00&  -6.102230e-01\\
1e-09   &1.110223e+02 & 1.115223e+02\\
1e-10  & 0.000000e+00 & 5.000000e-01\\
1e-11 &  0.000000e+00 & 5.000000e-01\\
1e-12&   0.000000e+00 & 5.000000e-01\\
1e-13  & 1.110223e+10 & 1.110223e+10\\
1e-14 &  -1.110223e+12 & -1.110223e+12\\
1e-15&   0.000000e+00 & 5.000000e-01\\
1e-16&   -1.110223e+16 & -1.110223e+16\\ \hline
\end{tabular}
\end{table}
One can see that the error term appears to be second order as anticipated at first but actually increases as $h$ decreases when $h<10^{-4}$. As discussed in class, this is due to the limitations of finite precision arithmetic. As $h$ gets smaller, round-off errors become more prominent and dominate our approximation error. Eventually, we run into catastrophic cancellation and obtain errors that are much larger than that of even $h=0.1$. 

\section{Problem 2}
Using the same FD formula for the same $u$ and $x$ as in problem 1, we perform two steps of Richardson extrapolation for $h=0.2$. From class, we have that if $\phi_0(h)$ denotes our original approximation, the first step of Richardson extrapolation is given by 
\[
\phi_1(h)=\frac{4\phi_0(h/2)-\phi_0(h)}{3}
\]
and the second step is given by 
\[
\phi_2(h)=\frac{16\phi_1(h/2)-\phi_1(h)}{15}.
\]
The following MATLAB code computes and prints the values of $\phi_0,\phi_1,\phi_2$ at the necessary values of $h$.
\begin{verbatim}
u=@(x) sin(x);
x = pi/6;
upptrue = -sin(x); %true u''(x) value
fprintf(' h       phi_0(h)       Error\n')

h = [0.2 0.1 0.05];
upp = (u(x+h)+u(x-h)-2*u(x))./h.^2; %phi_0
err = upp-upptrue;

for i=1:length(h)
    fprintf('%0.2f   %d %d\n',h(i), upp(i), err(i))
end

%phi_1
fprintf(' h       phi_1(h)        Error\n')
R1 = (4*upp(2)-upp(1))/3; %combine 0.2 and 0.1
errR1=R1-upptrue;
R2 = (4*upp(3)-upp(2))/3; %combine 0.1 and 0.05
errR2=R2-upptrue;
fprintf('%0.2f  %d  %d\n',h(1), R1, errR1)
fprintf('%0.2f  %d  %d\n',h(2), R2, errR2)

R3 = (16*R2-R1)/15; %phi_2
errR3=R3-upptrue;
fprintf(' h       phi_2(h)        Error\n')
fprintf('%0.2f  %d  %d\n',h(1), R3, errR3)
\end{verbatim}
The resulting output is displayed in the following table. 
\begin{table}[H]\centering
\begin{tabular}{|r|r|r|} \hline
& {FD Approximation} & {Error} \\ \hline
$\phi_0(0.2)$   &-4.983356e-01  &1.664446e-03\\
$\phi_0(0.1)$   &-4.995835e-01  &4.165278e-04\\
$\phi_0(0.05)$   &-4.998958e-01  &1.041580e-04\\ \hline
$\phi_1(0.2)$  &-4.999994e-01  &5.550598e-07\\
$\phi_1(0.1)$ &-5.000000e-01  &3.471449e-08\\ \hline
$\phi_2(0.2)$  &-5.000000e-01  &2.480538e-11\\ \hline
\end{tabular}
\end{table}
From class, we know that one step of Richardson extrapolation should give $\OO(h^4)$ error and two steps should give $\OO(h^6)$ error. This does appear to be roughly true of our obtained error values, as $5.55e-7/(0.2)^4\approx3.47e-08/(0.1)^4$ and the error when going from $\phi_1$ to $\phi_2$ decreases by approximately the same magnitude as when going from $\phi_0$ to $\phi_1$ (as they should since $\phi_0$ is $\OO(h^2))$.

\section{Problem 3}
To derive the error term for the approximation
\[
u' (x) \approx \frac{1}{2h} [ -3 u(x) + 4 u(x+h) - u(x+2h) ],
\]
we first write out the Taylor series
\begin{align*}
u(x+h)&=u(x)+hu'(x)+\frac{h^2}{2!}u''(x)+\frac{h^3}{3!}u'''(x)+\OO(h^4)\\&=
u(x)+hu'(x)+\frac{h^2}{2}u''(x)+\frac{h^3}{6}u'''(x)+\OO(h^4)
\end{align*}
\begin{align*}
u(x+2h)&=u(x)+2hu'(x)+\frac{(2h)^2}{2!}u''(x)+\frac{(2h)^3}{3!}u'''(x)+\OO(h^4)\\&=u(x)+2hu'(x)+2h^2u''(x)+\frac{4h^3}{3}u'''(x)+\OO(h^4) 
\end{align*}
From this, we can compute
\begin{align*}
&\frac{1}{2h} [ -3 u(x) + 4 u(x+h) - u(x+2h) ]\\&=\frac{1}{2h}\left((-3+4-1)u(x)+(4h-2h)u'(x)+\bigg(4\frac{h^2}{2}-2h^2\bigg)u''(x)+\bigg(4\frac{h^3}{6}-\frac{4h^3}{3}\bigg)u'''(x) +\OO(h^4)\right)\\&=
\frac{1}{2h}\left(2hu'(x)-\frac{2h^3}{3}u'''(x)+\OO(h^4)\right)=u'(x)-\frac{h^2}{3}u'''(x)+\OO(h^3).
\end{align*}
Thus, the error term of the approximation is given by 
\[
u'(x)-\frac{h^2}{3}u'''(x)+\OO(h^3)-u'(x)=-\frac{h^2}{3}u'''(x)+\OO(h^3).
\]

\section{Problem 4}
Re-purposing the Taylor series written out in problem 3, we can compute
\begin{align*}
&A u(x) + B u(x+h) + C u(x+2h)\\&=
(A+B+C)u(x)+(B+2C)hu'(x)+ \left(\frac{B}{2}+2C\right)h^2u''(x)+\left(\frac{B}{6}+\frac{4C}{3}\right)h^3u'''(x)+\ldots.   
\end{align*}
In order to achieve maximal order of accuracy when using this as an approximation to $u''(x)$, we require that 
\begin{equation*}
\begin{split}
&A+B+C=0\\
&B+2C=0\\
&\frac{B}{2}+2C=\frac{1}{h^2}
\end{split}
\end{equation*}
Of course, we would obtain higher order accuracy if we could also set $\frac{B}{6}+\frac{4C}{3}=0$, but we already have three equations and three unknowns, so we cannot do this in general. \\
Solving this system of equations, we get that $A=\frac{1}{h^2}$, $B=-\frac{2}{h^2}$, $C=\frac{1}{h^2}$. 
To make the above statement about this being the maximal order of accuracy more clear, note that $\left(\frac{B}{6}+\frac{4C}{3}\right)h^3=h$, so the term $\left(\frac{B}{6}+\frac{4C}{3}\right)h^3u'''(x)$ is only zero when $u'''(x)=0$ and not in general. Now, we plug in these values of $A, B$, and $C$ to conclude that
\[
A u(x) + B u(x+h) + C u(x+2h)=u''(x)+hu'''(x)+\ldots=u''(x)+\OO(h),
\]
so this approximation is of order $h$ accuracy.

\section{Problem 5}
Using the centered difference formulae 
\[
u''(x)\approx\frac{u_{j+1}+u_{j-1}-2u_j}{h^2}
\]
and 
\[
u'(x)\approx\frac{u_{j+1}-u_{j-1}}{2h},
\]
the solution of the BVP \[u'' + 2xu' - x^2 u = x^2 ,~~~u(0)=1,~~~u(1) = 0\] can be approximated by 
\[
\frac{u_{j+1}+u_{j-1}-2u_j}{h^2}+2x_j\frac{u_{j+1}-u_{j-1}}{2h}-x_j^2u_j=x_j^2
\]
for all interior gridpoints $x_j$. If we take $h=1/4$, our grid becomes $x_0=0,~ x_1=1/4,~ x_2=1/2,~ x_3=3/4,~ x_4=1$. Note that to impose the boundary conditions, we take $u_0=1,~ u_1=0$. Now, we write out this equation explicitly for $j=1,2,3$.
\begin{equation*}
\begin{split}
&\frac{u_{2}+1-2u_1}{(1/4)^2}+x_1\frac{u_{2}-1}{1/4}-(1/4)^2u_1=(1/4)^2,\\
&\frac{u_{3}+u_{1}-2u_2}{(1/4)^2}+(1/2)\frac{u_{3}-u_{1}}{1/4}-(1/2)^2u_2=(1/2)^2,\\
&\frac{0+u_{2}-2u_3}{(1/4)^2}+(3/4)\frac{0-u_2}{h}-(3/4)^2u_3=(3/4)^2.
\end{split}
\end{equation*}
These can be simplified the equations
\begin{equation*}
\begin{split}
&(-32-1/16)u_1+(16+1)u_2=1/16+1-16,\\
&(16-2)u_1+(-32-1/4)u_2+(16+2)u_3=1/4,\\
&(16-3)u_2+(-32-9/16)u_3=9/16,
\end{split}
\end{equation*}
which can be further simplified to
\begin{equation*}
\begin{split}
&-513/16u_1+17u_2=-239/16,\\
&14u_1-129/4u_2+18u_3=1/4,\\
&13u_2-521/16u_3=9/16.
\end{split}
\end{equation*}
Thus can be rewritten as the matrix equation
\[
\begin{pmatrix}
-513/16&17&0\\
14&-129/4&18\\
0&13&-521/16
\end{pmatrix}
\begin{pmatrix}
u_1\\u_2\\u_3
\end{pmatrix}=\begin{pmatrix}
-239/16\\1/4\\9/16
\end{pmatrix}.
\]
\end{document}
