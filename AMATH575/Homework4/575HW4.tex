\documentclass{article}
\usepackage[utf8]{inputenc}
\usepackage{hyperref}
\usepackage{listings}
\usepackage{multimedia} % to embed movies in the PDF file
\usepackage{graphicx}
\usepackage{comment}
\usepackage[english]{babel}
\usepackage{amsmath}
\usepackage{amsfonts}
\usepackage{wrapfig}
\usepackage{multirow}
\usepackage{verbatim}
\usepackage{float}
\usepackage{cancel}
\usepackage{caption}
\usepackage{subcaption}
\usepackage{mathdots}
\usepackage{/home/cade/Homework/latex-defs}


\title{AMATH 575 Problem Set 4}
\author{Cade Ballew \#2120804}
\date{May 31, 2023}

\begin{document}
	
\maketitle

\section{Problem 2}
Consider the ``all-to-all'' coupled system of pulse-coupled phase oscillators on the N-dimensional torus, with coupling strength $\epsilon>0$
\begin{equation*}
	\dot \theta_i = \omega + \epsilon z(\theta_i) \frac{1}{N} \sum_{j=1}^N g(\theta_j) \, \, \,  \mod \; \; 2 \pi  
\end{equation*}
$i=1...N$, and let $z(\theta)=A \sin \theta + B \cos \theta $, $g(\theta) = \sum_{k=1}^\infty a_k \sin(k \theta) + b_k \cos(k \theta)$.

\subsection{Part a}
Applying the same substitution as in class, the averaged system is given by
\[
\dot\psi_i=\epsilon\frac{1}{2\pi}\int_{\psi_i}^{2\pi+\psi_i}z(s)\sum_{j=1}^N\frac{1}{N}g(\psi_j-\psi_i+s)ds,
\]
so we need to compute the integral
\[
I=\int_{\psi_i}^{2\pi+\psi_i}(A \sin s + B \cos s)\sum_{j=1}^N\sum_{k=1}^\infty (a_k \sin(k (\psi_j-\psi_i+s)) + b_k \cos(k (\psi_j-\psi_i+s)))ds.
\]
Applying trig identities, the term inside the summation can be rewritten as
\begin{align*}
&a_k\sin(k (\psi_j-\psi_i))\cos(ks)+a_k\cos(k (\psi_j-\psi_i))\sin(ks)\\&+b_k\cos(k (\psi_j-\psi_i))\cos(ks)-b_k\sin(k (\psi_j-\psi_i))\sin(ks).
\end{align*}
Observe that this is a linear combination of $\sin(ks)$ and $\cos(ks)$, so 
\[
I=\sum_{j=1}^N\sum_{k=1}^\infty\int_{\psi_i}^{2\pi+\psi_i}(A \sin s + B \cos s)(C \sin(ks) + D \cos(ks))ds,
\] 
for some $C,D$ independent of $s$. By the orthogonal of trig polynomials, this integral is zero for $k\neq1$. Thus,
\[
I=\sum_{j=1}^N\int_{\psi_i}^{2\pi+\psi_i}(A \sin s + B \cos s) (a_1 \sin( \psi_j-\psi_i+s) + b_1 \cos(\psi_j-\psi_i+s))ds.
\]
We compute this integral using Mathematica and find that
\[
I=\sum_{j=1}^N\pi\left((Ba_1-Ab_1)\sin(\psi_j-\psi_i)+(Aa_1+Bb_1)\cos(\psi_j-\psi_i)\right).
\]
Thus, we have the averaged system 
 \begin{equation*}
	\dot \psi_i =  \epsilon  \frac{1}{N} \sum_{j=1}^N f(\psi_j - \psi_i) \, \, \,  \mod \; \; 2 \pi, 
\end{equation*} 
where 
\[
f(\psi)=\frac{1}{2}\left((Ba_1-Ab_1)\sin\psi+(Aa_1+Bb_1)\cos\psi\right).
\]

\subsection{Part b}
From homework 3, we know that this averaged system is guaranteed to be a gradient system when $f$ is an odd function. Since sine is an odd function and cosine is an even function, $f$ will be odd iff
\[
Aa_1+Bb_1=0.
\]

\subsection{Part c}
To find a general condition on our constants that guarantees that $\psi_i=c$ for all $i$  is a fixed point for all constants $c$. At these points, we have that 
\[
\dot\psi_i=\epsilon\frac{1}{2N}\sum_{j=1}^N(Aa_1+Bb_1)=\frac{\epsilon}{2}(Aa_1+Bb_1),
\]
so these are guaranteed to be fixed points if 
\[
Aa_1+Bb_1=2\pi m.
\]
for some $m\in\mathbb{Z}$, since we have the $\text{mod}2\pi$. Now, note that 
\begin{align*}
&\pp{\dot\psi_i}{\psi_i}=\frac{1}{2N}\sum_{j\neq i}\left(-(Ba_1-Ab_1)\cos(\psi_j-\psi_i)+(Aa_1+Bb_1)\sin(\psi_j-\psi_i)\right),\\
&\pp{\dot\psi_i}{\psi_j}=\frac{1}{2N}\left((Ba_1-Ab_1)\cos(\psi_j-\psi_i)-(Aa_1+Bb_1)\sin(\psi_j-\psi_i)\right),
\end{align*}
for $j\neq i$. Evaluating this at a fixed point $\psi_i=c$ for all $i$, we get that the Jacobian $J$ has entries
\begin{align*}
&J_{ii}=-\frac{N-1}{2N}(Ba_1-Ab_1),\\
&J_{ij}=\frac{1}{2N}(Ba_1-Ab_1),
\end{align*}
again for $i\neq j$. We can write this out as 
\[
J=\frac{1}{2N}(Ba_1-Ab_1)\begin{pmatrix}
	1-N&1&\cdots&1\\
	1&1-N&\cdots&1\\
	1&1&\ddots&1\\
	1&1&\cdots&1-N
\end{pmatrix}.
\]
It is easy to see that this matrix has a zero eigenvalue with eigenvector
\[
\begin{pmatrix}
	1\\\vdots\\1
\end{pmatrix}
\]
and an eigenvalue of $\frac{1}{2}(Ab_1-Ba_1)$ with $N-1$ linear independent eigenvectors
\[
\begin{pmatrix}
	1\\-1\\0\\\vdots\\0
\end{pmatrix},\begin{pmatrix}
0\\1\\-1\\\vdots\\0
\end{pmatrix},\ldots,\begin{pmatrix}
0\\\vdots\\0\\1\\-1
\end{pmatrix}.
\]
Thus, if $S,U,C$ denote the stable, unstable, and center manifolds, respectively, if $Ab_1-Ba_1<0$, then $\dim(S)=N-1,\dim(U)=0,\dim(C)=1$, if $Ab_1-Ba_1>0$, then $\dim(S)=0,\dim(U)=N-1,\dim(C)=1$, and if $Ab_1-Ba_1=0$, then $\dim(S)=0,\dim(U)=0,\dim(C)=N$.

\section{Problem 3}
Consider a two-dimensional flow with linear part $$
J=
\left[ {\begin{array}{cc}
		1 & 0 \\
		0 & \lambda \\
\end{array} } \right] $$ where $\lambda \neq 0$ is an arbitrary real parameter. Following Bernard's notes, we apply (11.26) to the basis for $H_2$ (11.25). Using Mathematica, we get
\begin{align*}
&L^{(2)}_J\begin{pmatrix}
	x^2\\0
\end{pmatrix}=\begin{pmatrix}
-x^2\\0
\end{pmatrix},\\
&L^{(2)}_J\begin{pmatrix}
	xy\\0
\end{pmatrix}=\begin{pmatrix}
-\lambda xy\\0
\end{pmatrix},\\
&L^{(2)}_J\begin{pmatrix}
	y^2\\0
\end{pmatrix}=\begin{pmatrix}
	(1-2\lambda)y^2\\0
\end{pmatrix},\\
&L^{(2)}_J\begin{pmatrix}
	0\\x^2
\end{pmatrix}=\begin{pmatrix}
	0\\(-2+\lambda)x^2
\end{pmatrix},\\
&L^{(2)}_J\begin{pmatrix}
	0\\xy
\end{pmatrix}=\begin{pmatrix}
	0\\-xy
\end{pmatrix},\\
&L^{(2)}_J\begin{pmatrix}
	0\\y^2
\end{pmatrix}=\begin{pmatrix}
	0\\-\lambda y^2
\end{pmatrix}.\\
\end{align*}
In the case $\lambda\neq1/2,2$, we can see that $\text{range}\left(L^{(2)}_J\right)=H_2$. Thus, there are no quadratic terms in this case, so up to quadratic terms
\[
\begin{cases}
	\dot x=y,\\
	\dot y=\lambda x.
\end{cases}
\]
In the case $\lambda=1/2$, we have that 
\[
\begin{pmatrix}
	y^2\\0
\end{pmatrix},
\]
is not in the range of $L^{(2)}_J$. Thus, the normal form up to quadratic terms is given by
\[
\begin{cases}
	\dot x=y+ay^2,\\
	\dot y=\lambda x,
\end{cases}
\]
for some nonzero constant $a$. For the case $\lambda=2$, we have that 
\[
\begin{pmatrix}
	0\\x^2
\end{pmatrix},
\]
is not in the range of $L^{(2)}_J$. Thus, the normal form up to quadratic terms is given by
\[
\begin{cases}
	\dot x=y,\\
	\dot y=\lambda x+ax^2,
\end{cases}
\]
for some nonzero constant $a$.

\section{Problem 4}
To determine the Takens--Bogdanov normal form to third order, we again look at
\[
L^{(3)}_J=Jh_3-Dh_3J\begin{pmatrix}
	x\\y
\end{pmatrix}.
\]
Now, consider 
\[
H_3=\Span\left\{\begin{pmatrix}
	x^3\\0
\end{pmatrix},\begin{pmatrix}
x^2y\\0
\end{pmatrix},\begin{pmatrix}
xy^2\\0
\end{pmatrix},\begin{pmatrix}
y^3\\0
\end{pmatrix},\begin{pmatrix}
0\\x^3
\end{pmatrix},\begin{pmatrix}
0\\x^2y
\end{pmatrix},\begin{pmatrix}
0\\xy^2
\end{pmatrix},\begin{pmatrix}
0\\y^3
\end{pmatrix}\right\}.
\]
Using Mathematica, we find
\begin{align*}
	&L^{(2)}_J\begin{pmatrix}
		x^3\\0
	\end{pmatrix}=\begin{pmatrix}
		-3x^2y\\0
	\end{pmatrix},\\
	&L^{(2)}_J\begin{pmatrix}
		x^2y\\0
	\end{pmatrix}=\begin{pmatrix}
		-2xy^2\\0
	\end{pmatrix},\\
	&L^{(2)}_J\begin{pmatrix}
		xy^2\\0
	\end{pmatrix}=\begin{pmatrix}
		-y^3\\0
	\end{pmatrix},\\
	&L^{(2)}_J\begin{pmatrix}
		y^3\\0
	\end{pmatrix}=\begin{pmatrix}
		0\\0
	\end{pmatrix},\\
	&L^{(2)}_J\begin{pmatrix}
		0\\x^3
	\end{pmatrix}=\begin{pmatrix}
		x^3\\-3x^2y
	\end{pmatrix},\\
	&L^{(2)}_J\begin{pmatrix}
		0\\x^2y
	\end{pmatrix}=\begin{pmatrix}
		x^2y\\-2xy^2
	\end{pmatrix},\\
	&L^{(2)}_J\begin{pmatrix}
		0\\xy^2
	\end{pmatrix}=\begin{pmatrix}
		xy^2\\-y^3
	\end{pmatrix},\\
		&L^{(2)}_J\begin{pmatrix}
		0\\y^3
	\end{pmatrix}=\begin{pmatrix}
		y^3\\0
	\end{pmatrix}.\\
\end{align*}
By inspection, we see that this is a 6-dimensional set, and that 
\[
\begin{pmatrix}
x^3\\0
\end{pmatrix},\begin{pmatrix}
0\\x^3
\end{pmatrix},
\]
are not contained in it. Thus, the normal form up to third order terms is given by
\[
\begin{cases}
	\dot x=y+a_1x^2+b_1x^3,\\
	\dot y=a_2x^2+b_2x^3,
\end{cases}
\]
for nonzero constants $a_1,a_2,b_1,b_2$.

\section{Problem 5}
Recall from homework 1 that the 
Lorenz equations
\begin{equation*}
	\left\{\begin{array}{l}
		x^{\prime}=10(-x+y) \\
		y^{\prime}=r x-y-x z \\
		z^{\prime}=-\frac{8}{3} z+x y
	\end{array}\right.,
\end{equation*}
have a fixed point at the origin that is stable for $r<1$ and unstable for $r>1$ and two additional fixed points at $\left(-2\sqrt{\frac{2}{3}(r-1)},-2\sqrt{\frac{2}{3}(r-1)},r-1\right)$, $\left(2\sqrt{\frac{2}{3}(r-1)},2\sqrt{\frac{2}{3}(r-1)},r-1\right)$ when $r>1$ that are stable for $r<\frac{470}{19}$. This implies that we have a pitchfork bifurcation at $r=1$ since we go from 1 fixed point to 3 and undergo a change in stability.

\section{Problem 6}
Consider the one-parameter family of one-dimensional maps,
\begin{equation*}
	x \mapsto x^2 + c,
\end{equation*}
where $c$ is a real-valued parameter.

\subsection{Part 1}
Using Mathematica, we find that this map has fixed points at
\[
x=\frac{1}{2}\pm\frac{1}{2}\sqrt{1-4c}
\]
if $c<1/4$. if $c=1/4$, we have one fixed point $\bar x=1/2$, and if $c>1/4$, we have no fixed points. The Jacobian of our map at these fixed points is given by
\[
1\pm\sqrt{1-4c},
\]
is greater than 1 in absolute value when $c<1/4$, so both fixed points are unstable in this case. Because of this, there is a bifurcation at $c=1/4$.

\subsection{Part 2}
We consider $c=-3/4$ and let $p_-=-1/2$ be the smaller fixed point. Then,
\[
f'_{-3/4}(p_-)=-1.
\]
I convinced myself that as c descends through
$-3/4$, we see the emergence of an (attracting) 2-cycle.

\subsection{Part 3}
We solve for the period 2 points by using Mathematica to find the fixed points of the map $f^2_c(\cdot)$ where $c\leq1/4$. This gives two fixed points 
at
\[
x=\frac{1}{2}\pm\frac{1}{2}\sqrt{1-4c},
\]
for all $c\leq1/4$ and two additional fixed points at 
\[
x=-\frac{1}{2}\pm\frac{1}{2}\sqrt{-3-4c},
\]
for $c<-3/4$.

\end{document}
