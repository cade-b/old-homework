\documentclass{article}
\usepackage[utf8]{inputenc}
\usepackage{listings}
\usepackage{multimedia} % to embed movies in the PDF file
\usepackage{graphicx}
\usepackage{comment}
\usepackage[english]{babel}
\usepackage{amsmath}
\usepackage{amsfonts}
\usepackage{subfigure}
\usepackage{wrapfig}
\usepackage{multirow}
\usepackage{verbatim}
%!TEX root = main.tex



\newcommand{\eref}[1]{\mbox{\rm(\ref{#1})}}
\newcommand{\tref}[1]{\mbox{\rm\ref{#1}}}
\newcommand{\set}[2]{\left\{ #1 \; : \; #2 \right\} }
\newcommand{\deq}{\raisebox{0pt}[1ex][0pt]{$\stackrel{\scriptscriptstyle{\rm def}}{{}={}}$}}

\newcommand {\DS} {\displaystyle}

\newcommand{\real}{\mathbb{R}}



\newcommand {\half} {\mbox{$\frac{1}{2}$}}
\newcommand{\force}{{\mathbf{f}}}
\newcommand{\strain}{{\boldsymbol{\varepsilon}}}
\newcommand{\stress}{{\boldsymbol{\sigma}}}
\renewcommand{\div}{{\boldsymbol{\nabla}}}

\newcommand {\cA} {{\cal A}}
\newcommand {\cB} {{\cal B}}
\newcommand {\cC} {{\cal C}}
\newcommand {\cD} {{\cal D}}
\newcommand {\cE} {{\cal E}}
\newcommand {\cK} {{\cal K}}
\newcommand {\cL} {{\cal L}}
\newcommand {\cP} {{\cal P}}
\newcommand {\cQ} {{\cal Q}}
\newcommand {\cR} {{\cal R}}
\newcommand {\cV} {{\cal V}}
\newcommand {\cW} {{\cal W}}
\newcommand {\CC} {{\cal C}}
\newcommand {\CD} {{\cal D}}
\newcommand {\CH} {{\cal H}}
\newcommand {\CS} {{\cal S}}
\newcommand {\CU} {{\cal U}}
\newcommand {\CY} {{\cal Y}}



\newcommand{\bzero}{\mathbf{0}}
\newcommand{\ba}{\mathbf{a}}
\newcommand{\bb}{\mathbf{b}}
\newcommand{\bc}{\mathbf{c}}
\newcommand{\bd}{\mathbf{d}}
\newcommand{\be}{\mathbf{e}}
\newcommand{\bg}{\mathbf{g}}
\newcommand{\bh}{\mathbf{h}}
\newcommand{\bl}{\mathbf{l}}
\newcommand{\bn}{\mathbf{n}}
\newcommand{\bp}{\mathbf{p}}
\newcommand{\bq}{\mathbf{q}}
\newcommand{\br}{\mathbf{r}}
\newcommand{\bs}{\mathbf{s}}
\newcommand{\bt}{\mathbf{t}}
\newcommand{\bu}{\mathbf{u}}
\newcommand{\bv}{\mathbf{v}}
\newcommand{\bw}{\mathbf{w}}
\newcommand{\bx}{\mathbf{x}}
\newcommand{\by}{\mathbf{y}}
\newcommand{\bz}{\mathbf{z}}
\newcommand{\bA}{{\mathbf A}}
\newcommand{\bB}{\mathbf{B}}
\newcommand{\bC}{\mathbf{C}}
\newcommand{\bD}{\mathbf{D}}
\newcommand{\bE}{\mathbf{E}}
\newcommand{\bF}{\mathbf{F}}
\newcommand{\bG}{\mathbf{G}}
\newcommand{\bH}{\mathbf{H}}
\newcommand{\bI}{\mathbf{I}}
\newcommand{\bJ}{\mathbf{J}}
\newcommand{\bK}{\mathbf{K}}
\newcommand{\bL}{\mathbf{L}}
\newcommand{\bM}{\mathbf{M}}
\newcommand{\bN}{\mathbf{N}}
\newcommand{\bO}{\mathbf{O}}
\newcommand{\bP}{\mathbf{P}}
\newcommand{\bQ}{\mathbf{Q}}
\newcommand{\bR}{\mathbf{R}}
\newcommand{\bS}{\mathbf{S}}
\newcommand{\bU}{\mathbf{U}}
\newcommand{\bV}{\mathbf{V}}
\newcommand{\bW}{\mathbf{W}}
\newcommand{\bX}{\mathbf{X}}
\newcommand{\bY}{\mathbf{Y}}
\newcommand{\bZ}{\mathbf{Z}}

\newcommand{\bgamma}{{\boldsymbol{\gamma}}}
\newcommand{\bmu}{{\boldsymbol{\mu}}}
\newcommand{\bkappa}{{\boldsymbol{\kappa}}}
\newcommand{\blambda}{{\boldsymbol{\lambda}}}
\newcommand{\bLambda}{{\boldsymbol{\Lambda}}}
\newcommand{\bpi}{{\boldsymbol{\pi}}}
\newcommand{\bPi}{{\boldsymbol{\Pi}}}
\newcommand{\btheta}{{\boldsymbol{\theta}}}
\newcommand{\bTheta}{{\boldsymbol{\Theta}}}
\newcommand{\bSigma}{{\boldsymbol{\Sigma}}}






\title{AMATH 568 Homework 2}
\author{Cade Ballew \#2120804}
\date{January 19, 2022}

\begin{document}
	
\maketitle
	
\section{Problem 1}
To show that, for each fixed $\ell\geq0$, the integral
\begin{align*}
	H(t) = \int_0^\infty e^{-x^2 - 2 t x} (tx)^\ell dx, \quad t \geq 0,
\end{align*}
is $\OO(1)$ for $t \in [0,\infty)$, first consider some special cases. If $t=0$, then $H(t)=0,$ so $|H(t)|\leq K$ for any $K>0$. Thus, going forward we only consider $t \in (0,\infty)$.\\
Now, consider the case where $\ell=0$. Then, 
\begin{align*}
|H(t)|=\int_0^\infty e^{-x^2 - 2t x} dx\leq\int_0^\infty e^{-x^2} dx=\frac{\sqrt{\pi}}{2},
\end{align*}
so if we take $K=\frac{\sqrt{\pi}}{2}$, we have that $|H(t)|<K*1$ for all $t \in [0,\infty)$, meaning that $H(t)$ is $\OO(1)$.\\
Now, consider $\ell>0$. Then, using the change of variables $u=2tx$,
\begin{align*}
	|H(t)| = \int_0^\infty e^{-x^2 - 2 t x} (tx)^\ell dx=\int_0^\infty e^{-\left(\frac{u}{2t}\right)^2 - u} \left(\frac{u}{2}\right)^\ell \frac{du}{2t}\leq\frac{1}{2^{\ell+1}}\int_0^\infty e^{-u}u^{\ell-1}\frac{u}{t}e^{-\left(\frac{u}{2t}\right)^2}du.
\end{align*}
Now, because we consider $t\neq0$, let $v=\frac{u}{t}$. Then,
\[
\frac{u}{t}e^{-\left(\frac{u}{2t}\right)^2}=ve^{-v^2/4}\leq\sqrt{2}e^{-1/2}=\sqrt{\frac{2}{e}}<1,
\] 
because $ve^{-v^2/4}$ is maximized at $v=\sqrt{2}$ by elementary calculus. Thus, 
\[
|H(t)|<\frac{1}{2^{\ell+1}}\int_0^\infty e^{-u}u^{\ell-1}=\frac{\Gamma(\ell)}{2^{\ell+1}}.
\]
This is constant with respect to $t$, so we simply take $K=\frac{\Gamma(\ell)}{2^{\ell+1}}$, and then $|H(t)|<K*1$ for all $t \in [0,\infty)$, meaning that $H(t)$ is $\OO(1)$.

\section{Problem 2}
We wish to apply Watson's lemma to derive an asymptotic expansion of
\begin{align*}
	F(\lambda)  = \int_0^\infty e^{-\lambda t} \frac{\sin t}{t^{3/2}} d t, \quad \lambda > 0, \quad \lambda \to \infty
\end{align*}
by taking $\phi(t)=\frac{\sin t}{t^{3/2}}$ and $g(t)=\frac{\sin t}{t}$ so that $\phi(t)=t^\sigma g(t)$ where $\sigma=-1/2>-1$. To do this, we need to show that $\phi(t)$ is absolutely integrable on $[0,\infty)$ and that $g(t)$ is infinitely differentiable in a neighborhood of $t=0$. To see that $\phi(t)$ is absolutely integrable, write 
\[
\int_0^\infty|\phi(t)|dt=\underbrace{\int_0^1\left|\frac{\sin t}{t^{3/2}}\right|dt}_{I_1}+\underbrace{\int_1^\infty\left|\frac{\sin t}{t^{3/2}}\right|dt}_{I_2}. 
\]
Using the fact that $|\sin(t)|=\sin(t)\leq t$ on $[0,1]$, 
\[
I_1\leq\int_0^1\frac{t}{t^{3/2}}dt=\int_0^1\frac{dt}{\sqrt{t}}=2.
\]
Using the fact that $|\sin{t}|\leq1$,
\[
I_2\leq\int_1^\infty\frac{dt}{t^{3/2}}=2.
\]
Thus, 
\[
\int_0^\infty|\phi(t)|dt=I_1+I_2\leq4<\infty.
\]
To see that $g(t)$ is infinitely differentiable around $t=0$, we use the Taylor series centered at $t=0$ for the sine function to write
\[
g(t)=\frac{1}{t}\sum_{j=0}^\infty(-1)^j\frac{t^{2j+1}}{(2j+1)!}=\sum_{j=0}^\infty(-1)^j\frac{t^{2j}}{(2j+1)!}=1-\frac{t^2}{3!}+\frac{t^4}{5!}-\ldots.
\]
Since this Taylor series for $\sin(t)$ holds for all $t\in\real$ and this expression for $g(t)$ is a polynomial, $g(t)$ is clearly infinitely differentiable around $t=0$. Furthermore, we can compute $g^{(j)}(0)$ by noting that it is precisely the constant term of the series for $g^{(j)}(t)$. This will be zero for odd $j$ since our series contains only even powers of $t$. Thus, we can write
\[
g^{(2j)}(0)=(-1)^j(2j)!\frac{1}{(2j+1)!}=\frac{(-1)^j}{2j+1}.
\]
With this in hand, we apply Watson's lemma and use our above expression to conclude that 
\begin{align*}
F(\lambda)\sim\sum_{n=0}^{\infty}\frac{g^{(n)}(0)\Gamma(\sigma+n+1)}{n!\lambda^{\sigma+n+1}}=\sum_{j=0}^{\infty}\frac{(-1)^j}{2j+1}\frac{\Gamma(2j+1/2)}{(2j)!\lambda^{2j+1/2}}=\sum_{j=0}^{\infty}\frac{(-1)^j\Gamma(2j+1/2)}{(2j+1)!\lambda^{2j+1/2}},
\end{align*}
as $\lambda\to\infty$ with $\lambda>0$ where we have reindexed $n\to2j$.
	
\section{Problem 3}
To derive an asymptotic expansion of \begin{align*}
	u(x,t) = \frac{1}{2 \pi} \int_{-\infty}^\infty e^{i k x - k^2 t} \hat f(k) d k, \quad \hat f(k) = \int_{-\infty}^\infty e^{-i k x} f(x) d x
\end{align*}
for fixed $x$, let $\phi(k)=e^{ikx}\hat{f}(k)$, $a=-\infty$, and $b=\infty$ and consider the generalization of Watson's lemma on page 6 of part 2 of the course notes where $f$ and $\hat f$ are assumed to decay rapidly enough so that $\phi(k)$ is absolutely integrable on $(-\infty,\infty)$ and $\phi(k)$ has an infinite number of continuous derivatives in a neighborhood of $k=0$. Then, the result on page 8 of part 2 of the notes gives that 
\[
u(x,t)\sim\frac{1}{2\pi}\sqrt{\frac{\pi}{t}}\sum_{j=0}^\infty\frac{\phi^{(2j)}(0)}{t^j}\frac{1}{2^{2j}j!}
\]
as $t\to\infty$, $t>0$. Now, we compute derivatives of $\phi(k)$. By the product rule, 
\begin{align*}
\phi'(k)&=ixe^{ikx}\int_{-\infty}^\infty e^{-ikx}f(x)dx+e^{ikx}\int_{-\infty}^\infty \frac{d}{dk}(e^{-ikx}f(x))dx\\&=
ixe^{ikx}\int_{-\infty}^\infty e^{-ikx}f(x)dx-e^{ikx}\int_{-\infty}^\infty ixe^{-ikx}f(x)dx
\end{align*}
Again, by the product rule,
\begin{align*}
\phi''(k)&=(ix)^2e^{ikx}\int_{-\infty}^\infty e^{-ikx}f(x)dx-ixe^{ikx}\int_{-\infty}^\infty ix e^{-ikx}f(x)dx\\&-ixe^{ikx}\int_{-\infty}^\infty ix e^{-ikx}f(x)dx+e^{ikx}\int_{-\infty}^\infty (ix)^2e^{-ikx}f(x)dx\\&=
(ix)^2e^{ikx}\int_{-\infty}^\infty e^{-ikx}f(x)dx-2ixe^{ikx}\int_{-\infty}^\infty ix e^{-ikx}f(x)dx+e^{ikx}\int_{-\infty}^\infty (ix)^2e^{-ikx}f(x)dx.
\end{align*}
Thus, we can compute
\[
\phi(0)=\hat{f}(0)=\int_{-\infty}^\infty f(x)dx
\]
and
\begin{align*}
\phi''(0)&=(ix)^2\int_{-\infty}^\infty f(x)dx-2ix\int_{-\infty}^\infty ix f(x)dx+\int_{-\infty}^\infty (ix)^2f(x)dx\\&=
-x^2\int_{-\infty}^\infty f(x)dx+2x\int_{-\infty}^\infty xf(x)dx-\int_{-\infty}^\infty x^2f(x)dx.
\end{align*}
Now, we can simply plug in $j=0,1$ to find the first two nonzero terms of our expansion. Namely, the first term is given by
\[
\frac{1}{2\pi}\sqrt{\frac{\pi}{t}}\phi(0)=\frac{1}{2\pi}\sqrt{\frac{\pi}{t}}\int_{-\infty}^\infty f(x)dx
\]
and the second term is given by
\begin{align*}
\frac{1}{2\pi}\sqrt{\frac{\pi}{t}}\frac{\phi''(0)}{t}\frac{1}{2^{2}1!}=
\frac{1}{2\pi}\sqrt{\frac{\pi}{t}}\frac{1}{4t}\left(-x^2\int_{-\infty}^\infty f(x)dx+2x\int_{-\infty}^\infty xf(x)dx-\int_{-\infty}^\infty x^2f(x)dx\right).
\end{align*}
	
\section{Problem 4}
To compute the asymptotic expansion of
\begin{align*}
	u(x,t) = \frac{1}{2 \pi} \int_{-\infty}^\infty e^{i k x - k^4 t} \hat f(k) d k, \quad \hat f(k) = \int_{-\infty}^\infty e^{-i k x} f(x) d x,
\end{align*}
we derive an additional generalization of Watson's lemma based on the generalization used in problem 3. Namely, we consider 
\[
F(\lambda)=\int_a^b e^{-\lambda t^4}\phi(t)dt
\]
where $a<0<b$, $\lambda>0$, and $\phi(t)$ is absolutely integrable and has an infinite number of continuous derivatives in a neighborhood of $t=0$.\\
Steps 1 and 2 of our derivation are essentially the same as their counterparts on page 7 of part 2 of the course notes. 	
	
\end{document}
