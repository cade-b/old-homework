\documentclass{article}
\usepackage[utf8]{inputenc}
\usepackage{listings}
\usepackage{multimedia} % to embed movies in the PDF file
\usepackage{graphicx}
\usepackage{comment}
\usepackage[english]{babel}
\usepackage{amsmath}
\usepackage{amsfonts}
\usepackage{subfigure}
\usepackage{wrapfig}
\usepackage{multirow}
\usepackage{tikz}
\usepackage{verbatim}
%!TEX root = main.tex



\newcommand{\eref}[1]{\mbox{\rm(\ref{#1})}}
\newcommand{\tref}[1]{\mbox{\rm\ref{#1}}}
\newcommand{\set}[2]{\left\{ #1 \; : \; #2 \right\} }
\newcommand{\deq}{\raisebox{0pt}[1ex][0pt]{$\stackrel{\scriptscriptstyle{\rm def}}{{}={}}$}}

\newcommand {\DS} {\displaystyle}

\newcommand{\real}{\mathbb{R}}



\newcommand {\half} {\mbox{$\frac{1}{2}$}}
\newcommand{\force}{{\mathbf{f}}}
\newcommand{\strain}{{\boldsymbol{\varepsilon}}}
\newcommand{\stress}{{\boldsymbol{\sigma}}}
\renewcommand{\div}{{\boldsymbol{\nabla}}}

\newcommand {\cA} {{\cal A}}
\newcommand {\cB} {{\cal B}}
\newcommand {\cC} {{\cal C}}
\newcommand {\cD} {{\cal D}}
\newcommand {\cE} {{\cal E}}
\newcommand {\cK} {{\cal K}}
\newcommand {\cL} {{\cal L}}
\newcommand {\cP} {{\cal P}}
\newcommand {\cQ} {{\cal Q}}
\newcommand {\cR} {{\cal R}}
\newcommand {\cV} {{\cal V}}
\newcommand {\cW} {{\cal W}}
\newcommand {\CC} {{\cal C}}
\newcommand {\CD} {{\cal D}}
\newcommand {\CH} {{\cal H}}
\newcommand {\CS} {{\cal S}}
\newcommand {\CU} {{\cal U}}
\newcommand {\CY} {{\cal Y}}



\newcommand{\bzero}{\mathbf{0}}
\newcommand{\ba}{\mathbf{a}}
\newcommand{\bb}{\mathbf{b}}
\newcommand{\bc}{\mathbf{c}}
\newcommand{\bd}{\mathbf{d}}
\newcommand{\be}{\mathbf{e}}
\newcommand{\bg}{\mathbf{g}}
\newcommand{\bh}{\mathbf{h}}
\newcommand{\bl}{\mathbf{l}}
\newcommand{\bn}{\mathbf{n}}
\newcommand{\bp}{\mathbf{p}}
\newcommand{\bq}{\mathbf{q}}
\newcommand{\br}{\mathbf{r}}
\newcommand{\bs}{\mathbf{s}}
\newcommand{\bt}{\mathbf{t}}
\newcommand{\bu}{\mathbf{u}}
\newcommand{\bv}{\mathbf{v}}
\newcommand{\bw}{\mathbf{w}}
\newcommand{\bx}{\mathbf{x}}
\newcommand{\by}{\mathbf{y}}
\newcommand{\bz}{\mathbf{z}}
\newcommand{\bA}{{\mathbf A}}
\newcommand{\bB}{\mathbf{B}}
\newcommand{\bC}{\mathbf{C}}
\newcommand{\bD}{\mathbf{D}}
\newcommand{\bE}{\mathbf{E}}
\newcommand{\bF}{\mathbf{F}}
\newcommand{\bG}{\mathbf{G}}
\newcommand{\bH}{\mathbf{H}}
\newcommand{\bI}{\mathbf{I}}
\newcommand{\bJ}{\mathbf{J}}
\newcommand{\bK}{\mathbf{K}}
\newcommand{\bL}{\mathbf{L}}
\newcommand{\bM}{\mathbf{M}}
\newcommand{\bN}{\mathbf{N}}
\newcommand{\bO}{\mathbf{O}}
\newcommand{\bP}{\mathbf{P}}
\newcommand{\bQ}{\mathbf{Q}}
\newcommand{\bR}{\mathbf{R}}
\newcommand{\bS}{\mathbf{S}}
\newcommand{\bU}{\mathbf{U}}
\newcommand{\bV}{\mathbf{V}}
\newcommand{\bW}{\mathbf{W}}
\newcommand{\bX}{\mathbf{X}}
\newcommand{\bY}{\mathbf{Y}}
\newcommand{\bZ}{\mathbf{Z}}

\newcommand{\bgamma}{{\boldsymbol{\gamma}}}
\newcommand{\bmu}{{\boldsymbol{\mu}}}
\newcommand{\bkappa}{{\boldsymbol{\kappa}}}
\newcommand{\blambda}{{\boldsymbol{\lambda}}}
\newcommand{\bLambda}{{\boldsymbol{\Lambda}}}
\newcommand{\bpi}{{\boldsymbol{\pi}}}
\newcommand{\bPi}{{\boldsymbol{\Pi}}}
\newcommand{\btheta}{{\boldsymbol{\theta}}}
\newcommand{\bTheta}{{\boldsymbol{\Theta}}}
\newcommand{\bSigma}{{\boldsymbol{\Sigma}}}






\title{AMATH 568 Homework 1}
\author{Cade Ballew \#2120804}
\date{January 12, 2022}

\begin{document}

\maketitle

\section{Problem 1}
\subsection{Part a}
Taking $a\in\mathbb{C}$, to show that $\displaystyle \frac{1}{z -a} = \OO\left(\frac{1}{1 + |z|} \right)$ as $z \to \infty$, $z \in \mathbb C$, we wish to find constants $K,M>0$ such that for $|z|>M$, $\left|\frac{1}{z -a}\right|\leq K\left|\frac{1}{1 + |z|}\right|=K\frac{1}{1 + |z|}$. Take $M=2|a|$ in the case where $a\neq0$. Then, by the reverse triangle inequality, when $|z|>M$,
\[
|z-a|\leq\left||z|-|a|\right|=|z|-|a|>2|a|-|a|=|a|.
\]
Thus,
\[
\left|\frac{1}{z -a}\right|<\frac{1}{|a|}=\frac{1+|z|}{|a|}\frac{1}{1+|z|}.
\]
Now, note that when $|z|>M$,
\[
\frac{1+|z|}{|a|}>\frac{1+M}{|a|}=\frac{1+2|a|}{|a|}=\frac{1}{|a|}+2.
\]
Thus, if we take $K\geq\frac{1}{|a|}+2$, we have found such an $M$ and $K$, so the definition is satisfied.\\
Now, consider the case where $a=0$. Then, we can take $M=1$, meaning that for $z>M$, 
\[
\left|\frac{1}{z-a}\right|=\frac{1}{|z|}<1=(1+|z|)\frac{1}{1+|z|}.
\]
Also,
\[
1+|z|>1+M=2,
\]
so we simply need to take $K=2$, and $\left|\frac{1}{z -a}\right|\leq K\frac{1}{1 + |z|}$ for $z>M$.

\subsection{Part b}
Now, we wish to show that $\displaystyle \frac{1}{\mathrm{dist}(z,[a,b])} = \OO\left(\frac{1}{1 + |z|} \right)$ as $z \to \infty$, $z \in \mathbb C$ with $a,b\in\real$. First, note that if $a=b$, $\mathrm{dist}(z,[a,b])=|z-a|$, so this amounts to showing that $\displaystyle \frac{1}{z -a} = \OO\left(\frac{1}{1 + |z|} \right)$ as $z \to \infty$, $z \in \mathbb C$ which was done in part a. Now, assume that $a\neq b$. Now, consider a region $D$ defined by revolving $[a,b]$ in the complex plane. Namely, if $a$ and $b$ have the same sign, $D$ is an annulus defined by $\min\{|a|,|b|\}\leq |z|\leq\max\{|a|,|b|\}$ and if they have different signs, $D$ is a disk defined by $|z|\leq\max\{|a|,|b|\}$. Regardless, if $|z|\geq\max\{|a|,|b|\}$, $\mathrm{dist}(z,D)=|z|-\max\{|a|,|b|\}$. Now, note that $[a,b]\subset D$, so clearly, $\mathrm{dist}(z,[a,b])\geq\mathrm{dist}(z,D)$ for any $z\in \compl$. Let $A=\max\{|a|,|b|\}$ and take $M=2A$. Then, if $|z|>M$, by the above and the reverse triangle inequality, 
\[
\mathrm{dist}(z,[a,b])\geq\mathrm{dist}(z,D)=|z-A|\geq\left||z|-|A|\right|=|z|-A>2A-A=A.
\]
Thus, if $|z|>M$,
\[
\left|\frac{1}{\mathrm{dist}(z,[a,b])}\right|=\frac{1}{\mathrm{dist}(z,[a,b])}<\frac{1}{A}=\frac{1+|z|}{A}\frac{1}{1+|z|}.
\]
Now, we simply take
\[
K\geq\frac{1+|z|}{A}>\frac{1+2A}{A}=\frac{1}{A}+2=\frac{1}{\max\{|a|,|b|\}}+2
\]
to get that 
\[
\left|\frac{1}{\mathrm{dist}(z,[a,b])}\right|\leq K\frac{1}{1+|z|}
\]
when $|z|>M$. Thus, $\displaystyle \frac{1}{\mathrm{dist}(z,[a,b])} = \OO\left(\frac{1}{1 + |z|} \right)$ as $z \to \infty$, $z \in \mathbb C$.

\section{Problem 2 (Exercise 1.8)}
Suppose that $\mu$ is a continuous parameter and that for each $\mu \in [0,1]$ we have that $f(z,\mu) = \OO(g(z,\mu))$ as $z \to z_0$ from $D$. Despite the proof on page 21 of the text, it is not true that if the integrals exist in the Riemann sense for all $z$ close enough to $z_0$ then
  \begin{align*}
    \int_0^1 f(z,\mu) d \mu = \OO \left( \int_0^1 |g(z,\mu)| d \mu \right) \quad \text{as $z \to z_0$ from $D$}.
  \end{align*} 
Even though both integrals can be written as Riemann sums, 
\[
\int_0^1 f(z,\mu) d \mu=\lim_{N\to\infty}\sum_{i=0}^N\Delta\mu f(z,\mu_i),\quad\\
\int_0^1 g(z,\mu) d \mu=\lim_{N\to\infty}\sum_{i=0}^N\Delta\mu g(z,\mu_i)\\
\]
this needs to be true in the limit, requiring an infinite number of terms, so we may not be able to find the minimum of $\delta_1,\delta_2,\ldots$ and the maximum of $K_1,K_2,\ldots$ as is done in the proof. If we include the additional hypothesis that there are positive real numbers $\delta, K$ such that $|f(z,\mu)|\leq K|g(z,\mu)|$ for all $0<|z-z_0|<\delta$ and all $\mu\in[0,1]$, then this is no longer an issue and the proof is valid with our new definitions of $\delta$ and $K$. \\
As a counterexample, consider $f(z,\mu)=\mu^{-1}e^{-z/\mu}$ and $g(z,\mu)=e^{-z/\mu}$ as $z\to0$, $z>0$. Clearly, for each given $\mu\in(0,1]$, $f(z,\mu)=\OO(g(z,\mu))$, because $|\mu^{-1}e^{-z/\mu}|\leq K|e^{-z/\mu}|$ if $K\geq\mu^{-1}$. First, note that 
\[
\left|\int_0^1 g(z,\mu) d \mu\right|=\left|\int_0^1 e^{-z/\mu} d \mu\right|\leq\max_{\mu\in(0,1]}e^{-z/\mu}=e^{-z}\leq1
\]
since $z>0$. However, using the substitution $u=z/\mu$ which gives that $d\mu=\frac{-\mu}{u}du$ and taking $z<1$, 
\begin{align*}
\int_0^1 f(z,\mu) d \mu&=\int_0^1 \mu^{-1}e^{-z/\mu} d \mu=\int_z^\infty\frac{e^{-u}}{u}du\\&=\int_1^\infty\frac{e^{-u}}{u}du+\int_z^1\frac{e^{-u}-1}{u}du+\int_z^1\frac{du}{u}.
\end{align*}
The last integral in this expression for the integral of $f$ is unbounded (tends to infinity) as $z\to0$, so there is no possible way we can find a $\delta,K>0$ such that 
\[
\left|\int_0^1 f(z,\mu) d \mu\right|\leq K\left|\int_0^1 g(z,\mu) d \mu\right|\leq K
\]
when $z<\delta$ since we can always increase $\int_z^1\frac{du}{u}$ by taking $z$ to be closer to 0.

\section{Problem 3 (Exercise 1.11)}
Consider the function $e^{-2z}$ for $|\arg z|<\pi/4$. This implies that $|\arg 2z|<\pi/2$, so $2z$ is in the right half-plane or, equivalently, $\Re(2z)>0$. Now, fix $\epsilon>0$ and observe that 
\[
|e^{-2z}|=|e^{-2\Re(z)}||e^{-2\Im(z)}|=|e^{-2\Re(z)}|=e^{-2\Re(z)}.
\]
Because we restrict $z$ to the sector $|\arg z|<\pi/4$, we know that $\Re(z)\to\infty$ as $z\to\infty$. We wish to show that $e^{-2z}$ is $\oo(1)$, so we wish to find an $M>0$ such that $|e^{-2z}|<\epsilon$ whenever $|z|>M$. However, this is precisely the definition of a limit at infinity in the complex plane, so we need only show that $\lim_{z\to\infty,|\arg z|<\pi/4}e^{-2z}=0$. However, 
\[
\lim_{z\to\infty,|\arg z|<\pi/4}|e^{-2z}|=\lim_{z\to\infty,|\arg z|<\pi/4}|e^{-2\Re(z)}|=0
\]
because $\Re(z)\to\infty$ here, so $\lim_{z\to\infty,|\arg z|<\pi/4}e^{-2z}=0$ is indeed true. Thus, $e^{-2z}$ is $\oo(1)$, meaning that as $z\to\infty$ in this sector
\[
2\cosh(z)=e^z+e^{-z}=e^z(1+e^{-2z})=e^z(1+\oo(1)),
\]
so $2\cosh(z)$ is approximated by $e^z$ in the sector $|\arg z|<\pi/4$ in the sense of small relative error. 
This is also true in the sense of absolute error, because the sector is contained in the right half-plane, $\Re(z)>0$, meaning that 
\begin{align*}
\lim_{z\to\infty,|\arg z|<\pi/4}|e^{-z}|&=\lim_{z\to\infty,|\arg z|<\pi/4}|e^{-\Re(z)}||e^{-\Im(z)}|\\&=\lim_{z\to\infty,|\arg z|<\pi/4}|e^{-\Re(z)}|=\lim_{z\to\infty,|\arg z|<\pi/4}e^{-\Re(z)}=0,
\end{align*}
so $e^{-z}$ is $\oo(1)$ as $z\to\infty$ in the sector $|\arg z|<\pi/4$, and $f(z)=e^z+\oo(1)$ as $z\to\infty$ in the sector $|\arg z|<\pi/4$.\\
Now, consider the function $e^{2z}$ in the sector $|\arg (-z)|<\pi/4$. Then, $|\arg(-2z)|<\pi/2$, so $-2z$ is in the right half-plane or, equivalently, $\Re(2z)<0$. Now observe that 
\[
|e^{2z}|=|e^{2\Re(z)}||e^{2\Im(z)}|=|e^{2\Re(z)}|=e^{2\Re(z)}.
\]
Because we restrict $z$ to the sector $|\arg (-z)|<\pi/4$, we know that $\Re(z)\to-\infty$ as $z\to\infty$. Thus, 
\[
\lim_{z\to\infty,|\arg (-z)|<\pi/4}|e^{2z}|=\lim_{z\to\infty,|\arg (-z)|<\pi/4}|e^{2\Re(z)}|=0
\]
because $\Re(z)\to-\infty$ here, so $\lim_{z\to\infty,|\arg (-z)|<\pi/4}e^{2z}=0$, meaning that $e^{2z}$ is $\oo(1)$, and as $z\to\infty$ in this sector,
\[
2\cosh(z)=e^z+e^{-z}=e^{-z}(1+e^{2z})=e^{-z}(1+\oo(1)),
\]
so $2\cosh(z)$ is approximated by $e^{-z}$ in the sector $|\arg (-z)|<\pi/4$ in the sense of small relative error. Again, this is also true in the sense of absolute error, because the sector is contained in the left half plane, $\Re(z)<0$, meaning that 
\begin{align*}
\lim_{z\to\infty,|\arg (-z)|<\pi/4}|e^{z}|&=\lim_{z\to\infty,|\arg (-z)|<\pi/4}|e^{\Re(z)}||e^{\Im(z)}|\\&=\lim_{z\to\infty,|\arg (-z)|<\pi/4}|e^{\Re(z)}|=\lim_{z\to\infty,|\arg (-z)|<\pi/4}e^{\Re(z)}=0,
\end{align*}
so $e^{z}$ is $\oo(1)$ as $z\to\infty$ in the sector $|\arg (-z)|<\pi/4$, and $f(z)=e^{-z}+\oo(1)$ as $z\to\infty$ in the sector $|\arg (-z)|<\pi/4$.

\section{Problem 4}
By definition, for $z\in\compl$,
\begin{align*}
\erfc(z)=\frac{2}{\sqrt{\pi}}\int_z^\infty e^{-s^2}ds=\frac{2}{\sqrt{\pi}}\left(\int_z^0 e^{-s^2}ds+\int_0^\infty e^{-s^2}ds\right).
\end{align*}
Additionally, using the change of variables $s\to-s$, 
\begin{align*}
\erfc(-z)&=\frac{2}{\sqrt{\pi}}\int_{-z}^\infty e^{-s^2}ds=\frac{2}{\sqrt{\pi}}\left(\int_{-z}^0 e^{-s^2}ds+\int_0^\infty e^{-s^2}ds\right)\\&=
\frac{2}{\sqrt{\pi}}\left(\int_{z}^0 e^{-s^2}(-ds)+\int_0^{-\infty} e^{-s^2}(-ds)\right)\\&=
\frac{2}{\sqrt{\pi}}\left(\int_{0}^z e^{-s^2}ds+\int_{-\infty}^0 e^{-s^2}ds\right).
\end{align*}
Thus,
\begin{align*}
\erfc(z)+\erfc(-z)&=\frac{2}{\sqrt{\pi}}\left(\int_z^0 e^{-s^2}ds+\int_0^\infty e^{-s^2}ds+\int_{0}^z e^{-s^2}ds+\int_{-\infty}^0 e^{-s^2}ds\right)\\&=
\frac{2}{\sqrt{\pi}}\int_{-\infty}^\infty e^{-s^2}ds=\frac{2}{\sqrt{\pi}}\sqrt{\pi}=2.
\end{align*}
From class, we have that when $\Re(z)\geq0$ and as $|z|\to\infty$, 
\[
\erfc(z)\sim \frac{e^{-z^2}}{\sqrt{\pi}}\sum_{m=0}^\infty(-1)^m\frac{\left(\frac{1}{2}\right)_m}{z^{2m+1}},
\]
so the above identity gives that 
\[
\erfc(-z)\sim 2-\frac{e^{-z^2}}{\sqrt{\pi}}\sum_{m=0}^\infty(-1)^m\frac{\left(\frac{1}{2}\right)_m}{z^{2m+1}}
\]
when $\Re(z)\geq0$ and as $|z|\to\infty$. To find an expression for $\erfc(z)$ when $\Re(z)<0$, we simply swap $z$ and $-z$ to get that 
\[
\erfc(z)\sim 2-\frac{e^{-(-z)^2}}{\sqrt{\pi}}\sum_{m=0}^\infty(-1)^m\frac{\left(\frac{1}{2}\right)_m}{(-z)^{2m+1}}=2+\frac{e^{-z^2}}{\sqrt{\pi}}\sum_{m=0}^\infty(-1)^m\frac{\left(\frac{1}{2}\right)_m}{z^{2m+1}}
\]
for $\Re(z)<0$ as $|z|\to\infty$.

\section{Problem 5}
Consider the function defined on $\real$
\[
f(z)=\begin{cases}
e^{-1/z},\quad z>0\\
0, \quad z\leq0
\end{cases}
\]
at the point $z_0=0$. The derivative of $e^{-1/z}$ is $z^{-2}e^{-1/z}$, so in general, the product rule tells us that the $n$th derivative will be some linear combination of products $z^{-k}e^{-1/z}$, where $k\in\mathbb{Z}$. However, from the right, $e^{-1/z}$ tends to zero faster than any polynomial, so
\[
\lim_{z\to0+}\left(e^{-1/z}\right)^{(k)}=0
\]
for all $k$. Thus, our function is infinitely differentiable at $0$ since this matches the derivative from the left at 0. However, this implies that its Taylor series centered at $z_0=0$ is identically 0, but $f(z)$ takes nonzero values arbitraily close to $z_0=0$ when $z>0$, as $e^{-1/z}\neq0$ for any $z\in\real$. 

\end{document}
