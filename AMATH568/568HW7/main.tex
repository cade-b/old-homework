\documentclass{article}
\usepackage[utf8]{inputenc}
\usepackage{listings}
\usepackage{multimedia} % to embed movies in the PDF file
\usepackage{graphicx}
\usepackage{comment}
\usepackage[english]{babel}
\usepackage{amsmath}
\usepackage{amsfonts}
\usepackage{subfigure}
\usepackage{wrapfig}
\usepackage{multirow}
\usepackage{verbatim}
%!TEX root = main.tex



\newcommand{\eref}[1]{\mbox{\rm(\ref{#1})}}
\newcommand{\tref}[1]{\mbox{\rm\ref{#1}}}
\newcommand{\set}[2]{\left\{ #1 \; : \; #2 \right\} }
\newcommand{\deq}{\raisebox{0pt}[1ex][0pt]{$\stackrel{\scriptscriptstyle{\rm def}}{{}={}}$}}

\newcommand {\DS} {\displaystyle}

\newcommand{\real}{\mathbb{R}}



\newcommand {\half} {\mbox{$\frac{1}{2}$}}
\newcommand{\force}{{\mathbf{f}}}
\newcommand{\strain}{{\boldsymbol{\varepsilon}}}
\newcommand{\stress}{{\boldsymbol{\sigma}}}
\renewcommand{\div}{{\boldsymbol{\nabla}}}

\newcommand {\cA} {{\cal A}}
\newcommand {\cB} {{\cal B}}
\newcommand {\cC} {{\cal C}}
\newcommand {\cD} {{\cal D}}
\newcommand {\cE} {{\cal E}}
\newcommand {\cK} {{\cal K}}
\newcommand {\cL} {{\cal L}}
\newcommand {\cP} {{\cal P}}
\newcommand {\cQ} {{\cal Q}}
\newcommand {\cR} {{\cal R}}
\newcommand {\cV} {{\cal V}}
\newcommand {\cW} {{\cal W}}
\newcommand {\CC} {{\cal C}}
\newcommand {\CD} {{\cal D}}
\newcommand {\CH} {{\cal H}}
\newcommand {\CS} {{\cal S}}
\newcommand {\CU} {{\cal U}}
\newcommand {\CY} {{\cal Y}}



\newcommand{\bzero}{\mathbf{0}}
\newcommand{\ba}{\mathbf{a}}
\newcommand{\bb}{\mathbf{b}}
\newcommand{\bc}{\mathbf{c}}
\newcommand{\bd}{\mathbf{d}}
\newcommand{\be}{\mathbf{e}}
\newcommand{\bg}{\mathbf{g}}
\newcommand{\bh}{\mathbf{h}}
\newcommand{\bl}{\mathbf{l}}
\newcommand{\bn}{\mathbf{n}}
\newcommand{\bp}{\mathbf{p}}
\newcommand{\bq}{\mathbf{q}}
\newcommand{\br}{\mathbf{r}}
\newcommand{\bs}{\mathbf{s}}
\newcommand{\bt}{\mathbf{t}}
\newcommand{\bu}{\mathbf{u}}
\newcommand{\bv}{\mathbf{v}}
\newcommand{\bw}{\mathbf{w}}
\newcommand{\bx}{\mathbf{x}}
\newcommand{\by}{\mathbf{y}}
\newcommand{\bz}{\mathbf{z}}
\newcommand{\bA}{{\mathbf A}}
\newcommand{\bB}{\mathbf{B}}
\newcommand{\bC}{\mathbf{C}}
\newcommand{\bD}{\mathbf{D}}
\newcommand{\bE}{\mathbf{E}}
\newcommand{\bF}{\mathbf{F}}
\newcommand{\bG}{\mathbf{G}}
\newcommand{\bH}{\mathbf{H}}
\newcommand{\bI}{\mathbf{I}}
\newcommand{\bJ}{\mathbf{J}}
\newcommand{\bK}{\mathbf{K}}
\newcommand{\bL}{\mathbf{L}}
\newcommand{\bM}{\mathbf{M}}
\newcommand{\bN}{\mathbf{N}}
\newcommand{\bO}{\mathbf{O}}
\newcommand{\bP}{\mathbf{P}}
\newcommand{\bQ}{\mathbf{Q}}
\newcommand{\bR}{\mathbf{R}}
\newcommand{\bS}{\mathbf{S}}
\newcommand{\bU}{\mathbf{U}}
\newcommand{\bV}{\mathbf{V}}
\newcommand{\bW}{\mathbf{W}}
\newcommand{\bX}{\mathbf{X}}
\newcommand{\bY}{\mathbf{Y}}
\newcommand{\bZ}{\mathbf{Z}}

\newcommand{\bgamma}{{\boldsymbol{\gamma}}}
\newcommand{\bmu}{{\boldsymbol{\mu}}}
\newcommand{\bkappa}{{\boldsymbol{\kappa}}}
\newcommand{\blambda}{{\boldsymbol{\lambda}}}
\newcommand{\bLambda}{{\boldsymbol{\Lambda}}}
\newcommand{\bpi}{{\boldsymbol{\pi}}}
\newcommand{\bPi}{{\boldsymbol{\Pi}}}
\newcommand{\btheta}{{\boldsymbol{\theta}}}
\newcommand{\bTheta}{{\boldsymbol{\Theta}}}
\newcommand{\bSigma}{{\boldsymbol{\Sigma}}}






\title{AMATH 568 Homework 7}
\author{Cade Ballew \#2120804}
\date{February 25, 2022}

\begin{document}
	
\maketitle
	
\section{Problem 1}
Considering the ODE
  \begin{align*}  
    y''(x;\lambda) + \left[ \lambda \cos x - \lambda^2 \sin x \right] y(x;\lambda) = 0, \quad \lambda \to \infty,
  \end{align*}
we write it in the form 
  \begin{align*}  
    y''(x;\lambda) + f(x) y(x;\lambda) = 0, \quad \lambda \to \infty,
  \end{align*}
where 
\[
f(x)=\lambda^2\sum_{n=0}^\infty f_n(x)\lambda^{-n}
\]
with $f_0(x)=-\sin x$, $f_1(x)=\cos{x}$, and $f_j(x)=0$ for $j\geq2$. Then, we can apply the WKB method to find oscillatory and exponential solutions using (7.37) and (7.38) in the text. 
\begin{align*}
y_{\text{osc}}^\pm(x)=\frac{1}{(-\sin x)^{1/4}}\exp\left(\pm i\lambda\int_{x_0}^x\sqrt{-\sin{s}}ds\pm i\frac{1}{2}\int_{x_0}^x\frac{\cos{s}}{\sqrt{-\sin{s}}}ds\right)(1+\oo(1))
\end{align*}
\begin{align*}
y_{\text{exp}}^\pm(x)=\frac{1}{|\sin{x}|^{1/4}}\exp\left(\pm\lambda\int_{x_0}^x\sqrt{|\sin s|}ds\mp\frac{1}{2}\int_{x_0}^x\frac{\cos{s}}{\sqrt{|\sin{s}|}}ds\right)(1+\oo(1)).
\end{align*}
We know that the first expansion is valid in a region $[\alpha,\beta]$ where $f_0(x)=-\sin{x}>0$ and sufficiently far away from the turning points, i.e. for some region inside the interval such that $-\pi<\alpha\leq\beta<0$ if we want to look at the region near $x=0$ before the next sign change. Thus, we also take $x_0\in[\alpha,\beta]$. Similarly, the second expansion is valid in some region inside the region where $f_0(x)=-\sin{x}<0$ until the next turning point, i.e. in some region $[\alpha',\beta']$ inside the interval $(0,\pi)$; we also take $x_0\in[\alpha',\beta']$.

\section{Problem 2}
\subsection{Part a}
Defining \begin{align*}
      H_n(x) = (-1)^n e^{x^2} \frac{d^n}{d x^n} e^{-x^2}, \quad n = 0,1,2,\ldots,
    \end{align*}
we first compute 
\begin{align*}
H_n'(x)=2x(-1)^ne^{x^2}\frac{d^n}{d x^n} e^{-x^2}+(-1)^ne^{x^2}\frac{d^{n+1}}{d x^{n+1}} e^{-x^2}=2xH_n(x)-H_{n+1}(x),
\end{align*}
so
\begin{align*}
H_n''(x)=2H_n(x)+2xH_n'(x)-H_{n+1}'(x).
\end{align*}
Then, we can reduce the differential equation to a recurrence relation as
\begin{align*}
&H_n''(x) - 2 x H_n'(x) + 2 n H_n(x)=2H_n(x)+2xH_n'(x)-H_{n+1}'(x)- 2 x H_n'(x)+ 2 n H_n(x)\\&=
(2+2n)H_n(x)-H_{n+1}'(x)=2(n+1)H_n(x)-2xH_{n+1}(x)+H_{n+2}(x).
\end{align*}
Now, we use the generalized Leibnitz rule to compute
\begin{align*}
\frac{d^{n+2}}{d x^{n+2}} e^{-x^2}&=\frac{d^{n+1}}{d x^{n+1}} \left(\frac{d}{dx}e^{-x^2}\right)=\frac{d^{n+1}}{d x^{n+1}}(-2xe^{-x^2})\\&=
\sum_{k=0}^{n+1}\binom{n+1}{k}\frac{d^k}{dx^k}(-2x)\frac{d^{n+1-k}}{d x^{n+1-k}}e^{-x^2}=-2x\frac{d^{n+1}}{d x^{n+1}}e^{-x^2}-2(n+1)\frac{d^{n}}{d x^{n}}e^{-x^2}.
\end{align*}
From this we can use the definition of $H_n$ to find that 
\begin{align*}
&2(n+1)H_n(x)-2xH_{n+1}(x)+H_{n+2}(x)\\&=
2(n+1)(-1)^n e^{x^2} \frac{d^n}{d x^n} e^{-x^2}-2x(-1)^{n+1} e^{x^2} \frac{d^{n+1}}{d x^{n+1}} e^{-x^2}+(-1)^{n+2} e^{x^2} \frac{d^{n+2}}{d x^{n+2}} e^{-x^2}\\&=
(-1)^n e^{x^2}\left(2(n+1)\frac{d^n}{d x^n} e^{-x^2}+2x\frac{d^{n+1}}{d x^{n+1}} e^{-x^2}-2x\frac{d^{n+1}}{d x^{n+1}}e^{-x^2}-2(n+1)\frac{d^{n}}{d x^{n}}e^{-x^2}\right)\\&=
0.
\end{align*}
Thus, the Hermite polynomials satisfy the differential equation
\begin{align*}
      H_n''(x) - 2 x H_n'(x) + 2 n H_n(x) = 0.
    \end{align*}

\subsection{Part b}
Now, let $\psi_n(x) := (2^n n! \sqrt{\pi}) e^{-\frac{x^2}{2}} H_n(x)$. Then,
\begin{align*}
&\psi_n'(x)=(2^n n! \sqrt{\pi})(-x)e^{-\frac{x^2}{2}} H_n(x)+(2^n n! \sqrt{\pi}) e^{-\frac{x^2}{2}} H_n'(x)\\
&\psi_n''(x)=(2^n n! \sqrt{\pi})\left(-e^{-\frac{x^2}{2}} H_n(x)+x^2e^{-\frac{x^2}{2}} H_n(x)-xe^{-\frac{x^2}{2}} H_n'(x)-xe^{-\frac{x^2}{2}} H_n'(x)+e^{-\frac{x^2}{2}} H_n''(x)\right),
\end{align*}
so the differential equation
\begin{align*}
&\psi_n''(x) + (2n + 1 - x^2) \psi_n(x) =(2^n n! \sqrt{\pi})\big(-e^{-\frac{x^2}{2}} H_n(x)+x^2e^{-\frac{x^2}{2}} H_n(x)-xe^{-\frac{x^2}{2}} H_n'(x)\\&-xe^{-\frac{x^2}{2}} H_n'(x)+e^{-\frac{x^2}{2}} H_n''(x)+ (2n + 1 - x^2)e^{-\frac{x^2}{2}}H_n(x)\big)\\&=
\left(2^n n! \sqrt{\pi}e^{-\frac{x^2}{2}}\right)\left(2nH_n(x)-2xH_n'(x)+H_n''(x)\right)=0
\end{align*}
by the differential equation from part a.
\subsection{Part c}
From our definitions in part b, define
\[
\Psi_n(x)=\psi_n(cx)
\]
where $c$ is some undetermined constant. Then, 
\[
\Psi_n''(x)=c^2\psi_n''(cx),
\]
so our differential equation from part b gives that 
\[
-\frac{1}{c^2}\Psi_n''(x)-(2n+1-c^2x^2)\Psi(x)=0
\]
which we can rewrite as
\[
-\frac{1}{c^4}\Psi_n''(x)+x^2\Psi_n(x)=\frac{2n+1}{c^2}\Psi_n(x).
\]
Now, we set 
\[
\frac{1}{c^4}=\frac{\hbar^2}{2},
\]
so
\[
c=\left(\frac{2}{\hbar^2}\right)^{1/4}.
\]
Then, our equation becomes
\[
-\frac{\hbar^2}{2}\Psi_n''(x)+x^2\Psi_n(x)=\pm\sqrt{\frac{\hbar^2}{2}}(2n+1)\Psi_n(x)=\pm\sqrt{2}\hbar\left(n+\frac{1}{2}\right)\Psi_n(x).
\]
We need to choose the plus sign to ensure square integrability, so  the operator $\mathcal S_\hbar = - \frac{\hbar^2}{2} \frac{d^2}{d x^2} + x^2$ has $L^2$ eigenvalues $E=\sqrt{2}\hbar\left(n+\frac{1}{2}\right)$ where $n = 0,1,2,\ldots$.\\
Now, we attempt to verify (7.85) in the text by first computing $\phi(E)$ as defined on page 303. Here, we have that $V(x)=x^2$ so solving $x^2-E=0$ gives that $x_-=-\sqrt{E}$ and $x_+=\sqrt{E}$, and
\begin{align*}
\phi(E)=\sqrt{2}\int_{-\sqrt{E}}^{\sqrt{E}}\sqrt{E-x^2}dx=\sqrt{2}\frac{\pi E}{2}=\frac{\pi}{\sqrt{2}}E.
\end{align*}
Thus, (7.85) gives that 
\[
\frac{\pi}{\sqrt{2}}E=\pi\hbar\left(n+\frac{1}{2}\right),
\]
meaning that 
\[
E=\sqrt{2}\hbar\left(n+\frac{1}{2}\right)
\]
which matches what we already derived.

\section{Problem 3}
Consider the Airy equation
\[
y''(x)-xy(x)=0
\]
as $x\to\infty$. We perform the substitution $x=\lambda^\alpha z$ where $\lambda\to\infty,\alpha>0$ by letting $Y(z)=y(\lambda^\alpha z)$ so that $Y''(z)=\lambda^{2\alpha}y''(\lambda z)$ and
\[
y''(x)-xy(x)=\lambda^{-2\alpha}Y''(z)-\lambda^\alpha zY(z)=0
\]
which we rewrite as
\[
Y''(z)-\lambda^{3\alpha}zY(z)=0.
\]
To match the form we want for WKB, we need that $3\alpha=2$, so $\alpha=2/3$. Then, we can apply WKB to the new equation
\[
Y''(z)-f(z)Y(z)=0
\]
where 
\[
f(z)=\lambda^2\sum_{n=0}^\infty f_n(z)\lambda^{-n}
\]
with $f_0(z)=-z$ and $f_j(z)=0$ for $j\geq1$. Then, we look for solutions of the form $Y(z)=e^{\phi(z)}$ by letting $u(z)=\phi'(z)$ and solving the resulting Riccati equation
\[
u'(z)+u(z)^2+f(z)=0.
\]
Then, following page 9 of part 5 of the lecture notes, we look for 
\[
u(z)\sim\lambda\sum_{n=0}^\infty u_n(z)\lambda^{-n}.
\]
First, we need $u_0$ to solve
\[
0=u_0^2(z)+f_0(z)=u_0^2(z)-z.
\]
We can consider $z>0$ since we consider $x=\lambda^{2/3} z\to\infty$ and $\lambda\to\infty$, so we simply find that
\[
u_0(z)=\pm z^{1/2}.
\]
We then use our hierarchy of equations to compute
\[
u_1(z)=\frac{-1}{2u_0(z)}(u_0'(z)+f_1(z))=\frac{-1}{\pm2z^{1/2}}\frac{\pm1}{2z^{1/2}}=-\frac{1}{4z}
\]
and 
\[
u_2(z)=\frac{-1}{2u_0(z)}(u_1'(z)+u_1(z)^2+f_2(z))=\frac{-1}{\pm2z^{1/2}}\left(\frac{1}{4z^2}+\frac{1}{16z^2}\right)=\mp\frac{5}{32z^{5/2}}, 
\]
so
\[
u(z)\sim \pm z^{1/2}\lambda-\frac{1}{4z}\mp\frac{5}{32z^{5/2}\lambda}+\OO(\lambda^{-2})
\]From this, we can compute
\begin{align*}
\phi(z)&=\lambda\sum_{n=0}^\infty \int_{z_0}^zu_n(s)ds\lambda^{-n}=\pm\lambda\int_{z_0}^zs^{1/2}ds-\int_{z_0}^z\frac{ds}{4s}\mp\lambda^{-1}\int_{z_0}^z\frac{5}{32s^{5/2}}ds+\OO(\lambda^{-2})\\&=
\pm\lambda\left(\frac{2}{3}z^{3/2}-\frac{2}{3}z_0^{3/2}\right)-\left(\frac{1}{4}\log{z}-\frac{1}{4}\log{z_0}\right)\mp\lambda^{-1}\left(-\frac{5}{48z^{3/2}}+\frac{5}{48z_0^{3/2}}\right)+\OO(\lambda^{-2})\\&=
\pm\lambda\frac{2}{3}z^{3/2}-\frac{1}{4}\log{z}\pm\lambda^{-1}\frac{5}{48z^{3/2}}+C_1\lambda+C_0+C_{-1}\lambda^{-1}+\OO(\lambda^{-2})
\end{align*}
where $C_1,C_0,C_{-1}$ are constants depending on our choice of $z_0$. Thus, we have that 
\begin{align*}
Y(z)=\exp\left(\pm\frac{2}{3}(\lambda^{2/3} z)^{3/2}-\frac{1}{4}\log{z}\pm\frac{5}{48(\lambda^{2/3} z)^{3/2}}+C_1\lambda+C_0+C_{-1}\lambda^{-1}+\OO(\lambda^{-2})\right).
\end{align*}
Undoing our substitution, we find
\begin{align*}
y(x)&=\exp\left(\pm\frac{2}{3}x^{3/2}-\frac{1}{4}\log{\lambda^{-2/3}x}\pm\frac{5}{48x^{3/2}}+C_1\lambda+C_0+C_{-1}\lambda^{-1}+\OO(x^{-3})\right)\\&=
Cx^{-1/4}\exp\left(\pm\frac{2}{3}x^{3/2}\right)\exp\left(\pm\frac{5}{48x^{3/2}}+\OO(x^{-3})\right)
\end{align*}
where we have grouped $C_0$ and all terms depending on $\lambda$ into a single constant $C$ also depending on $\lambda$. Taylor expanding,
\begin{align*}
y(x)=Cx^{-1/4}\exp\left(\pm\frac{2}{3}x^{3/2}\right)\sum_{k=0}^\infty\frac{\left(\pm\frac{5}{48x^{3/2}}+\OO(x^{-3})\right)^k}{k!}.
\end{align*}
Taking the 0th and 1st terms of this power series,
\[
y(x)\sim Cx^{-1/4}\exp\left(\pm\frac{2}{3}x^{3/2}\right)\left(1\pm \frac{5}{48x^{3/2}}+\OO(x^{-3})\right).
\]
To match this with the expansion on DLMF, we need to take the minus sign and let $C=\frac{1}{2\sqrt{\pi}}$ (which amounts to choosing $z_0$ such that this holds). Then,
\[
y(x)\sim \frac{1}{2\sqrt{\pi}x^{1/4}}e^{-\frac{2}{3}x^{3/2}}\left(1-\ \frac{5}{48x^{3/2}}+\OO(x^{-3})\right).
\]
%Maybe say more explicitly how to match DLMF
To verify that this is the same as DLMF, we compute
\[
u_1=\frac{15}{216}u_0=\frac{5}{72}
\]
so
\begin{align*}
\text{Ai}(x)\sim\frac{e^{-\frac{2}{3}x^{3/2}}}{2\sqrt{\pi}x^{1/4}}\sum_{k=0}^\infty(-1)^k\frac{u_k}{\left(\frac{2}{3}x^{3/2}\right)^k}=x^{-1/4}\exp\left(\pm\frac{2}{3}x^{3/2}\right)\left(1\pm \frac{5}{48x^{3/2}}+\OO(x^{-3})\right)
\end{align*}
which is precisely what we obtained above.

\end{document}
