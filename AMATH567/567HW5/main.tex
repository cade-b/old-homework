\documentclass{article}
\usepackage[utf8]{inputenc}
\usepackage{listings}
\usepackage{multimedia} % to embed movies in the PDF file
\usepackage{graphicx}
\usepackage{comment}
\usepackage[english]{babel}
\usepackage{amsmath}
\usepackage{amsfonts}
\usepackage{subfigure}
\usepackage{wrapfig}
\usepackage{multirow}
\usepackage{tikz}
\usepackage{verbatim}

\newtheorem{theorem}{Theorem}[section]
\newtheorem{lemma}[theorem]{Lemma}
\newtheorem{corollary}[theorem]{Corollary}
%\newtheorem{algorithm}[theorem]{Algorithm}
\newtheorem{remark}[theorem]{Remark}
\newenvironment{proof}{\noindent {\bf Proof:} }{\hfill $\Box$ \\[2ex] }
\newenvironment{keywords}{\begin{quote} {\bf Key words} }
                         {\end{quote} }
\newenvironment{AMS}{\begin{quote} {\bf AMS subject classifications} }
                         {\end{quote} }


\newcommand{\eref}[1]{\mbox{\rm(\ref{#1})}}
\newcommand{\tref}[1]{\mbox{\rm\ref{#1}}}
\newcommand{\set}[2]{\left\{ #1 \; : \; #2 \right\} }
\newcommand{\deq}{\raisebox{0pt}[1ex][0pt]{$\stackrel{\scriptscriptstyle{\rm def}}{{}={}}$}}

\newcommand {\DS} {\displaystyle}

\newcommand{\real}{\mathbb{R}}
\newcommand{\compl}{\mathbb{C}}



\newcommand {\half} {\mbox{$\frac{1}{2}$}}
\newcommand{\force}{{\mathbf{f}}}
\newcommand{\strain}{{\boldsymbol{\varepsilon}}}
\newcommand{\stress}{{\boldsymbol{\sigma}}}
\renewcommand{\div}{{\boldsymbol{\nabla}}}

\newcommand {\cA} {{\cal A}}
\newcommand {\cB} {{\cal B}}
\newcommand {\cC} {{\cal C}}
\newcommand {\cD} {{\cal D}}
\newcommand {\cE} {{\cal E}}
\newcommand {\cL} {{\cal L}}
\newcommand {\cP} {{\cal P}}
\newcommand {\cQ} {{\cal Q}}
\newcommand {\cR} {{\cal R}}
\newcommand {\cV} {{\cal V}}
\newcommand {\cW} {{\cal W}}
\newcommand {\CH} {{\cal H}}
\newcommand {\CS} {{\cal S}}


\newcommand{\bzero}{\mathbf{0}}
\newcommand{\ba}{\mathbf{a}}
\newcommand{\bb}{\mathbf{b}}
\newcommand{\bc}{\mathbf{c}}
\newcommand{\bd}{\mathbf{d}}
\newcommand{\be}{\mathbf{e}}
\newcommand{\bff}{\mathbf{f}}
\newcommand{\bg}{\mathbf{g}}
\newcommand{\bh}{\mathbf{h}}
\newcommand{\bn}{\mathbf{n}}
\newcommand{\bp}{\mathbf{p}}
\newcommand{\bq}{\mathbf{q}}
\newcommand{\br}{\mathbf{r}}
\newcommand{\bs}{\mathbf{s}}
\newcommand{\bt}{\mathbf{t}}
\newcommand{\bu}{\mathbf{u}}
\newcommand{\bv}{\mathbf{v}}
\newcommand{\bw}{\mathbf{w}}
\newcommand{\bx}{\mathbf{x}}
\newcommand{\by}{\mathbf{y}}
\newcommand{\bz}{\mathbf{z}}
\newcommand{\bA}{\mathbf{A}}
\newcommand{\bB}{\mathbf{B}}
\newcommand{\bC}{\mathbf{C}}
\newcommand{\bD}{\mathbf{D}}
\newcommand{\bE}{\mathbf{E}}
\newcommand{\bF}{\mathbf{F}}
\newcommand{\bG}{\mathbf{G}}
\newcommand{\bH}{\mathbf{H}}
\newcommand{\bI}{\mathbf{I}}
\newcommand{\bJ}{\mathbf{J}}
\newcommand{\bK}{\mathbf{K}}
\newcommand{\bL}{\mathbf{L}}
\newcommand{\bM}{\mathbf{M}}
\newcommand{\bN}{\mathbf{N}}
\newcommand{\bO}{\mathbf{O}}
\newcommand{\bP}{\mathbf{P}}
\newcommand{\bQ}{\mathbf{Q}}
\newcommand{\bR}{\mathbf{R}}
\newcommand{\bS}{\mathbf{S}}
\newcommand{\bU}{\mathbf{U}}
\newcommand{\bV}{\mathbf{V}}
\newcommand{\bW}{\mathbf{W}}
\newcommand{\bX}{\mathbf{X}}
\newcommand{\bY}{\mathbf{Y}}
\newcommand{\bZ}{\mathbf{Z}}

\newcommand{\cO}{ {\cal O} }
\newcommand{\CT}{ {\cal T} }
\newcommand{\IL}{{\mathbb L}}
\newcommand{\sIL}{{{{\mathbb L}_s}}}
\newcommand{\bOmega}{{\boldsymbol{\Omega}}}
\newcommand{\bPsi}{{\boldsymbol{\Psi}}}

\newcommand{\bgamma}{{\boldsymbol{\gamma}}}
\newcommand{\bmu}{{\boldsymbol{\mu}}}
\newcommand{\blambda}{{\boldsymbol{\lambda}}}
\newcommand{\bLambda}{{\boldsymbol{\Lambda}}}
\newcommand{\bpi}{{\boldsymbol{\pi}}}
\newcommand{\bPi}{{\boldsymbol{\Pi}}}
\newcommand{\bphi}{{\boldsymbol{\phi}}}
\newcommand{\bPhi}{{\boldsymbol{\Phi}}}
\newcommand{\bpsi}{{\boldsymbol{\psi}}}
\newcommand{\btheta}{{\boldsymbol{\theta}}}
\newcommand{\bTheta}{{\boldsymbol{\Theta}}}
\newcommand{\bSigma}{{\boldsymbol{\Sigma}}}
\newcommand{\sump}{\sideset{}{^{'}}\sum} 
\DeclareMathOperator*{\Res}{Res}
\DeclareMathOperator{\OO}{O}
\DeclareMathOperator{\oo}{o}
\DeclareMathOperator{\erfc}{erfc}
\def\Xint#1{\mathchoice
   {\XXint\displaystyle\textstyle{#1}}%
   {\XXint\textstyle\scriptstyle{#1}}%
   {\XXint\scriptstyle\scriptscriptstyle{#1}}%
   {\XXint\scriptscriptstyle\scriptscriptstyle{#1}}%
   \!\int}
\def\XXint#1#2#3{{\setbox0=\hbox{$#1{#2#3}{\int}$}
     \vcenter{\hbox{$#2#3$}}\kern-.5\wd0}}
\def\ddashint{\Xint=}
\def\pvint{\Xint-}





\title{AMATH 567 Homework 5}
\author{Cade Ballew \#2120804}
\date{November 3, 2021}

\begin{document}

\maketitle

\section{Problem 1 (2.6.1)}
Let $C$ be the unit circle centered at the origin. 
\subsection{Part a}
Consider the function $f(z)=\sin z$ which is entire, so we can apply Cauchy's integral formula at 0 to it for any simple closed contour. Thus,
\[
\sin{0}=\frac{1}{2\pi i}\oint_C\frac{\sin{z}}{z-0}dz.
\]
This gives that 
\[
\oint_C\frac{\sin{z}}{z}dz=2\pi i\sin{0}=0.
\]
\subsection{Part b}
Consider the function $f(z)=\frac{1}{4}$ which is entire and apply theorem 2.6.2 in the text (the derivatives of Cauchy's theorem) at $1/2$ for $k=1$ to get
\[
f'(\frac{1}{2})=\frac{1!}{2\pi i}\oint_C\frac{f(z)}{(z-\frac{1}{2})^2}dz=\frac{1}{2\pi i}\oint_C\frac{1/4}{(z-\frac{1}{2})^2}dz=\frac{1}{2\pi i}\oint_C\frac{1}{(2z-1)^2}dz.
\]
However $f'(z)=0$ for any $z\in\mathbb{C}$ because $f$ is constant, so
\[
\oint_C\frac{1}{(2z-1)^2}dz=0.
\]
\subsection{Part c}
Now, consider the function $f(z)=\frac{1}{8}$ which is entire and apply theorem 2.6.2 in the text at $1/2$ for $k=2$ to get
\[
f''(\frac{1}{2})=\frac{2!}{2\pi i}\oint_C\frac{f(z)}{(z-\frac{1}{2})^3}dz=\frac{1}{\pi i}\oint_C\frac{1/8}{(z-\frac{1}{2})^3}dz=\frac{1}{\pi i}\oint_C\frac{1}{(2z-1)^3}dz.
\]
$f''(z)=0$ for any $z\in\mathbb{C}$ because $f$ is constant, so
\[
\oint_C\frac{1}{(2z-1)^3}dz=0.
\]
\subsection{Part d}
Consider $f(z)=e^z$ which is entire, so we apply Cauchy's formula at 0 to get
\[
f(0)=\frac{1!}{2\pi i}\oint_C\frac{f(z)}{z-0}dz=\frac{1}{2\pi i}\oint_C\frac{e^z}{z}dz.
\]
Thus,
\[
\oint_C\frac{e^z}{z}dz=2\pi if(0)=2\pi ie^0=2\pi i.
\]
\subsection{Part e}
Consider $f(z)=e^{z^2}$ which is once again entire, so we apply theorem 2.6.2 in the text for $k=1$ at 0 to get that
\[
f'(0)=\frac{1!}{2\pi i}\oint_C\frac{f(z)}{(z-0)^2}dz=\frac{1}{2\pi i}\oint_C\frac{e^{z^2}}{z^2}dz.
\]
Note that $f'(z)=2ze^{z^2}$, so 
\[
\oint_C\frac{e^{z^2}}{z^2}dz=2\pi i f'(0)= 0.
\]
Similarly, we apply the same theorem for $k=3$ to get that 
\[
f''(0)=\frac{2!}{2\pi i}\oint_C\frac{f(z)}{(z-0)^3}dz=\frac{1}{\pi i}\oint_C\frac{e^{z^2}}{z^3}dz.
\]
Note that $f''(z)=2e^{z^2}+4z^2e^{z^2}$, so 
\[
\oint_C\frac{e^{z^2}}{z^3}dz=\pi i f''(0)=2\pi i.
\]
Thus, we can conclude that 
\[
\oint_Ce^{z^2}\left(\frac{1}{z^2}-\frac{1}{z^3}\right)dz=\oint_C\frac{e^{z^2}}{z^2}dz-\oint_C\frac{e^{z^2}}{z^3}dz=0-2\pi i=-2\pi i.
\]

\section{Problem 2 (2.6.10)} 
Beginning with Cauchy's integral formula and letting the contour C be a circle of unit radius centered at the origin. Parameterize as $\zeta=e^{i\theta}$ which gives $d\zeta=ie^{i\theta}d\theta$. Then, 
\[
f(z)=\frac{1}{2\pi i}\oint_C\frac{f(\zeta)}{\zeta-z}d\zeta=\frac{1}{2\pi i}\int_0^{2\pi}\frac{f(e^{i\theta})}{e^{i\theta}-z}ie^{i\theta}d\theta=\frac{1}{2\pi }\int_0^{2\pi}\frac{f(e^{i\theta})e^{i\theta}}{e^{i\theta}-z}d\theta=\frac{1}{2\pi }\int_0^{2\pi}\frac{f(\zeta)\zeta}{\zeta-z}d\theta
\]
where $z$ lies inside the circle. Now, we can observe that 
\[
0=\frac{1}{2\pi i}\oint_C\frac{f(\zeta)}{\zeta-1/\overline{z}}d\zeta=\frac{1}{2\pi }\int_0^{2\pi}\frac{f(\zeta)\zeta}{\zeta-1/\overline{z}}d\theta.
\]
The first equality follows because the integrand has only one singularity at $\zeta=1/\overline{z}$ which is outside the unit circle that we are integrating over (it is equivalent to $\frac{1}{R}e^{i\theta}$ if $z=Re^{i\theta}$), so we can invoke Cauchy's theorem. The second follows because our steps in manipulating the integral to get that $\frac{1}{2\pi i}\oint_C\frac{f(\zeta)}{\zeta-z}d\zeta=\frac{1}{2\pi }\int_0^{2\pi}\frac{f(\zeta)\zeta}{\zeta-z}d\theta$ in the previous part did not depend on our choice of $z$. Now, note that $\zeta\overline{\zeta}=1$, so
\[
\frac{\zeta}{\zeta-1/\overline{z}}=\frac{\zeta\overline{\zeta}}{\zeta\overline{\zeta}-\overline{\zeta}/\overline{z}}=\frac{1}{1-\overline{\zeta}/\overline{z}}=\frac{\overline{z}}{\overline{z}-\overline{\zeta}}.
\]
Thus, 
\[
0=\frac{1}{2\pi}\int_0^{2\pi}\frac{f(\zeta)\overline{z}}{\overline{z}-\overline{\zeta}}.
\]
Now, if we subtract this from or add this to the first equation, we get that 
\[
f(z)=\frac{1}{2\pi}\int_0^{2\pi}f(\zeta)\left(\frac{\zeta}{\zeta-z}\mp\frac{\overline{z}}{\overline{z}-\overline{\zeta}}\right)d\theta=\frac{1}{2\pi}\int_0^{2\pi}f(\zeta)\left(\frac{\zeta}{\zeta-z}\pm\frac{\overline{z}}{\overline{\zeta}-\overline{z}}\right)d\theta.
\]
Using the plus sign,
\[
f(z)=\frac{1}{2\pi}\int_0^{2\pi}f(\zeta)\frac{\zeta(\overline{\zeta}-\overline{z})+\overline{z}(\zeta-z)}{(\zeta-z)(\overline{\zeta-z})}d\theta=\frac{1}{2\pi}\int_0^{2\pi}f(\zeta)\frac{\zeta\overline{\zeta}-z\overline{z}}{|\zeta-z|^2}d\theta=\frac{1}{2\pi}\int_0^{2\pi}f(\zeta)\frac{1-|z|^2}{|\zeta-z|^2}d\theta.
\]
\subsection{Part a}
Now, take $z=re^{i\phi}$ and define $u(r,\phi)=\Re f$. Then, 
\begin{align*}
u(r,\phi)&=\Re\left(\frac{1}{2\pi}\int_0^{2\pi}f(\zeta)\frac{1-|z|^2}{|\zeta-z|^2}d\theta\right)=\Re\left(\frac{1}{2\pi}\int_0^{2\pi}f(\zeta)\frac{1-r^2}{|\zeta|^2-2\Re(\overline{\zeta}z)+r^2}d\theta\right)\\&=
\Re\left(\frac{1}{2\pi}\int_0^{2\pi}f(e^{i\theta})\frac{1-r^2}{1-2\Re(re^{i(\phi-\theta)})+r^2}d\theta\right)\\&=
\Re\left(\frac{1}{2\pi}\int_0^{2\pi}(u(1,\theta)+iv(1,\theta))\underbrace{\frac{1-r^2}{1-2r\cos(\phi-\theta)+r^2}}_{\text{real}}d\theta\right)\\&=
\frac{1}{2\pi}\int_0^{2\pi}u(1,\theta)\frac{1-r^2}{1-2r\cos(\phi-\theta)+r^2}d\theta=\frac{1}{2\pi}\int_0^{2\pi}u(\theta)\frac{1-r^2}{1-2r\cos(\phi-\theta)+r^2}d\theta.
\end{align*}
\subsection{Part b}
Using the minus sign in the formula for $f(z)$, 
\[
f(z)=\frac{1}{2\pi}\int_0^{2\pi}f(\zeta)\frac{\zeta(\overline{\zeta}-\overline{z})-\overline{z}(\zeta-z)}{(\zeta-z)(\overline{\zeta-z})}d\theta=\frac{1}{2\pi}\int_0^{2\pi}f(\zeta)\frac{1-2\zeta\overline{z}+|z|^2}{|\zeta-z|^2}d\theta.
\]
Again taking $z=re^{i\phi}$,
\[
f(z)=\frac{1}{2\pi}\int_0^{2\pi}f(\zeta)\frac{1-2e^{i\theta}re^{-i\phi}+r^2}{1-2r\cos(\phi-\theta)+r^2}d\theta=\frac{1}{2\pi}\int_0^{2\pi}f(\zeta)\frac{1+r^2-2re^{i(\theta-\phi)}}{1-2r\cos(\phi-\theta)+r^2}d\theta.
\]
Taking the imaginary part of this, 
\begin{align*}
v(r,\phi)&=\Im\left(\frac{1}{2\pi}\int_0^{2\pi}f(\zeta)\frac{1+r^2-2re^{i(\theta-\phi)}}{1-2r\cos(\phi-\theta)+r^2}d\theta\right)\\&=
\Im\left(\frac{1}{2\pi}\int_0^{2\pi}(u(1,\theta)+iv(1,\theta))\left(\frac{1+r^2-2r\cos(\theta-\phi)}{1-2r\cos(\phi-\theta)+r^2}-i\frac{2r\sin(\theta-\phi)}{1-2r\cos(\phi-\theta)+r^2}\right)d\theta\right)\\&=
-\frac{1}{2\pi}\int_0^{2\pi}u(\theta)\frac{2r\sin(\theta-\phi)}{1-2r\cos(\phi-\theta)+r^2}d\theta+\frac{1}{2\pi}\int_0^{2\pi}v(\theta)\frac{1+r^2-2r\cos(\theta-\phi)}{1-2r\cos(\phi-\theta)+r^2}d\theta\\&=
\frac{1}{2\pi}\int_0^{2\pi}v(\theta)d\theta+\frac{1}{\pi}\int_0^{2\pi}u(\theta)\frac{r\sin(\phi-\theta)}{1-2r\cos(\phi-\theta)+r^2}d\theta\\&=
v(r=0)+\frac{1}{\pi}\int_0^{2\pi}u(\theta)\frac{r\sin(\phi-\theta)}{1-2r\cos(\phi-\theta)+r^2}d\theta
\end{align*}
where the penultimate step follows because sine is an odd function, and the last step from the first equation we derived in this problem taken at $z=0$. Note that the first term is constant with respect to $r$ and $\phi$.
\subsection{Part c}
Consider
\begin{align*}
\Im\left(\frac{\zeta+z}{\zeta-z}\right)&=\Im\left(\frac{(\zeta+z)(\overline{\zeta-z)}}{(\zeta-z)(\overline{\zeta-z})}\right)=\Im\left(\frac{1-r^2+2i\Im(z\overline{\zeta})}{|\zeta-z|^2}\right)\\&=
\Im\left(\frac{1-r^2+2i\Im(re^{i(\phi-\theta)})}{1-2r\cos(\phi-\theta)+r^2}\right)=\Im\left(\frac{1-r^2+2ir\sin(\phi-\theta)}{1-2r\cos(\phi-\theta)+r^2}\right)\\&=\Im\left(\frac{1-r^2}{1-2r\cos(\phi-\theta)+r^2}+i\frac{2r\sin(\phi-\theta)}{1-2r\cos(\phi-\theta)+r^2}\right)\\&=
\frac{2r\sin(\phi-\theta)}{1-2r\cos(\phi-\theta)+r^2}.
\end{align*}
Note that we need not worry about dividing by 0, because we cannot have that $\overline{z}=\overline{\zeta}$, since $z$ inside the circle, but $\zeta$ is on it, so $\phi\neq\theta$. Thus, the result from part b can be expressed as 
\[
v(r,\phi)=v(0)+\frac{1}{2\pi}\Im\left(\int_0^{2\pi}u(\theta)\frac{\zeta+z}{\zeta-z}d\theta\right).
\]

\section{Problem 3 (2.6.2 in lecture notes)}
Let $f(z)=R(x,y)\exp(i\theta(x,y))$ where $R$ and $\theta$ are real-valued functions of $x$ and $y$. If $f(z)$ is analytic, then it satisfies the Cauchy-Riemann equations, so using the fact that $f(x,y)=\underbrace{R(x,y)\cos(\theta(x,y))}_u+i\underbrace{R(x,y)\sin(\theta(x,y))}_v$, we can write
\[
u_x=R_x\cos(\theta(x,y))-\theta_xR(x,y)\sin(\theta(x,y))=R_y\sin(\theta(x,y))+\theta_yR(x,y)\cos(\theta(x,y))=v_y
\]
and 
\[
v_x=R_x\sin(\theta(x,y))+\theta_xR(x,y)\cos(\theta(x,y))=-R_y\cos(\theta(x,y))+\theta_yR(x,y)\sin(\theta(x,y))=-u_y.
\]
To simplify these, multiply the first equation by $\cos(\theta(x,y))$ and the second by $\sin(\theta(x,y))$ and add which gives that 
\[
R_x=R_x(\cos^2(\theta(x,y))+\sin^2(\theta(x,y)))=\theta_y(\cos^2(R(x,y))(\theta(x,y))+\sin^2(\theta(x,y)))=\theta_yR(x,y).
\]
Multiplying the first equation by $-\sin(\theta(x,y))$ and the second by $\cos(\theta(x,y))$ and adding gives that 
\[
\theta_xR(x,y)=\theta_xR(x,y)(\cos^2(\theta(x,y))+\sin^2(\theta(x,y)))=-R_y(\cos^2(\theta(x,y))+\sin^2(\theta(x,y)))=-R_y.
\]
Now that the $x$ and $y$ derivatives of $R$ and $\theta$ are connected, we consider the case where $R(x,y)=R$ is a constant. Then, $R_x,R_y=0$ so our equations become
\begin{align*}
\theta_yR&=0\\
\theta_xR&=0.
\end{align*}
If $R=0$, then these equations hold for all $\theta(x,y)$, but in that case, $f(z)=0$ is constant. If we assume $R\neq0$, then we have $\theta_x,\theta_y=0$, so $\theta(x,y)=\theta$ is constant with respect to $x$ and $y$. Thus, $f(z)=R\exp{i\theta}$ is constant.\\
If $\theta(x,y)=\theta$ is constant, $\theta_x,\theta_y=0$, so the equations become
\begin{align*}
R_x&=0\\
-R_y&=0.
\end{align*}
Thus, $R(x,y)=R$ is constant with respect to $x$ and $y$, meaning that $f(z)=R\exp{i\theta}$ is constant.

\section{Problem 4 (2.6.11 in lecture notes)}
Consider the Legendre polynomials defined by 
\[
P_n(z)=\frac{1}{2^n n!}\frac{d^n}{dz^n}(z^2-1)^n
\]
for $n\in\mathbb{N}$. Because $(z^2-1)^n$ is entire for any $n\in\mathbb{N}$, we can apply theorem 2.6.2 to compute the derivative above by taking $f(z)=(z^2-1)^n$ and $C$ to be any simple closed contour encircling $z$. Namely,
\[
\frac{d^n}{dz^n}(z^2-1)^n=\frac{n!}{2\pi i}\oint_C\frac{(t^2-1)^n}{(t-z)^{n+1}}dt.
\]
Thus,
\[
P_n(z)=\frac{1}{2^n n!}\frac{n!}{2\pi i}\oint_C\frac{(t^2-1)^n}{(t-z)^{n+1}}dt=\frac{1}{2\pi i}\oint_C\frac{(t^2-1)^n}{2^n(t-z)^{n+1}}dt
\]
for any $n\in\mathbb{N}$ and any simple closed contour $C$ encircling $z$.

\section{Problem 5 (2.6.12 in lecture notes)}
\subsection{Part a}
Parameterizing the unit circle $C(0,1)$ as $z=e^{it}$, $dz=ie^{it}dt$, we get that 
\begin{align*}
\frac{1}{2\pi i}\oint_{C(0,1)}\left(z+\frac{1}{z}\right)^n\frac{dz}{z}&=\frac{1}{2\pi i}\int_0^{2\pi}(e^{it}+e^{-it})^n\frac{ie^{it}dt}{e^{it}}\\&=
\frac{1}{2\pi}\int_0^{2\pi}\left(2\cos t\right)^ndt=\frac{2^n}{2\pi }\int_0^{2\pi}\cos^nt dt
\end{align*}
by the definition of the complex cosine.
\subsection{Part b}
Using the binomial formula on the LHS of part a, 
\begin{align*}
\frac{1}{2\pi i}\oint_{C(0,1)}\left(z+\frac{1}{z}\right)^n\frac{dz}{z}&=\frac{1}{2\pi i}\oint_{C(0,1)}\sum_{\ell=0}^n\binom{n}{\ell}z^{n-\ell}\frac{1}{z^\ell}\frac{dz}{z}\\&=
\frac{1}{2\pi i}\oint_{C(0,1)}\sum_{\ell=0}^n\binom{n}{\ell}z^{n-2\ell-1}dz\\&=
\frac{1}{2\pi i}\sum_{\ell=0}^n\binom{n}{\ell}\oint_{C(0,1)}z^{n-2\ell-1}dz.
\end{align*}
From this and part a, we find that for $k\in\mathbb{N}$, 
\begin{align*}
\frac{1}{2\pi }\int_0^{2\pi}\cos^{2k}t dt&=\frac{1}{2^{2k}}\frac{1}{2\pi i}\oint_{C(0,1)}\left(z+\frac{1}{z}\right)^{2k}\frac{dz}{z}\\&=
\frac{1}{2^{2k}}\frac{1}{2\pi i}\sum_{\ell=0}^{2k}\binom{2k}{\ell}\oint_{C(0,1)}z^{2k-2\ell-1}dz
\end{align*}
From pages 44 and 45 of the lecture notes, because our contour encircles the origin and $2k-2\ell-1$ is an integer for $\ell\in\{1,\ldots, 2k+1\}$, we have that
\[
\oint_{C(0,1)}z^{2k-2\ell-1}dz=
\begin{cases}
0, &2k-2\ell-1\neq-1\\
2\pi i, &2k-2\ell-1=-1
\end{cases}.
\]
Thus, $\oint_{C(0,1)}z^{2k-2\ell-1}dz$ will only be nonzero for $\ell=k$, meaning that we can write 
\begin{align*}
\frac{1}{2\pi }\int_0^{2\pi}\cos^{2k}t dt&=\frac{1}{2^{2k}}\frac{1}{2\pi i}\binom{2k}{k}\oint_{C(0,1)}z^{-1}dz=\frac{1}{2^{2k}}\frac{1}{2\pi i}\binom{2k}{k}2\pi i\\&=
\frac{1}{2^{2k}}\frac{(2k)!}{k! k!}=\frac{(2k)!}{2^{2k}(k!)^2}.
\end{align*}
Similarly, for $k\in\mathbb{N}$, 
\begin{align*}
\frac{1}{2\pi }\int_0^{2\pi}\cos^{2k+1}t dt&=\frac{1}{2^{2k+1}}\frac{1}{2\pi i}\oint_{C(0,1)}\left(z+\frac{1}{z}\right)^{2k+1}\frac{dz}{z}\\&=
\frac{1}{2^{2k+1}}\frac{1}{2\pi i}\sum_{\ell=0}^{2k+1}\binom{2k+1}{\ell}\oint_{C(0,1)}z^{2k+1-2\ell-1}dz
\end{align*}
Once again, we have that 
\[
\oint_{C(0,1)}z^{2k-2\ell-1}dz=
\begin{cases}
0, &2k+1-2\ell-1\neq-1\\
2\pi i, &2k+1-2\ell-1=-1
\end{cases}.
\]
by the same reasoning as before. However, $2k+1-2\ell-1=-1$ cannot hold for integers $k$ and $\ell$, because solving this yields $\ell=k+\frac{1}{2}$. Thus, $\oint_{C(0,1)}z^{2k-2\ell-1}dz=0$ for all $\ell\in\{1,\ldots, 2k+1\}$, meaning that 
\[
\frac{1}{2\pi }\int_0^{2\pi}\cos^{2k+1}t dt=0.
\]


\end{document}
