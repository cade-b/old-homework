\documentclass{article}
\usepackage[utf8]{inputenc}
\usepackage{listings}
\usepackage{multimedia} % to embed movies in the PDF file
\usepackage{graphicx}
\usepackage{comment}
\usepackage[english]{babel}
\usepackage{amsmath}
\usepackage{amsfonts}
\usepackage{subfigure}
\usepackage{wrapfig}
\usepackage{multirow}
\usepackage{tikz}
\usepackage{verbatim}

\newtheorem{theorem}{Theorem}[section]
\newtheorem{lemma}[theorem]{Lemma}
\newtheorem{corollary}[theorem]{Corollary}
%\newtheorem{algorithm}[theorem]{Algorithm}
\newtheorem{remark}[theorem]{Remark}
\newenvironment{proof}{\noindent {\bf Proof:} }{\hfill $\Box$ \\[2ex] }
\newenvironment{keywords}{\begin{quote} {\bf Key words} }
                         {\end{quote} }
\newenvironment{AMS}{\begin{quote} {\bf AMS subject classifications} }
                         {\end{quote} }


\newcommand{\eref}[1]{\mbox{\rm(\ref{#1})}}
\newcommand{\tref}[1]{\mbox{\rm\ref{#1}}}
\newcommand{\set}[2]{\left\{ #1 \; : \; #2 \right\} }
\newcommand{\deq}{\raisebox{0pt}[1ex][0pt]{$\stackrel{\scriptscriptstyle{\rm def}}{{}={}}$}}

\newcommand {\DS} {\displaystyle}

\newcommand{\real}{\mathbb{R}}
\newcommand{\compl}{\mathbb{C}}



\newcommand {\half} {\mbox{$\frac{1}{2}$}}
\newcommand{\force}{{\mathbf{f}}}
\newcommand{\strain}{{\boldsymbol{\varepsilon}}}
\newcommand{\stress}{{\boldsymbol{\sigma}}}
\renewcommand{\div}{{\boldsymbol{\nabla}}}

\newcommand {\cA} {{\cal A}}
\newcommand {\cB} {{\cal B}}
\newcommand {\cC} {{\cal C}}
\newcommand {\cD} {{\cal D}}
\newcommand {\cE} {{\cal E}}
\newcommand {\cL} {{\cal L}}
\newcommand {\cP} {{\cal P}}
\newcommand {\cQ} {{\cal Q}}
\newcommand {\cR} {{\cal R}}
\newcommand {\cV} {{\cal V}}
\newcommand {\cW} {{\cal W}}
\newcommand {\CH} {{\cal H}}
\newcommand {\CS} {{\cal S}}


\newcommand{\bzero}{\mathbf{0}}
\newcommand{\ba}{\mathbf{a}}
\newcommand{\bb}{\mathbf{b}}
\newcommand{\bc}{\mathbf{c}}
\newcommand{\bd}{\mathbf{d}}
\newcommand{\be}{\mathbf{e}}
\newcommand{\bff}{\mathbf{f}}
\newcommand{\bg}{\mathbf{g}}
\newcommand{\bh}{\mathbf{h}}
\newcommand{\bn}{\mathbf{n}}
\newcommand{\bp}{\mathbf{p}}
\newcommand{\bq}{\mathbf{q}}
\newcommand{\br}{\mathbf{r}}
\newcommand{\bs}{\mathbf{s}}
\newcommand{\bt}{\mathbf{t}}
\newcommand{\bu}{\mathbf{u}}
\newcommand{\bv}{\mathbf{v}}
\newcommand{\bw}{\mathbf{w}}
\newcommand{\bx}{\mathbf{x}}
\newcommand{\by}{\mathbf{y}}
\newcommand{\bz}{\mathbf{z}}
\newcommand{\bA}{\mathbf{A}}
\newcommand{\bB}{\mathbf{B}}
\newcommand{\bC}{\mathbf{C}}
\newcommand{\bD}{\mathbf{D}}
\newcommand{\bE}{\mathbf{E}}
\newcommand{\bF}{\mathbf{F}}
\newcommand{\bG}{\mathbf{G}}
\newcommand{\bH}{\mathbf{H}}
\newcommand{\bI}{\mathbf{I}}
\newcommand{\bJ}{\mathbf{J}}
\newcommand{\bK}{\mathbf{K}}
\newcommand{\bL}{\mathbf{L}}
\newcommand{\bM}{\mathbf{M}}
\newcommand{\bN}{\mathbf{N}}
\newcommand{\bO}{\mathbf{O}}
\newcommand{\bP}{\mathbf{P}}
\newcommand{\bQ}{\mathbf{Q}}
\newcommand{\bR}{\mathbf{R}}
\newcommand{\bS}{\mathbf{S}}
\newcommand{\bU}{\mathbf{U}}
\newcommand{\bV}{\mathbf{V}}
\newcommand{\bW}{\mathbf{W}}
\newcommand{\bX}{\mathbf{X}}
\newcommand{\bY}{\mathbf{Y}}
\newcommand{\bZ}{\mathbf{Z}}

\newcommand{\cO}{ {\cal O} }
\newcommand{\CT}{ {\cal T} }
\newcommand{\IL}{{\mathbb L}}
\newcommand{\sIL}{{{{\mathbb L}_s}}}
\newcommand{\bOmega}{{\boldsymbol{\Omega}}}
\newcommand{\bPsi}{{\boldsymbol{\Psi}}}

\newcommand{\bgamma}{{\boldsymbol{\gamma}}}
\newcommand{\bmu}{{\boldsymbol{\mu}}}
\newcommand{\blambda}{{\boldsymbol{\lambda}}}
\newcommand{\bLambda}{{\boldsymbol{\Lambda}}}
\newcommand{\bpi}{{\boldsymbol{\pi}}}
\newcommand{\bPi}{{\boldsymbol{\Pi}}}
\newcommand{\bphi}{{\boldsymbol{\phi}}}
\newcommand{\bPhi}{{\boldsymbol{\Phi}}}
\newcommand{\bpsi}{{\boldsymbol{\psi}}}
\newcommand{\btheta}{{\boldsymbol{\theta}}}
\newcommand{\bTheta}{{\boldsymbol{\Theta}}}
\newcommand{\bSigma}{{\boldsymbol{\Sigma}}}
\newcommand{\sump}{\sideset{}{^{'}}\sum} 
\DeclareMathOperator*{\Res}{Res}
\DeclareMathOperator{\OO}{O}
\DeclareMathOperator{\oo}{o}
\DeclareMathOperator{\erfc}{erfc}
\def\Xint#1{\mathchoice
   {\XXint\displaystyle\textstyle{#1}}%
   {\XXint\textstyle\scriptstyle{#1}}%
   {\XXint\scriptstyle\scriptscriptstyle{#1}}%
   {\XXint\scriptscriptstyle\scriptscriptstyle{#1}}%
   \!\int}
\def\XXint#1#2#3{{\setbox0=\hbox{$#1{#2#3}{\int}$}
     \vcenter{\hbox{$#2#3$}}\kern-.5\wd0}}
\def\ddashint{\Xint=}
\def\pvint{\Xint-}





\title{AMATH 567 Homework 9}
\author{Cade Ballew \#2120804}
\date{December 1, 2021}

\begin{document}

\maketitle

\section{Problem 1 (4.2.1)}
\subsection{Part c}
Consider the contour $C=C_1+C_R$ where $C_1$ is the line on the real axis from $-R$ to $R$ and $C_R$ is the upper semicircle of radius $R$ from $R$ to $-R$ as defined in the text. Then, for sufficiently large $R$ and taking $a\neq b$ with $a,b\in\real$, the residue theorem gives that 
\begin{align*}
\oint_C\frac{dz}{(z^2 + a^2)(z^2 + b^2)}&=2\pi i\Res_{z=|a|i}\frac{1}{(z^2 + a^2)(z^2 + b^2)}+2\pi i\Res_{z=|b|i}\frac{1}{(z^2 + a^2)(z^2 + b^2)}\\&=
2\pi i\lim_{z\to|a|i}\frac{1}{(z + |a|i)(z^2 + b^2)}+2\pi i\lim_{z\to|b|i}\frac{1}{(z + |b|i)(z^2 + a^2)}\\&=
\frac{2\pi i}{2|a|i(-a^2 + b^2)}+\frac{2\pi i}{2|b|i(-b^2 + a^2)}=\frac{\pi(-|b|+|a|)}{|ab|(a^2-b^2)}\\&=
\frac{\pi}{|ab|(|a|+|b|)}.
\end{align*}
Now, letting $f(z)=\frac{1}{(z^2 + a^2)(z^2 + b^2)}$, theorem 4.2.1 in the text gives that 
\[
\lim_{R\to\infty}\int_{C_R}f(z)dz=0
\]
because the denominator is degree 4 and the numerator is degree 0. Also,
\[
\lim_{R\to\infty}\int_{C_1}f(z)dz=\lim_{R\to\infty}\int_{-R}^Rf(z)dz=\int_{-\infty}^\infty f(x)dx
\]
by definition. Thus, we can take the limit as $R\to\infty$ of both sides of the equation $\oint_{C}f(z)dz=\int_{C_1}f(z)dz+\int_{C_R}f(z)dz$ to get that 
\[
\int_{-\infty}^\infty f(x)dx = \frac{\pi}{|ab|(|a|+|b|)}.
\]
Now, we simply use the fact that $f$ is even as a function of $x$ to conclude that 
\[
\int_{0}^\infty \frac{1}{(x^2 + a^2)(x^2 + b^2)}dx = \frac{\pi}{2|ab|(|a|+|b|)}.
\]
Now, in search of extra credit, consider the case where $a=b$. Namely, $f(z)=\frac{1}{(z^2+a^2)^2}$. Then, for the same contour $C$ sufficiently large, the residue theorem and theorem 4.2 in the notes give that 
\begin{align*}
\oint_Cf(z)dz&=2\pi i\Res_{z=|a|i}\frac{1}{(z+|a|i)^2(z-|a|i)^2}=2\pi i\lim_{z\to|a|i}\frac{d}{dz}\frac{1}{(z+|a|i)^2}\\&=2\pi i\lim_{z\to|a|i}\frac{-2}{(z+|a|i)^3}=2\pi i\frac{-2}{8|a|^3(-i)}=\frac{\pi}{2|a|^3}.
\end{align*}
The remaining manipulations we did are not affected by the fact that $a=b$ now, because the degree of the polynomial in the denominator of $f$ is unchanged. Thus, 
\[
\int_{0}^\infty \frac{1}{(x^2 + a^2)(x^2 + b^2)}dx =\frac{1}{2}\oint_Cf(z)dz=\frac{\pi}{4|a|^3}.
\]

\section{Problem 2 (4.2.2)} 
\subsection{Part b}
Consider the same contour $C=C_1+C_R$ as in problem 1 and let
\[
f(z)=\frac{e^{ikz}}{(z^2 + a^2)(z^2 + b^2)}
\]
where $a^2, b^2, k > 0$. Then, if we let $a\neq b$, the residue theorem gives that 
\begin{align*}
\oint_Cf(z)&=2\pi i\Res_{z=|a|i}\frac{e^{ikz}}{(z^2 + a^2)(z^2 + b^2)}+2\pi i\Res_{z=|b|i}\frac{e^{ikz}}{(z^2 + a^2)(z^2 + b^2)}\\&=
2\pi i\lim_{z\to|a|i}\frac{e^{ikz}}{(z + |a|i)(z^2 + b^2)}+2\pi i\lim_{z\to|b|i}\frac{e^{ikz}}{(z + |b|i)(z^2 + a^2)}\\&=
\frac{2\pi ie^{-k|a|}}{2|a|i(-a^2 + b^2)}+\frac{2\pi ie^{-k|b|}}{2|b|i(-b^2 + a^2)}=\frac{\pi(-e^{-k|a|}|b|+e^{-k|b|}|a|)}{|ab|(a^2-b^2)}.
\end{align*}
If we consider the case where $a=b$, then $f(z)=\frac{e^{ikz}}{(z^2 + a^2)^2}$ and the residue theorem and theorem 4.2 in the lecture notes give that 
\begin{align*}
\oint_Cf(z)dz&=2\pi i\Res_{z=|a|i}\frac{e^{ikz}}{(z+|a|i)^2(z-|a|i)^2}=2\pi i\lim_{z\to|a|i}\frac{d}{dz}\frac{e^{ikz}}{(z+|a|i)^2}\\&=2\pi i\lim_{z\to|a|i}\frac{ike^{ikz}(z+a|i|)^2-2e^{ikz}(z+|a|i)}{(z+|a|i)^4}=2\pi i\frac{ike^{-k|a|}2|a|i-2e^{-k|a|}}{8|a|^3(-i)}\\&=
\frac{-4\pi k|a|e^{-k|a|}-4\pi e^{-k|a|}}{-8|a|^3}=\frac{\pi e^{-k|a|}(k|a|+1)}{2|a|^3}.
\end{align*}
Note that in both cases, we have obtained a real number. Now, Jordan's lemma gives that 
\[
\lim_{R\to\infty}\int_{C_R}f(z)dz=0
\]
because the polynomial in the denominator is degree 4 and the polynomial in the numerator is degree 0. Also,
\[
\lim_{R\to\infty}\int_{C_1}f(z)dz=\lim_{R\to\infty}\int_{-R}^Rf(z)dz=\int_{-\infty}^\infty f(x)dx
\]
by definition. Thus, we can take the limit as $R\to\infty$ of both sides of the equation $\oint_{C}f(z)dz=\int_{C_1}f(z)dz+\int_{C_R}f(z)dz$ to get that 
\[
\int_{-\infty}^\infty f(x)dx = \frac{\pi(-e^{-k|a|}|b|+e^{-k|b|}|a|)}{|ab|(a^2-b^2)}
\]
in the case where $a\neq b$ and 
\[
\int_{-\infty}^\infty f(x)dx = \frac{\pi e^{-k|a|}(k|a|-1)}{2|a|^3}
\]
in the case where $a=b$. Now, Euler's formula gives that 
\[
\int_{-\infty}^\infty\frac{e^{ikx}}{(x^2 + a^2)(x^2 + b^2)}dz=\int_{-\infty}^\infty \frac{\cos(kx)}{(x^2 + a^2)(x^2 + b^2)}dx+i\int_{-\infty}^\infty \frac{\sin(kx)}{(x^2 + a^2)(x^2 + b^2)}.
\]
Thus, 
\[
\int_{-\infty}^\infty \frac{\cos(kx)}{(x^2 + a^2)(x^2 + b^2)}dx = \Re\left(\int_{-\infty}^\infty f(x)dx\right)=\frac{\pi(-e^{-k|a|}|b|+e^{-k|b|}|a|)}{|ab|(a^2-b^2)}
\]
when $a\neq b$ and 
\[
\int_{-\infty}^\infty \frac{\cos(kx)}{(x^2 + a^2)(x^2 + b^2)}dx = \Re\left(\int_{-\infty}^\infty f(x)dx\right)=\frac{\pi e^{-k|a|}(k|a|+1)}{2|a|^3}
\]
when $a=b$.

\subsection{Part h}
Consider the substitution 4.2.7 in the text where we consider $z$ on the unit circle and take $z=e^{i\theta}$ which gives that $dz=ie^{i\theta}d\theta=izd\theta$ and $\sin\theta=(z-1/z)/2i$. Let $C$ be the unit circle. Then, 
\begin{align*}
\int_0^{2 \pi} \frac{1}{(5 - 3 \sin\theta)^2} d \theta&=\oint_C\frac{1}{(5-\frac{3}{2i}(z-1/z))^2}\frac{dz}{iz}=\oint_C\frac{1}{\left(\frac{1}{2iz}(10iz-3(z^2-1))\right)}\frac{dz}{iz}\\&=
\oint_C\frac{(2iz)^2}{iz}\frac{dz}{(3z^2-10iz-3)^2}=4i\oint_C\frac{z}{(3z-i)^2(z-3i)^2}dz\\&=
\frac{4i}{9}\oint_C\frac{z}{(z-i/3)^2(z-3i)^2}dz
\end{align*}
Now, we note that the integrand has only one singularity inside $C$ at $z=i/3$ and apply the residue theorem and theorem 4.2 in the lecture notes to get 
\begin{align*}
\int_0^{2 \pi} \frac{1}{(5 - 3 \sin\theta)^2} d\theta&=\frac{4i}{9}2\pi i\Res_{z=i/3}\frac{z}{(z-i/3)^2(z-3i)^2}=-\frac{8\pi}{9}\lim_{z\to i/3}\frac{d}{dz}\frac{z}{(z-3i)^2}\\&=
-\frac{8\pi}{9}\lim_{z\to i/3}\frac{(z-3i)^2-2z(z-3i)}{(z-3i)^4}=-\frac{8\pi}{9}\frac{-8i/3-2i/3}{(-8i/3)^3}\\&=
\frac{8\pi}{9}\frac{10i/3}{i}\left(\frac{3}{8}\right)^3=\frac{10\pi}{9}\left(\frac{3}{8}\right)^2=\frac{10\pi}{64}=\frac{5\pi}{32}.
\end{align*}

\section{Problem 3 (4.2.4)}
Let $C$ be a rectangular contour with corners at $\pm R$ and $\pm R + i$. Let $C_1$ denote the line from $-R$ to $R$, $C_2$ denote the line from $R$ to $R+i$, $C_3$ denote the line from $R+i$ to $-R+i$, and $C_4$ denote the line from $-R+i$ to $-R$. \\Now, notice that $f(z)=\frac{\cosh(az)}{\cosh(\pi z)}$ ($|a|<\pi$, $a$ real) has simple poles at the zeroes of $\cosh(\pi z)$ which occur at $i\left(\frac{1}{2}+k\right)$ for any $k\in\mathbb{Z}$ (we found this on homework 2). The only one of these singularities that occurs inside $C$ is at $z=i/2$. Then, by the residue theorem, 
\begin{align*}
\oint_Cf(z)dz&=2\pi i\Res_{z=i/2}\frac{\cosh(az)}{\cosh(\pi z)}=2\pi i\lim_{z\to i/2}\frac{\cosh(az)(z-i/2)}{\cosh(\pi z)}\\&=
2\pi i\lim_{z\to i/2}\frac{\cosh(az)+a(z-i/2)\sinh(az)}{\pi\sinh(\pi z)}=2\pi i\frac{\cosh(ai/2)}{\pi\sinh(ai/2)}\\&=
2\pi i\frac{\cosh(ai/2)}{\pi i}=2\cosh\frac{ai}{2}=2\cos\frac{a}{2}
\end{align*}
by L'Hopital's rule and the fact that $\cosh iz = \cos z$.\\
Now, break this integral up into its components. Namely, let $I_k=\oint_{C_k}f(z)dz$ for $k=1,2,3,4$. Then, $I_1=\int_{-R}^Rf(x)dx$. To analyze $I_3$ we use the sum formula for the hyperbolic cosine to get
\begin{align*}
I_3&=\int_R^{-R}\frac{\cosh(a(x+i))}{\cosh(\pi(x+i))}dx=-\int_{-R}^R\frac{\cosh(ax)\cosh(ia)+\sinh(ax)\sinh(ia)}{\cosh(\pi x)\underbrace{\cosh(i\pi)}_{-1}+\sinh(\pi x)\underbrace{\sinh(i\pi)}_0}dx\\&=
\cosh(ia)\int_{-R}^R\frac{\cosh(ax)}{\cosh(\pi x)}dx+\sinh(ia)\int_{-R}^R\underbrace{\frac{\sinh(ax)}{\cosh(\pi x)}}_{\text{odd function}}dx\\&=
\cosh(ia)I_1=\cos aI_1.
\end{align*}
Now, look to bound $f$. If we let $z=x+iy$, 
\[
|f(z)|=\left|\frac{e^{a z}+e^{-a z}}{e^{\pi z}+e^{-\pi z}}\right|=\left|\frac{e^{a x}e^{aiy}+e^{-a x}e^{-aiy}}{e^{\pi x}e^{i\pi y}+e^{-\pi x}e^{-i\pi y}}\right|
\]
By the triangle inequality and symmetry, we have that 
\[
|e^{a x}e^{aiy}+e^{-a x}e^{-aiy}|\leq|e^{a x}e^{ia y}|+|e^{-a x}e^{-ia y}|=|e^{ax}|+|e^{-ax}|=e^{|ax|}+e^{-|ax|}.
\]
By the reverse triangle inequality and symmetry, 
\[
|e^{\pi x}e^{i\pi y}+e^{-\pi x}e^{-i\pi y}|\geq\left||e^{\pi x}e^{i\pi y}|-|-e^{-\pi x}e^{-i\pi y}|\right|=\left||e^{\pi x}|-|e^{-\pi x}|\right|=e^{\pi|x|}-e^{-\pi|x|}.
\]
Thus,
\begin{align*}
|f(z)|&\leq\frac{e^{|ax|}+e^{-|ax|}}{e^{\pi|x|}-e^{-\pi|x|}}=\frac{e^{|ax|}}{e^{\pi|x|}}\frac{1+e^{-2|ax|}}{1-e^{-2\pi|x|}}=\frac{(e^{|x|})^{|a|}}{(e^{|x|)^\pi}}\frac{1+e^{-2|ax|}}{1-e^{-2\pi|x|}}\\&=
e^{|x|(|a|-\pi)}\frac{1+e^{-2|ax|}}{1-e^{-2\pi|x|}}.
\end{align*}
Note that as $|x|\to\infty$, the RHS tends to 0, because $|a|-\pi<0$, and $e^{-b}\to0$ as $b\to\infty$. As $R\to\infty$,  $|x|\to\infty$ on both $C_2$ and $C_4$, meaning that $|f(z)|\to0$ $C_2$ and $C_4$. Thus, as $R\to\infty$, $I_2=I_4=0$, because the modulus of an integral is bounded by the maximum value of the integrand on the contour times the length of said contour (theorem 2.4.2 in the text). \\
From above we have that $2\cos\frac{a}{2}=I_1+I_2+I_3+I_4$. Taking the limit of both sides, we have $2\cos\frac{a}{2}=\lim_{R\to\infty}(I_1+\cos aI_1)$. Note that $\lim_{R\to\infty}I_1=\int_{-\infty}^\infty f(x)dx$. Thus,
\[
\int_{-\infty}^\infty f(x)dx=\frac{2\cos\frac{a}{2}}{1+\cos a}=\frac{2\cos\frac{a}{2}}{2\cos^2\frac{a}{2}}=\sec\frac{a}{2}
\]
by the double angle formula. Now, we simply use the fact that $f$ is even as a function of $x$ to conculde that 
\[
\int_0^\infty \frac{\cosh(ax)}{\cosh(\pi x)} d x = \frac{1}{2} \sec\left(\frac{a}{2}\right).
\]

\section{Problem 4}
\subsection{Part a}
Let $f(z)$ be analytic at $z=k$ ($k \in \mathbb{Z}$). Then, by theorem 4.2 in the lecture notes and L'Hopital's rule, 
\[
\Res_{z=k}f(z)\cot(\pi z)=\lim_{z\to k}\frac{f(z)(z-k)}{\tan(\pi z)}=\lim_{z\to k}\frac{f'(z)(z-k)+f(z)}{\pi\sec^2(\pi z)}=\frac{f(k)}{\pi\sec^2(\pi k)}=\frac{1}{\pi}f(k).
\]
\subsection{Part b}
Let $\Gamma_N$ be a square contour with corners at $(N+1/2)(\pm 1\pm i)$, $N \in \mathbb{Z}^+$. Let $\Gamma_1$ represent the bottom of this square, $\Gamma_2$ the right, $\Gamma_3$ the top, and $\Gamma_4$ the left. Repeating an argument from homework 8, if we let $z=x+iy$, the definition of the complex cotangent gives that
\[
\cot(\pi z)=i\frac{e^{i\pi z}+e^{-i\pi z}}{e^{i\pi z}-e^{-i\pi z}}= i\frac{e^{i\pi x}e^{-\pi y}+e^{-i\pi x}e^{\pi y}}{e^{i\pi x}e^{-\pi y}-e^{-i\pi x}e^{\pi y}}.
\]
Now, by the triangle inequality
\[
|e^{i\pi x}e^{-\pi y}+e^{-i\pi x}e^{\pi y}|\leq|e^{i\pi x}e^{-\pi y}|+|e^{-i\pi x}e^{\pi y}|=e^{-\pi y}+e^{\pi y}=e^{-\pi |y|}+e^{\pi |y|}
\]
where the last step follows by symmetry. Similarly, the reverse triangle inequality gives that 
\[
|e^{i\pi x}e^{-\pi y}-e^{-i\pi x}e^{\pi y}|\geq\left||e^{i\pi x}e^{-\pi y}|-|e^{-i\pi x}e^{\pi y}|\right|=|e^{-\pi y}-e^{\pi y}|=e^{\pi |y|}-e^{-\pi |y|}
\]
where the last step follows from symmetry and the fact that $e^a\geq e^{-a}$ for $a\geq0$. Then, 
\begin{align*}
|\cot(\pi z)|&=\left|i\frac{e^{i\pi x}e^{-\pi y}+e^{-i\pi x}e^{\pi y}}{e^{i\pi x}e^{-\pi y}-e^{-i\pi x}e^{\pi y}}\right|\leq\frac{e^{-\pi |y|}+e^{\pi |y|}}{e^{\pi |y|}-e^{-\pi |y|}}\\&=
\frac{e^{\pi |y|}+e^{-\pi |y|}}{e^{\pi |y|}-e^{-\pi |y|}}=\coth(\pi|y|)=\frac{e^{2\pi |y|}+1}{e^{2\pi |y|}-1}.
\end{align*} 
Thus, on $\Gamma_1$ and $\Gamma_3$, 
\begin{align*}
|\cot(\pi z)|&\leq\coth(\pi|y|)=\coth(\pi(N+\frac{1}{2}))\leq\coth(\pi/2)=\frac{e^\pi+1}{e^\pi-1}=1+\frac{2}{e^\pi-1}\\&<
1+\frac{1}{2^3-1}=\frac{9}{7}<2
\end{align*}
by the fact that $\coth(z)$ is monotonically  decreasing for $z>0$ and $e>2,\pi>3$. 
On $\Gamma_2$ and $\Gamma_4$, 
\begin{align*}
|\cot(\pi z)|&=\left|i\frac{e^{2i\pi x}e^{-2\pi y}+1}{e^{2i\pi x}e^{-2\pi y}-1}\right|=\left|\frac{e^{\pm i\pi2(N+1/2) x}e^{-2\pi y}+1}{e^{\pm i\pi2(N+1/2) x}e^{-2\pi y}-1}\right|=\left|\frac{-e^{-2\pi y}+1}{-e^{-2\pi y}-1}\right|\\&=
\left|\frac{1-e^{-2\pi y}}{1+e^{-2\pi y}}\right|=\left|\frac{e^{\pi y}-e^{-\pi y}}{e^{\pi y}+e^{-\pi y}}\right|=|\tanh(y)|\leq1<2
\end{align*}
by the definition and range of the hyperbolic tangent on the real line. 
Thus, $|\cot(\pi z)|\leq2$ for $z$ on all of $\Gamma_N$. 

\subsection{Part c}
Suppose $f(z)=p(z)/q(z)$, where $p(z)$ and $q(z)$ are polynomials, so that the degree of $q(z)$ is at least two more than	the degree of $p(z)$. We apply theorem 2.4.2 in the text to find that 
\begin{align*}
\lim_{N\to \infty}\left|\oint_{\Gamma_N}\frac{p(z)}{q(z)}\cot(\pi z)dz\right|\leq\lim_{N\to \infty}\sup_{z\in\Gamma_N}\left|\frac{p(z)}{q(z)}\cot(\pi z)\right|L
\end{align*}
where $L$ is the length of $\Gamma_N$. From the definition of $\Gamma_N$, $L=4*2(N+1/2)$, so 
\begin{align*}
\lim_{N\to \infty}\left|\oint_{\Gamma_N}\frac{p(z)}{q(z)}\cot(\pi z)dz\right|&\leq\lim_{N\to \infty}\sup_{z\in\Gamma_N}\left|\frac{p(z)}{q(z)}\right||\cot(\pi z)|8\left(N+\frac{1}{2}\right)\\&\leq
16\left(N+\frac{1}{2}\right)\sup_{z\in\Gamma_N}\left|\frac{p(z)}{q(z)}\right|
\end{align*}
by part b. Now, let $p(z)=a_mz^m+\ldots+a_0$ and $q(z)=b_rz^r+\ldots+b_0$ where $r\geq m+2$. Then, by the triangle inequality and reverse triangle inequality, 
\begin{align*}
\left|\frac{p(z)}{q(z)}\right|=\frac{|a_mz^m+\ldots+a_0|}{|b_rz^r+\ldots+b_0|}\leq\frac{|a_m||z|^m+\ldots+|a_0|}{\left||b_r||z|^r-\ldots-|b_0|\right|}
\end{align*}
If we consider $N$ large enough so that $|b_r||z|^r\geq|b_{r-1}||z|^{r-1}-\ldots-|b_0|$ for $z\in\Gamma_N$ and also note that $N+\frac{1}{2}\leq|z|\leq\sqrt{2}\left(N+\frac{1}{2}\right)$ if $z\in\Gamma_N$ because the maximum modulus on a square is achieved at the corners and the minimum on the real or imaginary axes, 
\begin{align*}
\lim_{N\to \infty}\left|\oint_{\Gamma_N}\frac{p(z)}{q(z)}\cot(\pi z)dz\right|&\leq\lim_{N\to \infty}16\left(N+\frac{1}{2}\right)\frac{|a_m|(\sqrt{2}\left(N+\frac{1}{2}\right))^m+\ldots+|a_0|}{|b_r|\left(N+\frac{1}{2}\right)^r-|b_{r-1}|\left(\sqrt{2}\left(N+\frac{1}{2}\right)\right)^{r-1}-\ldots-|b_0|}\\&=0
\end{align*}
by the properties of real limits because this is equivalent to having $N^{m+1}$ in the numerator and $N^r$ in the denominator and $r>m+1$.

\subsection{Part d}
Suppose that $q(z)$ has no roots at the integers and let $\{z_j\}$ represent the roots of $q(z)$. Then, as $N\to\infty$, part c and the residue theorem (noting that the singularities of $\cot(\pi z)$ occur at the integers) give that 
\[
0=\oint_{\Gamma_N}f(z)\cot(\pi z)dz=2\pi i\left(\sum_j\Res_{z=z_j} f(z)\cot(\pi z)+\sum_{k=-\infty}^\infty\Res_{z=k}f(z)\cot(\pi z)\right)
\]
Thus, 
\[
\sum_j\Res_{z=z_j} f(z)\cot(\pi z)=-\sum_{k=-\infty}^\infty\Res_{z=k}f(z)\cot(\pi z)=-\sum_{k=-\infty}^\infty\frac{f(k)}{\pi}
\]
by part a. Thus,
\[
\sum_{k=-\infty}^\infty\frac{p(k)}{q(k)}=-\pi\sum_j\Res_{z=z_j} f(z)\cot(\pi z).
\]

\section{Problem 5}
\subsection{Part a}
Note that the function $\frac{1}{z^2+1}=\frac{1}{(z-i)(z+i)}$ has simple poles at $z=\pm i$. Using the result of problem 4,
\begin{align*}
\sum_{k=-\infty}^\infty \frac{1}{k^2+1}&=-\pi\left(\Res_{z=i}\frac{\cot(\pi z)}{z^2+1}+\Res_{z=-i}\frac{\cot(\pi z)}{z^2+1}\right)=-\pi\lim_{z\to i}\frac{\cot(\pi z)}{z+i}-\pi\lim_{z\to -i}\frac{\cot(\pi z)}{z-i}\\&=
\frac{-\pi}{2i}\cot(\pi i)+\frac{\pi}{2i}\cot(-\pi i).
\end{align*}
Now, the fact that $\cot(z)$ is odd as a function of $z$ and that $\coth{z}=i\cot(iz)$ gives that
\[
\sum_{k=-\infty}^\infty \frac{1}{k^2+1}=\frac{-\pi}{2i}\cot(\pi i)+\frac{-\pi}{2i}\cot(\pi i)=-\frac{\pi}{i}\cot(\pi i)=\pi\coth\pi.
\]
\subsection{Part b}
Now, note that 
\[
\frac{1}{z^4+1}=\frac{1}{(z+\sqrt i)(z-\sqrt{i})(z+i\sqrt i)(z-i\sqrt{i})}.
\]
Similarly to part a (again using the fact that cotangent is an odd function), we have that
\begin{align*}
\sum_{k=-\infty}^\infty \frac{1}{k^4+1}&=-\pi\biggr(\Res_{z=\sqrt i}\frac{\cot(\pi z)}{(z+\sqrt i)(z-\sqrt{i})(z^2+i)}+\Res_{z=-\sqrt i}\frac{\cot(\pi z)}{(z+\sqrt i)(z-\sqrt{i})(z^2+i)}\\&+
\Res_{z=-i\sqrt i}\frac{\cot(\pi z)}{(z+i\sqrt i)(z-i\sqrt{i})(z^2-i)}+\Res_{z=i\sqrt i}\frac{\cot(\pi z)}{(z+i\sqrt i)(z-i\sqrt{i})(z^2-i)}\biggr)\\&=
-\pi\left(\frac{\cot(\pi\sqrt{i})}{2i(2\sqrt{i})}+\frac{\cot(-\pi\sqrt{i})}{2i(-2\sqrt{i})}+\frac{\cot(-\pi i\sqrt{i})}{-2i\sqrt{i}(-2i)}+\frac{\cot(\pi i\sqrt{i})}{2i\sqrt{i}(-2i)}\right)\\&=
-\pi\left(\frac{\cot(\pi\sqrt{i})}{4i\sqrt{i}}+\frac{\cot(\pi\sqrt{i})}{4i\sqrt{i}}+\frac{\cot(\pi i\sqrt{i})}{4\sqrt{i}}+\frac{\cot(\pi i\sqrt{i})}{4\sqrt{i}}\right)\\&=
-\pi\left(\frac{\cot(\pi\sqrt{i})}{2i\sqrt{i}}+\frac{\cot(\pi i\sqrt{i})}{2\sqrt{i}}\right)=-\pi\left(\frac{-\sqrt{i}}{2}\cot(\pi\sqrt{i})-\frac{i\sqrt{i}}{2}\cot(\pi i\sqrt{i})\right)\\&=
\frac{\pi}{2}\left(e^{\frac{i\pi}{4}}\cot(\pi e^{\frac{i\pi}{4}})+e^{\frac{3i\pi}{4}}\cot(\pi e^{\frac{3i\pi}{4}})\right).
\end{align*}
However, we know the answer should be a real number. To verify this, we input our result into Wolfram-Alpha which gives
\[
\frac{\pi}{2}\left(e^{\frac{i\pi}{4}}\cot(\pi e^{\frac{i\pi}{4}})+e^{\frac{3i\pi}{4}}\cot(\pi e^{\frac{3i\pi}{4}})\right)=-\frac{\pi(\sin(\sqrt{2}\pi)+\sinh(\sqrt{2}\pi))}{\sqrt{2}(\cos(\sqrt{2}\pi)\cosh(\sqrt{2}\pi))}
\]
which is clearly a real number. 

\subsection{Part c}
Now, we use the result from problem 4 and theorem 4.2 in the lecture notes to compute
\begin{align*}
\sum_{k=-\infty}^\infty \frac{1}{(4k^2-1)^2}&=-\pi\left(\Res_{z=-1/2}\frac{\cot(\pi z)}{16(z-1/2)^2(z+1/2)^2}+\Res_{z=1/2}\frac{\cot(\pi z)}{16(z-1/2)^2(z+1/2)^2}\right)\\&=
-\frac{\pi}{16}\lim_{z\to-1/2}\frac{d}{dz}\frac{\cot(\pi z)}{(z-1/2)^2}-\frac{\pi}{16}\lim_{z\to1/2}\frac{d}{dz}\frac{\cot(\pi z)}{(z+1/2)^2}\\&=
-\frac{\pi}{16}\lim_{z\to-1/2}\frac{-\pi\csc^2(\pi z)(z-\frac{1}{2})^2-2(z-\frac{1}{2})\cot(\pi z)}{(z-1/2)^4}\\&-\frac{\pi}{16}\lim_{z\to1/2}\frac{-\pi\csc^2(\pi z)(z+\frac{1}{2})^2-2(z+\frac{1}{2})\cot(\pi z)}{(z-1/2)^4}\\&=
-\frac{\pi}{16}(-\pi)-\frac{\pi}{16}(-\pi)=\frac{\pi^2}{8}.
\end{align*}

\end{document}
