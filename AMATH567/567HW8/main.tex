\documentclass{article}
\usepackage[utf8]{inputenc}
\usepackage{listings}
\usepackage{multimedia} % to embed movies in the PDF file
\usepackage{graphicx}
\usepackage{comment}
\usepackage[english]{babel}
\usepackage{amsmath}
\usepackage{amsfonts}
\usepackage{subfigure}
\usepackage{wrapfig}
\usepackage{multirow}
\usepackage{tikz}
\usepackage{verbatim}

\newtheorem{theorem}{Theorem}[section]
\newtheorem{lemma}[theorem]{Lemma}
\newtheorem{corollary}[theorem]{Corollary}
%\newtheorem{algorithm}[theorem]{Algorithm}
\newtheorem{remark}[theorem]{Remark}
\newenvironment{proof}{\noindent {\bf Proof:} }{\hfill $\Box$ \\[2ex] }
\newenvironment{keywords}{\begin{quote} {\bf Key words} }
                         {\end{quote} }
\newenvironment{AMS}{\begin{quote} {\bf AMS subject classifications} }
                         {\end{quote} }


\newcommand{\eref}[1]{\mbox{\rm(\ref{#1})}}
\newcommand{\tref}[1]{\mbox{\rm\ref{#1}}}
\newcommand{\set}[2]{\left\{ #1 \; : \; #2 \right\} }
\newcommand{\deq}{\raisebox{0pt}[1ex][0pt]{$\stackrel{\scriptscriptstyle{\rm def}}{{}={}}$}}

\newcommand {\DS} {\displaystyle}

\newcommand{\real}{\mathbb{R}}
\newcommand{\compl}{\mathbb{C}}



\newcommand {\half} {\mbox{$\frac{1}{2}$}}
\newcommand{\force}{{\mathbf{f}}}
\newcommand{\strain}{{\boldsymbol{\varepsilon}}}
\newcommand{\stress}{{\boldsymbol{\sigma}}}
\renewcommand{\div}{{\boldsymbol{\nabla}}}

\newcommand {\cA} {{\cal A}}
\newcommand {\cB} {{\cal B}}
\newcommand {\cC} {{\cal C}}
\newcommand {\cD} {{\cal D}}
\newcommand {\cE} {{\cal E}}
\newcommand {\cL} {{\cal L}}
\newcommand {\cP} {{\cal P}}
\newcommand {\cQ} {{\cal Q}}
\newcommand {\cR} {{\cal R}}
\newcommand {\cV} {{\cal V}}
\newcommand {\cW} {{\cal W}}
\newcommand {\CH} {{\cal H}}
\newcommand {\CS} {{\cal S}}


\newcommand{\bzero}{\mathbf{0}}
\newcommand{\ba}{\mathbf{a}}
\newcommand{\bb}{\mathbf{b}}
\newcommand{\bc}{\mathbf{c}}
\newcommand{\bd}{\mathbf{d}}
\newcommand{\be}{\mathbf{e}}
\newcommand{\bff}{\mathbf{f}}
\newcommand{\bg}{\mathbf{g}}
\newcommand{\bh}{\mathbf{h}}
\newcommand{\bn}{\mathbf{n}}
\newcommand{\bp}{\mathbf{p}}
\newcommand{\bq}{\mathbf{q}}
\newcommand{\br}{\mathbf{r}}
\newcommand{\bs}{\mathbf{s}}
\newcommand{\bt}{\mathbf{t}}
\newcommand{\bu}{\mathbf{u}}
\newcommand{\bv}{\mathbf{v}}
\newcommand{\bw}{\mathbf{w}}
\newcommand{\bx}{\mathbf{x}}
\newcommand{\by}{\mathbf{y}}
\newcommand{\bz}{\mathbf{z}}
\newcommand{\bA}{\mathbf{A}}
\newcommand{\bB}{\mathbf{B}}
\newcommand{\bC}{\mathbf{C}}
\newcommand{\bD}{\mathbf{D}}
\newcommand{\bE}{\mathbf{E}}
\newcommand{\bF}{\mathbf{F}}
\newcommand{\bG}{\mathbf{G}}
\newcommand{\bH}{\mathbf{H}}
\newcommand{\bI}{\mathbf{I}}
\newcommand{\bJ}{\mathbf{J}}
\newcommand{\bK}{\mathbf{K}}
\newcommand{\bL}{\mathbf{L}}
\newcommand{\bM}{\mathbf{M}}
\newcommand{\bN}{\mathbf{N}}
\newcommand{\bO}{\mathbf{O}}
\newcommand{\bP}{\mathbf{P}}
\newcommand{\bQ}{\mathbf{Q}}
\newcommand{\bR}{\mathbf{R}}
\newcommand{\bS}{\mathbf{S}}
\newcommand{\bU}{\mathbf{U}}
\newcommand{\bV}{\mathbf{V}}
\newcommand{\bW}{\mathbf{W}}
\newcommand{\bX}{\mathbf{X}}
\newcommand{\bY}{\mathbf{Y}}
\newcommand{\bZ}{\mathbf{Z}}

\newcommand{\cO}{ {\cal O} }
\newcommand{\CT}{ {\cal T} }
\newcommand{\IL}{{\mathbb L}}
\newcommand{\sIL}{{{{\mathbb L}_s}}}
\newcommand{\bOmega}{{\boldsymbol{\Omega}}}
\newcommand{\bPsi}{{\boldsymbol{\Psi}}}

\newcommand{\bgamma}{{\boldsymbol{\gamma}}}
\newcommand{\bmu}{{\boldsymbol{\mu}}}
\newcommand{\blambda}{{\boldsymbol{\lambda}}}
\newcommand{\bLambda}{{\boldsymbol{\Lambda}}}
\newcommand{\bpi}{{\boldsymbol{\pi}}}
\newcommand{\bPi}{{\boldsymbol{\Pi}}}
\newcommand{\bphi}{{\boldsymbol{\phi}}}
\newcommand{\bPhi}{{\boldsymbol{\Phi}}}
\newcommand{\bpsi}{{\boldsymbol{\psi}}}
\newcommand{\btheta}{{\boldsymbol{\theta}}}
\newcommand{\bTheta}{{\boldsymbol{\Theta}}}
\newcommand{\bSigma}{{\boldsymbol{\Sigma}}}
\newcommand{\sump}{\sideset{}{^{'}}\sum} 
\DeclareMathOperator*{\Res}{Res}
\DeclareMathOperator{\OO}{O}
\DeclareMathOperator{\oo}{o}
\DeclareMathOperator{\erfc}{erfc}
\def\Xint#1{\mathchoice
   {\XXint\displaystyle\textstyle{#1}}%
   {\XXint\textstyle\scriptstyle{#1}}%
   {\XXint\scriptstyle\scriptscriptstyle{#1}}%
   {\XXint\scriptscriptstyle\scriptscriptstyle{#1}}%
   \!\int}
\def\XXint#1#2#3{{\setbox0=\hbox{$#1{#2#3}{\int}$}
     \vcenter{\hbox{$#2#3$}}\kern-.5\wd0}}
\def\ddashint{\Xint=}
\def\pvint{\Xint-}





\title{AMATH 567 Homework 8}
\author{Cade Ballew \#2120804}
\date{November 24, 2021}

\begin{document}

\maketitle

\section{Problem 1 (3.6.7)}
Consider \[
    \pi \cot(\pi z) - \left(\frac{1}{z} + \sump_{j = -\infty}^{\infty} \left(\frac{1}{z - j} + \frac{1}{j}\right)\right) = h(z).
    \]

\subsection{Part a}
To see that $h(z)$ is periodic of period 1, we show that the LHS is periodic of period 1. Clearly, $\pi \cot(\pi (z+1))=\pi \cot(\pi z)$, because $\cot$ is $\pi$-periodic. Let $S(z)$ denote the term in parentheses. Then,
\begin{align*}
S(z+1)&=\frac{1}{z+1}+\sump_{j = -\infty}^{\infty} \left(\frac{1}{z+1 - j} + \frac{1}{j}\right)\\&=\frac{1}{z+1}+\left(\frac{1}{z}+1\right)+\sump_{\substack{j = -\infty,\\j\neq1}}^{\infty} \left(\frac{1}{z +1- j} + \frac{1}{j}\right)\\&=
\frac{1}{z}+\left(\frac{1}{z+1}+1\right)+\sump_{\substack{j = -\infty,\\j\neq-1}}^{\infty} \left(\frac{1}{z - j} + \frac{1}{j+1}\right)
\end{align*}
Now, consider the series 
\begin{align*}
\sump_{\substack{j = -\infty,\\j\neq-1}}^{\infty} \left(\frac{1}{j} - \frac{1}{j+1}\right)&=\sum_{j=-\infty}^{-2}\left(\frac{1}{j} - \frac{1}{j+1}\right)+\sum_{j=1}^{\infty}\left(\frac{1}{j} - \frac{1}{j+1}\right)\\&=
\lim_{N\to-\infty}\left(\frac{1}{N} - \frac{1}{-2+1}\right)+\lim_{N\to\infty}\left(\frac{1}{1} - \frac{1}{N+1}\right)\\&=
1+1=2.
\end{align*}
Note that this is uniformly convergent as a function of $z$, because it is constant as a function of $z$. Now,
\begin{align*}
S(z+1)&=-2+\frac{1}{z}+\left(\frac{1}{z+1}+1\right)+\sump_{\substack{j = -\infty,\\j\neq-1}}^{\infty} \left(\frac{1}{z - j} + \frac{1}{j+1}\right)+\sump_{\substack{j = -\infty,\\j\neq-1}}^{\infty} \left(\frac{1}{j} - \frac{1}{j+1}\right)\\&=
\frac{1}{z}+\left(\frac{1}{z-(-1)}-1\right)+\sump_{\substack{j = -\infty,\\j\neq-1}}^{\infty} \left(\frac{1}{z - j} + \frac{1}{j}\right)\\&=
\frac{1}{z}+\sump_{j = -\infty}^{\infty} \left(\frac{1}{z - j} + \frac{1}{j}\right)=S(z)
\end{align*}
where we have combined the series because they are both uniformly convergent and reinserted the $j=-1$ term. Thus, the entire LHS is periodic of period 1, so it must hold that $h(z)$ is periodic of period 1. 

\subsection{Part b}
We first wish to show that as $y\to\pm\infty$, $\pi\cot(\pi z)$ is bounded for $x\in[0,1]$ where $z=x+iy$. To see this, use the definition of the complex cotangent which gives that
\[
\pi\cot(\pi z)=\pi i\frac{e^{i\pi z}+e^{-i\pi z}}{e^{i\pi z}-e^{-i\pi z}}=\pi i\frac{e^{i\pi x}e^{-\pi y}+e^{-i\pi x}e^{\pi y}}{e^{i\pi x}e^{-\pi y}-e^{-i\pi x}e^{\pi y}}.
\]
Now, by the triangle inequality
\[
|e^{i\pi x}e^{-\pi y}+e^{-i\pi x}e^{\pi y}|\leq|e^{i\pi x}e^{-\pi y}|+|e^{-i\pi x}e^{\pi y}|=e^{-\pi y}+e^{\pi y}=e^{-\pi |y|}+e^{\pi |y|}
\]
where the last step follows by symmetry. Similarly, the reverse triangle inequality gives that 
\[
|e^{i\pi x}e^{-\pi y}-e^{-i\pi x}e^{\pi y}|\geq\left||e^{i\pi x}e^{-\pi y}|-|e^{-i\pi x}e^{\pi y}|\right|=|e^{-\pi y}-e^{\pi y}|=e^{\pi |y|}-e^{-\pi |y|}
\]
where the last step follows from symmetry and the fact that $e^a\geq e^{-a}$ for $a\geq0$. Then, 
\begin{align*}
|\pi\cot(\pi z)|&=\left|\pi i\frac{e^{i\pi x}e^{-\pi y}+e^{-i\pi x}e^{\pi y}}{e^{i\pi x}e^{-\pi y}-e^{-i\pi x}e^{\pi y}}\right|\leq\pi\frac{e^{-\pi |y|}+e^{\pi |y|}}{e^{\pi |y|}-e^{-\pi |y|}}\\&=
\pi\frac{e^{-\pi |y|}+e^{\pi |y|}}{e^{\pi |y|}-e^{-\pi |y|}}=\pi\frac{e^{-2\pi |y|}+1}{1-e^{-2\pi |y|}}
\end{align*}
As $y\to\pm\infty$, this tends to $\pi$, because $e^a\to0$ ad $a\to-\infty$. Thus, $\pi\cot{\pi z}$ is bounded as $y\to\pm\infty$.\\

Now, if we define $S(z)$ as in part a, we can observe that 
\begin{align*}
S(z)&=\frac{1}{z}+\sum_{j=-\infty}^{-1}\left(\frac{1}{z - j} + \frac{1}{j}\right)+\sum_{j=1}^{\infty}\left(\frac{1}{z - j} + \frac{1}{j}\right)\\&=
\frac{1}{z}+\sum_{j=\infty}^{1}\left(\frac{1}{z + j} + \frac{1}{-j}\right)+\sum_{j=1}^{\infty}\left(\frac{1}{z - j} + \frac{1}{j}\right)\\&=
\frac{1}{z}+\sum_{j=1}^{\infty}\left(\frac{1}{z + j} - \frac{1}{j}\right)+\sum_{j=1}^{\infty}\left(\frac{1}{z - j} + \frac{1}{j}\right)=
\frac{1}{z}+\sum_{n=1}^{\infty}\left(\frac{1}{z + n} + \frac{1}{z-n}\right)\\&=
\frac{1}{z}+\sum_{n=1}^{\infty}\frac{2n}{z^2 - n^2}
\end{align*}
where we have negated the dummy variable in the first summation and are able to combine the series due to unifrom convergence. Then, the triangle inequality and the inequality $|z^2-n^2|\geq \frac{1}{\sqrt{2}}(y^2+n^2)$ where $z=x+iy$ and $x\in [0,1]$, $|y|\geq 2$ give that
\begin{align*}
|S(z)|&\leq \frac{1}{|z|} + \sum_{n=1}^\infty \frac{2|z|}{|z^2 - n^2|}\leq\frac{1}{|z|} + \sum_{n=1}^\infty \frac{2\sqrt2y}{|z^2 - n^2|}\leq\frac{1}{|z|} + y\sum_{n=1}^\infty \frac{2\sqrt2}{(y^2 + n^2)/\sqrt{2}}\\&=\frac{1}{|z|} + 4y\sum_{n=1}^\infty \frac{1}{y^2 + n^2}.
\end{align*}
Also, 
\[
y\sum_{n=1}^\infty \frac{1}{y^2 + n^2}=y\sum_{n=1}^\infty \frac{1}{y^2(1 + n^2/y^2)}=\frac{1}{y}\sum_{n=1}^\infty \frac{1}{1 + (n/y)^2}\leq\frac{1}{y}\int_0^\infty\frac{1}{1+(r/y)^2}dr
\]
where the last step follows from the integral test because our integrand is strictly decreasing and these are right Riemann sums. Thus, taking the $u$-substitution $u=r/y$,
\[
y\sum_{n=1}^\infty \frac{1}{y^2 + n^2}=\frac{1}{y}\int_0^\infty\frac{1}{1+u^2}ydu=\int_0^\infty\frac{1}{1+u^2}du=\left[\arctan u\right]_0^\infty=\pi/2.
\]
Thus, the reverse triangle inequality gives that
\[
|S(z)|\leq\frac{1}{|z^2|}+4\frac{\pi}{2}\leq\frac{1}{\left||x|-|y|\right|}+2\pi=\frac{1}{|y-x|}+2\pi
\]
for $y$ sufficiently large. Clearly, this tends to $2\pi$ as $y\to\infty$, so $S(z)$ is bounded for $0<x<1$ and $y\to\infty$. The same argument works for $y\to-\infty$, so we have that $S(z)$ is bounded for $0<x<1$.\\
Now, note that from page 87 of the lecture notes we have that  $\pi\cot{\pi z}$ has simple poles at $z_j=j$ for $j\in\mathbb{Z}$ and these poles have residue 1. Thus, its only pole on the strip $0\leq\Re(z)<1$ is at $z=0$. However, this cancels with the $1/z$ term from $S(z)$ when we consider the entire LHS. Thus, $h(z)$ is bounded and analytic in this strip, and the periodicity we derived in part a gives that it is bounded and analytic everywhere. Therefore, Liouville's theorem gives that $h(z)$ must be constant.  \\
Now, in search of extra credit, we prove the bound stated above.
Letting $n\in\mathbb{N}$ and $z=x+iy$, 
\begin{align*}
|z^2-n^2|^2&=|(x+iy)^2-n^2|^2=|(x^2+2ixy-y^2)-n^2|^2=|(x^2-y^2-n^2)+i2xy|^2\\&=
(x^2-y^2-n^2)^2+4x^2y^2=(x^2-y^2)^2-2n^2(x^2-y^2)+n^4+4x^2y^2\\&=
(x^2+y^2)^2-2n^2(x^2-y^2)+n^4\geq y^4-2n^2x^2+2n^2y^2+n^4\\&=
-2n^2x^2+(y^2+n^2)^2=-2n^2x^2+\frac{1}{2}(y^2+n^2)^2+\frac{1}{2}(y^2+n^2)^2.
\end{align*}
Now, we use the fact that $x\in[0,1]$ to get that $-2n^2x^2\geq-2n^2$ which gives
\begin{align*}
|z^2-n^2|^2&\geq-2n^2+\frac{1}{2}y^4+y^2n^2+\frac{1}{2}n^4+\frac{1}{2}(y^2+n^2)^2\\&=
n^2(y^2-2)+\frac{1}{2}(y^4+n^4)+\frac{1}{2}(y^2+n^2)^2\\&\geq
n^2(y^2-2)+\frac{1}{2}(y^2+n^2)^2\geq\frac{1}{2}(y^2+n^2)^2
\end{align*}
where the last line follows because $|y|\geq2$. Now, we simply take the square root of both sides to get that $|z^2-n^2|\geq\frac{1}{\sqrt{2}}(y^2+n^2)$, the desired inequality.  

\subsection{Part c}
Now, to see that both terms on the LHS are odd in $z$, note that cotangent is an odd function, so $\pi\cot(\pi z)$ is also odd. Additionally, 
\[
S(-z)=\frac{1}{-z}+\sum_{n=1}^\infty \frac{2(-z)}{(-z)^2 - n^2}=-\frac{1}{z}-\sum_{n=1}^\infty \frac{2z}{z^2 - n^2}=-S(z).
\]
Therefore the LHS is odd in $z$, meaning that $h(z)$ is odd. However, we know that $h(z)$ is constant, so we can write $h(z)=c$. Then, $h(z)=-h(-z)$, so $c=-c$, meaning that $c=0$ and $h(z)=0$.

\section{Problem 2 (3.3.1)} 
We know that 
\[
\frac{\sin \pi z}{\pi z}=\prod_{j=1}^\infty
\left(1-\frac{z^2}{j^2}\right).
\]
\subsection{Part a}
First, the Taylor series for sine gives that 
\[
\frac{\sin \pi z}{\pi z}=\frac{1}{\pi z}\left(\pi z-\frac{(\pi z)^3}{3!}+\frac{(\pi z)^5}{5!}-\ldots\right)=1-\frac{\pi^2}{3!}z^2+\frac{\pi^4}{5!}z^4-\dots
\]
meaning that the coefficient for the $z^2$ term on the RHS should be $-\frac{\pi^2}{3!}$. Looking at the partial product, we know that $\prod_{j=1}^N
\left(1-\frac{z^2}{j^2}\right)$ has $z^2$ term $\sum_{j=1}^N-\frac{1}{j^2}$. Since the infinite product is just the limit of the partial product as $N\to\infty$, we get that the infinite product has $z^2$ coefficients 
\[
\lim_{N\to\infty}\sum_{j=1}^N-\frac{1}{j^2}=-\sum_{j=1}^\infty\frac{1}{j^2}.
\]
Equating these coefficients and dividing by $-1$, we get that 
\[
\sum_{j=1}^\infty\frac{1}{j^2}=\frac{\pi^2}{3!}=\frac{\pi^2}{6}.
\]
\subsection{Part b}
Now, we note that the LHS has $z^4$ coefficient $\frac{\pi^4}{5!}$. If we again consider the RHS as a partial product from $j=1$ to $N$, we get that the $z^4$ coefficient is $\sum_{j\neq k}\frac{1}{j^2k^2}$ where $j$ and $k$ are taken from 1 to $N$. By the same argument as in part a when we take the limit as $N\to\infty$, we get that the $z^4$ coefficient is the same sum but with $j$ and $k$ taken from 1 to infinity. Manipulating this using our result from part a, 
\begin{align*}
\sum_{j\neq k}\frac{1}{j^2k^2}&=\frac{1}{2}\sum_{j=1}^{\infty}\sum_{\substack{k = 1,\\k\neq j}}^\infty\frac{1}{j^2k^2}=\frac{1}{2}\sum_{j=1}^{\infty}\left(\sum_{k=1}^\infty\frac{1}{j^2k^2}-\frac{1}{j^4}\right)=\frac{1}{2}\sum_{j=1}^{\infty}\frac{1}{j^2}\left(\sum_{k=1}^\infty\frac{1}{k^2}-\frac{1}{j^2}\right)\\&=
\frac{1}{2}\sum_{j=1}^{\infty}\frac{1}{j^2}\left(\frac{\pi^2}{6}-\frac{1}{j^2}\right)=\frac{1}{2}\frac{\pi^2}{6}\sum_{j=1}^{\infty}\frac{1}{j^2}-\frac{1}{2}\sum_{j=1}^{\infty}\frac{1}{j^4}=\frac{\pi^4}{72}-\frac{1}{2}\sum_{j=1}^{\infty}\frac{1}{j^4}
\end{align*}
where we have split the sums because these are p-series which are absolutely convergent. Thus,
\[
\sum_{j=1}^{\infty}\frac{1}{j^4}=2\left(\frac{\pi^4}{72}-\frac{\pi^4}{5!}\right)=2\left(\frac{\pi^2}{72}-\frac{\pi^4}{120}\right)=\frac{\pi^4}{90}.
\]

\section{Problem 3 (4.1.2)}
\subsection{Part a}
Let \[
f(z)=\frac{z^2 + 1}{z^2 - a^2}
\]
where $a^2 < 1$.
\subsubsection{Part i}
Assuming $a\neq0$, the residue theorem gives that 
\[
\frac{1}{2\pi i}\oint_Cf(z)dz=\Res_{z=-a}\frac{z^2+1}{(z-a)(z+a)}+\Res_{z=a}\frac{z^2+1}{(z-a)(z+a)}=\frac{a^2+1}{-2a}+\frac{a^2+1}{2a}=0.
\]
In the case where $a=0$, 
\[
\frac{1}{2\pi i}\oint_Cf(z)dz=\Res_{z=0}\frac{z^2+1}{z^2}=\Res_{z=0}\left(1+\frac{1}{z^2}\right)=0,
\]
the same solution. 
\subsubsection{Part ii}
Enclosing the singular points outside $C$, the residue theorem gives that 
\[
\frac{1}{2\pi i}\oint_Cf(z)dz=\Res_{z=\infty}f(z)=\Res_{t=0}\frac{1}{t^2}\frac{1/t^2+1}{1/t^2-a^2}=0.
\]
This is because the function that we are taking the residue of is even as a function of $t$, meaning that its Laurent series cannot have coefficients of 0 for all its odd terms because this would require that $c_j=-c_j$ for all odd $j\in\mathbb{Z}$. Thus, $c_{-1}=0$. \\
Clearly, we have obtained the same result with both methods.

\subsection{Part b}
Let \[
f(z)=\frac{z^2 + 1}{z^3}.
\]
\subsubsection{Part i}
By the residue theorem, 
\[
\frac{1}{2\pi i}\oint_Cf(z)dz=\Res_{z=0}\frac{z^2+1}{z^3}=\Res_{z=0}\left(\frac{1}{z^3}+\frac{1}{z}\right)=1.
\]
\subsubsection{Part ii}
Enclosing the singular points outside $C$, the residue theorem gives that 
\[
\frac{1}{2\pi i}\oint_Cf(z)dz=\Res_{z=\infty}f(z)=\Res_{t=0}\frac{1/t^2+1}{1/t^3}\frac{1}{t^2}=\Res_{t=0}\frac{1/t^2+1}{1/t}=\Res_{t=0}\left(\frac{1}{t}+t\right)=1.
\]
Clearly, we have obtained the same result with both methods.

\subsection{Part c}
Let $f(z)=z^2 e^{-1/z}$.
\subsubsection{Part i}
By the residue theorem, 
\[
\frac{1}{2\pi i}\oint_Cf(z)dz=\Res_{z=0}z^2 e^{-1/z}=\Res_{z=0}z^2\sum_{j=0}^\infty\frac{(-1/z)^j}{j!}=\Res_{z=0}\sum_{j=0}^\infty\frac{(-1)^j}{j!}z^{2-j}=-\frac{1}{6}.
\]
\subsubsection{Part ii}
Enclosing the singular points outside $C$, the residue theorem gives that 
\begin{align*}
\frac{1}{2\pi i}\oint_Cf(z)dz&=\Res_{z=\infty}f(z)=\Res_{z=0}\frac{1}{t^2}\frac{1}{t^2}e^{-t}=\Res_{z=0}\frac{1}{t^4}\sum_{j=0}^\infty\frac{(-t)^j}{j!}\\&=
\Res_{z=0}\sum_{j=0}^\infty\frac{(-1)^j}{j!}t^{j-4}=-\frac{1}{6}.
\end{align*}
Clearly, we have obtained the same result with both methods.
\end{document}
