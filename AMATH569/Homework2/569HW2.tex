\documentclass{article}
\usepackage[utf8]{inputenc}
\usepackage{listings}
\usepackage{multimedia} % to embed movies in the PDF file
\usepackage{graphicx}
\usepackage{comment}
\usepackage[english]{babel}
\usepackage{amsmath}
\usepackage{amsfonts}
\usepackage{wrapfig}
\usepackage{multirow}
\usepackage{verbatim}
\usepackage{float}
\usepackage{cancel}
\usepackage{caption}
\usepackage{subcaption}
\usepackage{/home/cade/Homework/latex-defs}


\title{AMATH 569 Homework 2}
\author{Cade Ballew \#2120804}
\date{April 27, 2022}

\begin{document}
	
\maketitle
	
\section{Problem 1}
Consider the PDE 
\begin{align*}
&\nabla^2u=0, \quad y>0,~ -\infty<x<\infty\\
&u(x,0)=f(x), \quad -\infty<x<\infty
\end{align*}
and assume that $f$ is of compact support and $u(x,t)\to0$ as $x\to\pm\infty$. Let $\mathcal{F}$ denote the Fourier transform applied in the $x$ domain and let
\[
U(x,y)=\mathcal{F}[u(x,y)].
\]
Then, 
\begin{align*}
\mathcal{F}[u_{xx}]&=\int_{-\infty}^\infty u_{xx}e^{i\omega x}dx=\left[u_xe^{i\omega x}\right]_{-\infty}^\infty-i\omega\int_{-\infty}^\infty u_{x}e^{i\omega x}dx\\&=\left[u_xe^{i\omega x}\right]_{-\infty}^\infty-i\omega\cancel{\left[ue^{i\omega x}\right]_{-\infty}^\infty}-\omega^2\int_{-\infty}^\infty ue^{i\omega x}dx\\&=
-\omega^2\int_{-\infty}^\infty ue^{i\omega x}dx=-\omega^2U
\end{align*}
if we make the additional assumption that $u_x(x,t)\to0$ as $x\to\pm\infty$. We also have that 
\begin{align*}
\mathcal{F}[u_{yy}]=\int_{-\infty}^\infty u_{yy}e^{i\omega x}dx=\frac{\partial^2}{\partial y^2}\int_{-\infty}^\infty ue^{i\omega x}dx=U_{yy}.
\end{align*}
Thus, Fourier transforming our equation gives new equation
\[
-\omega^2U(x,y)+U_{yy}(x,y)=0.
\]
This is an ODE that can easily be solved to have general solution
\[
U(x,y)=c_1(\omega)e^{|\omega|y}+c_2(\omega)e^{-|\omega|y}.
\]
To find a particular solution, note that we now have initial condition
\[
U(\omega,0)=\mathcal{F}[u(x,0)]=F(\omega)
\]
if we set $F(\omega)=\mathcal{F}[f(x)]$. We now also make the assumption that $u(x,y)\to0$ as $y\to\infty$ which implies that $U(\omega,y)\to0$ as $y\to\infty$. The latter requires that $c_1(x)=0$ which means that former then gives that $c_2(\omega)=F(\omega)$. Thus, 
\[
U(x,y)=F(\omega)e^{-|\omega|y}.
\]
Our solution $u$ is then obtained by applying the inverse Fourier transform 
\begin{align*}
u(x,y)=\mathcal{F}^{-1}[U(\omega,y)]=\frac{1}{2\pi}\int_{-\infty}^\infty F(\omega)e^{-|\omega|y}e^{-i\omega x}d\omega.
\end{align*}
%From this, we can clearly see that the additional assumptions we made are valid. Namely, this function tends to 0 beyond all orders as $y\to\infty$ as does $u_x(x,y)$ as $x\to\pm\infty$. %ok you really need a better arguement for this
If we wish to obtain a simplified solution, we need to apply the convolution theorem which says that 
\[
\mathcal{F}^{-1}[gh]=\mathcal{F}^{-1}[g]*\mathcal{F}^{-1}[h]
\]
for arbitrary functions $g,h$. Thus,
\begin{align*}
u(x,y)=\mathcal{F}^{-1}[F(\omega)]*\mathcal{F}^{-1}[e^{-|\omega y|}]=f(x)*\mathcal{F}^{-1}[e^{-|\omega y|}].
\end{align*}
We find that 
\begin{align*}
\mathcal{F}^{-1}[e^{-|\omega y|}]&=\frac{1}{2\pi}\int_{-\infty}^\infty e^{-|\omega|y}e^{-i\omega x}d\omega=\frac{1}{2\pi}\left(\int_{-\infty}^0 e^{\omega(y-ix)}d\omega+\frac{1}{2\pi}\int_{0}^\infty e^{-\omega(y+ix)}d\omega\right)\\&=
\frac{1}{2\pi}\left(\frac{1}{y-ix}+\frac{1}{y+ix}\right)=\frac{1}{\pi}\frac{y}{x^2+y^2}.
\end{align*}
This gives that 
\begin{align*}
u(x,y)=f(x)*\frac{1}{\pi}\frac{y}{x^2+y^2}=\frac{1}{\pi}\int_{-\infty}^{\infty}f(\tau)\frac{y}{(x-\tau)^2+y^2}d\tau.
\end{align*}
Now, we verify that the assumptions we made earlier are in fact valid. Noting that $f$ has compact support, our integral essentially has finite bounds in practice, so we can push limits inside the integral. As $y\to\infty$, $\frac{y}{(x-\tau)^2+y^2}\to0$, so $u(x,y)\to0$ as $y\to\infty$. We can also compute
\[
u_x(x,y)=\frac{1}{\pi}\int_{-\infty}^{\infty}f(\tau)\frac{-2y(x-\tau)}{((x-\tau)^2+y^2)^2}d\tau.
\]
Clearly, our assumption that $u_x(x,t)\to0$ as $x\to\pm\infty$ hold here as well as the integrand tends to zero.


\section{Problem 2}
Now, we seek to solve the problem 
\begin{align*}
&\frac{\partial}{\partial t}u=\frac{\partial^2}{\partial x^2}u, \quad 0<t,~0<x<\infty\\
&u(x,0)=0, \quad 0<x<\infty\\
&u(0,t)=T_0, \quad t>0.
\end{align*}
\subsection{Part a}
Consider the similarity transformation
\[
\eta=\frac{x}{t^\alpha}.
\]
Then,
\begin{align*}
\frac{\partial u}{\partial t}=\frac{\partial \eta}{\partial t}\frac{\partial u}{\partial \eta}=\frac{-\alpha x}{t^{\alpha+1}}\frac{\partial u}{\partial \eta}=-\frac{\alpha}{t}\eta\frac{\partial u}{\partial \eta},
\end{align*}
\begin{align*}
\frac{\partial u}{\partial x}=\frac{\partial \eta}{\partial x}\frac{\partial u}{\partial \eta}=\frac{1}{t^\alpha}\frac{\partial u}{\partial \eta},
\end{align*}
and
\begin{align*}
\frac{\partial^2 u}{\partial x^2}=\frac{\partial}{\partial x}\left(\frac{1}{t^\alpha}\frac{\partial u}{\partial \eta}\right)=\frac{1}{t^\alpha}\left(\frac{\partial \eta}{\partial x}\frac{\partial}{\partial \eta}\left(\frac{\partial u}{\partial \eta}\right)\right)=\frac{1}{t^\alpha}\left(\frac{1}{t^\alpha}\frac{\partial^2 u}{\partial \eta^2}\right)=\frac{1}{t^{2\alpha}}\frac{\partial^2 u}{\partial \eta^2}.
\end{align*}
Thus, our differential equation becomes 
\[
-\frac{\alpha}{t}\eta\frac{\partial u}{\partial \eta}=\frac{1}{t^{2\alpha}}\frac{\partial^2 u}{\partial \eta^2},
\]
so we can eliminate $t$ by taking $\alpha=\half$. Adjusting our boundary/initial conditions for the change of variables $\eta=x/\sqrt{t}$, we now have
\begin{align*}
	-&\frac{1}{2}\eta\frac{\partial u}{\partial \eta}=\frac{\partial^2 u}{\partial \eta^2}\\
	&u(\infty)=0,\\
	&u(0)=T_0.
\end{align*}
To solve this ODE, we set $v=u'$ so that we have
\[
\frac{v'}{v}=-\frac{1}{2}
\]
which has solution
\[
v(\eta)=Ce^{-\eta^2/4}
\]
after integrating. Then,
\[
u(\eta)=\int_0^\eta Ce^{-(\eta/2)^2}d\eta=c_1\erf\left(\frac{\eta}{2}\right)+c_2.
\]
Noting that $\erf(0)=0$, $\erf(\infty)=1$, our second condition gives that $c_2=T_0$ which in combination with our second condition gives that $c_1=-T_0$. Thus,
\[
u(\eta)=-T_0\erf\left(\frac{\eta}{2}\right)+T_0=T_0\erfc\left(\frac{\eta}{2}\right).
\]	
Undoing our change of variables, we conclude that 
\[
u(x,t)=T_0\erfc\left(\frac{x}{2\sqrt{t}}\right).
\]

\subsection{Part b}
Letting $\mathcal{L}$ denote the Laplace transform in $t$ and using a tilde to denote a function that has been transformed in this way, we now compute 
\[
\mathcal{L}[u_t(x,t)]=\int_0^\infty u_te^{-st}dt=\left[ue^{-st}\right]_0^\infty+s\int_0^\infty ue^{-st}dt=s\Tilde{u}(x,s)
\]
if we impose the additional assumption that $u(x,t)$ vanish as $t\to\infty$. We also have that
\[
\mathcal{L}[u_{xx}(x,t)]=\int_0^\infty u_{xx}e^{-st}dt=\frac{\partial^2}{\partial x^2}\int_0^\infty ue^{-st}dt=\Tilde{u}_{xx}(x,s).
\]
Thus, our PDE becomes
\[
s\Tilde{u}=\Tilde{u}_{xx},
\]
an ODE. Since we assume that $s>0$, this has general solution 
\[
\Tilde{u}(x,s)=c_1(s)e^{\sqrt{s}x}+c_2(s)e^{-\sqrt{s}x}.
\]
Now, we find one boundary condition by computing
\[
\tilde{u}(0,s)=\int_0^\infty u(0,t)e^{-st}dt=\int_0^\infty T_0e^{-st}dt=\frac{T_0}{s}.
\]
We also impose the assumption that $u(x,t)\to0$ as $x\to\infty$ which implies that $\Tilde{u}(x,s)\to0$ as $x\to\infty$ and allows us to conclude that $c_1(s)=0$ which then enables us to find that $c_2(s)=\frac{T_0}{s}$, so
\[
\Tilde{u}(x,s)=\frac{T_0}{s}e^{-\sqrt{s}x}.
\]
Now, we consult Wolfram-Alpha to compute the inverse transform
\[
u(x,t)=\mathcal{L}^{-1}[\Tilde{u}(x,s)]=T_0\erfc\left(\frac{x}{2\sqrt{t}}\right)
\]
which is the same solution we found in part a. From this solution, we notice that our assumption that $u(x,t)\to0$ as $t\to\infty$
clearly holds since $\frac{x}{2\sqrt{t}}\to0$ as $t\to\infty$ and $\erfc0=0$. Also, $\erfc z\to0$ as $z\to\infty$, so our assumption that $u(x,t)\to0$ as $x\to\infty$ also clearly holds. 
\end{document}
