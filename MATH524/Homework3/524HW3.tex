\documentclass{article}
\usepackage[top = 0.9in, bottom = 0.9in, left =1in, right = 1in]{geometry}
\usepackage[utf8]{inputenc}
\usepackage{hyperref}
\usepackage{listings}
\usepackage{multimedia} % to embed movies in the PDF file
\usepackage{graphicx}
\usepackage{comment}
\usepackage[english]{babel}
\usepackage{amsmath}
\usepackage{amsfonts}
\usepackage{wrapfig}
\usepackage{multirow}
\usepackage{verbatim}
\usepackage{float}
\usepackage{cancel}
\usepackage{caption}
\usepackage{subcaption}
\usepackage{mathdots}
\usepackage{/home/cade/Homework/latex-defs}


\title{MATH 524 Homework 3}
\author{Cade Ballew \#2120804}
\date{October 20, 2023}

\begin{document}
	
\maketitle
	
\section{Problem 1}
Suppose that $X$ and $Y$ are metric spaces and that $f:X\to Y$ is a continuous map. To show that if $E\in\mathcal B_Y$, then $f^{-1}(E)\in\mathcal B_X$, we first show that the family of $E\subset Y$ satisfying $f^{-1}(E)\in\mathcal B_X$ which we denote by $\mathcal A$ is a $\sigma$-algebra by verifying the required axioms. 
\begin{itemize}
	\item $f^{-1}(\emptyset)=\emptyset\in\mathcal B_X$, so $\emptyset\in\mathcal A$. 
	\item To show that $\mathcal A$ is closed under countable unions, let $E_j\in\mathcal A$ for $j\in\mathbb{N}$. Then, since the union of preimages is the preimage of the union,
	\[
	f^{-1}\left(\bigcup_{j=1}^\infty E_j\right)=\bigcup_{j=1}^\infty f^{-1}\left( E_j\right)\in\mathcal{B}_X,
	\]
	since $\mathcal{B}_X$ is a $\sigma$-algebra, so $\bigcup_{j=1}^\infty f^{-1} (E_j)\in\mathcal A$. 
	\item To show that $\mathcal A$ is closed under complements, let $E\in\mathcal{A}$. Then, since the complement of the preimage is the preimage of the complement,
	\[
	f^{-1}\left(E^c\right)=\left(f^{-1}(E)\right)^c\in\mathcal{B}_X,
	\]
	since $\mathcal{B}_X$ is a $\sigma$-algebra, so $E^c\in\mathcal A$.
\end{itemize}
Thus, $\mathcal A$ is a $\sigma$-algebra. Under continuous mappings, open sets have open preimages, so if $A\in Y$ is open, then $f^{-1}(A)$ is also open. Thus, $f^{-1}(A)\in\mathcal B_X$, so $A\in\mathcal A$ for all open sets $A$. Since $\mathcal B_Y$ is the $\sigma$-algebra generated by all open sets in $Y$, it follows from the definition that $\mathcal B_Y\subset\mathcal A$. Thus, if $E\in\mathcal B_Y$, then $E\in\mathcal A$, so $f^{-1}(E)\in\mathcal B_X$.

\section{Problem 2}
In the setup of Problem 1, consider
\[
\nu(E)=\mu\left(f^{-1}(E)\right),
\]
where $\mu$ is a Borel measure on $X$. By Problem 1, we have that $\nu(E)$ is defined for all $E\in\mathcal B_Y$ since this implies that $f^{-1}(E)\in\mathcal B_X$, and Borel measures have domain $\mathcal B_X$. It remains to show that $\nu$ is actually a measure on $\mathcal B_Y$. We do this by verifying the required axioms.
\begin{itemize}
	\item Since $\mu$ is a measure and therefore nonnegative, it follows that for any $E\in\mathcal B_Y$,
	\[
	\nu(E)=\mu\left(f^{-1}(E)\right)\geq0,
	\]
	so $\nu$ is nonnegative.
	\item By the definition of a preimage,
	\[
	\nu(\emptyset)=\mu\left(f^{-1}(\emptyset)\right)=\mu(\emptyset)=0,
	\]
	since $\mu$ is a measure. 
	\item Let $\{E_j\}_{j=1}^\infty$ be a collection of disjoint sets in $\mathcal B_Y$. Then, 
	\[
	\nu\left(\bigsqcup_{j=1}^\infty E_j\right)=\mu\left(f^{-1}\left(\bigsqcup_{j=1}^\infty E_j\right)\right)=\mu\left(\bigsqcup_{j=1}^\infty f^{-1}(E_j)\right)=\sum_{j=1}^\infty \mu\left(f^{-1}(E_j)\right)=\sum_{j=1}^\infty\nu(E_j),
	\]
	since $\mu$ is a measure and disjointness is preserved under preimages. To see this latter fact more explicitly, let $x\in f^{-1}(E_j)$. Then, $f(x)\in E_j$, so for any $k\neq j$, $f(x)\notin E_k$, and $x\notin f^{-1}(E_k)$. Thus, $\{f^{-1}(E_j)\}_{j=1}^\infty$ are disjoint. 
\end{itemize}
We have established that $\nu$ is a measure defined on $\mathcal B_Y$, so $\nu$ is a Borel measure on $Y$. 

\section{Problem 3}
Let $(\mathbb{Z}_2^\mathbb{N},\mathcal B,\mu)$ be the Borel measure space where $\mu$ is the extension of the premeasure $\mu_0$ of Homework 2 to the Borel sets in $\mathbb{Z}_2^\mathbb{N}$.

\subsection{Part a}
Consider the map $f$ defined such that for $b=(b_1,b_2,\ldots)$, 
\[
f(b)=\sum_{j=1}^\infty\frac{b_j}{2^j}.
\] 
To see that $f$ is continuous from $\mathbb{Z}_2^\mathbb{N}$ to $[0,1]$ with respect to the metric topology $\rho$ defined in Homework 1, fix $\epsilon>0$. By the triangle inequality,
\[
|f(b)-f(a)|=\left|\sum_{j=1}^\infty\frac{b_j-a_j}{2^j}\right|\leq\sum_{j=1}^\infty\frac{|b_j-a_j|}{2^j}=\rho(b,a),
\]
where we recall from Homework 1 that these infinite sums converge absolutely as they are bounded by $1$. Thus, if $\delta=\epsilon$, then $|f(b)-f(a)|<\epsilon$ whenever $\rho(b,a)<\delta$, so $f$ is continuous from $(\mathbb{Z}_2^\mathbb{N},\rho)$ to $\left([0,1],|\cdot|\right)$.

\subsection{Part b}
Given $a=(a_1,\ldots,a_n)\in\mathbb{Z}^n_2$, define the set
\[
A=\{b~|~b_j=a_j,~j=1,\ldots,n\}.
\]
To show that the image $f(A)$ is the closed interval $\left[\frac{q}{2^n},\frac{q+1}{2^n}\right]$ where $q=\sum_{j=1}^na_j2^{n-j}$, let $b\in A$. Then,
\begin{align*}
f(b)=\sum_{j=1}^\infty\frac{b_j}{2^j}=\frac{1}{2^n}\sum_{j=1}^\infty b_j2^{n-j}=\frac{1}{2^n}\sum_{j=1}^n a_j2^{n-j}+\frac{1}{2^n}\sum_{j=n+1}^\infty b_j2^{n-j}=\frac{q}{2^n}+\frac{1}{2^n}\sum_{j=1}^\infty\frac{b_{j+n}}{2^j}.
\end{align*}
Now, we bound
\[
\sum_{j=1}^\infty\frac{b_{j+n}}{2^j}\leq\sum_{j=1}^\infty\frac{1}{2^j}=1,
\] 
and
\[
\sum_{j=1}^\infty\frac{b_{j+n}}{2^j}\geq\sum_{j=1}^\infty\frac{0}{2^j}=0.
\]
Thus,
\[
\frac{q}{2^n}\leq f(b)\leq\frac{q}{2^n}+\frac{1}{2^n},
\]
so $f(A)\subset\left[\frac{q}{2^n},\frac{q+1}{2^n}\right]$. 

Now, let $x\in\left[\frac{q}{2^n},\frac{q+1}{2^n}\right]$, that is, 
\[
x=\frac{q}{2^n}+\frac{1}{2^n}c,
\]
for some $c\in[0,1]$. Denote a binary expansion of $c$ by
\[
c=\sum_{j=1}^\infty\frac{c_j}{2^j},
\]
for some $(c_1,c_2,\ldots)\in\mathbb{Z}^\mathbb{N}_2$. From class, we know that such an expansion exists, although it may not be unique. In the case where the expansion is not unique, there are exactly two expansions, so we simply choose one to denote as above. Then,
\begin{align*}
x=\frac{q}{2^n}+\frac{1}{2^n}\sum_{j=1}^\infty\frac{c_j}{2^j}=\frac{1}{2^n}\sum_{j=1}^na_j2^{n-j}+\sum_{j=1}^\infty\frac{c_j}{2^{n+j}}=\sum_{j=1}^n\frac{a_j}{2^{j}}+\sum_{j=n+1}^\infty\frac{c_{j-n}}{2^{j}}.
\end{align*}
Let $d=(a_1,\ldots,a_n,c_{1},c_{2},\ldots)\in A$. Then, 
\[
f(d)=\sum_{j=1}^n\frac{a_j}{2^{j}}+\sum_{j=n+1}^\infty\frac{c_{j-n}}{2^{j}}=x,
\]
so $x\in f(A)$, and $\left[\frac{q}{2^n},\frac{q+1}{2^n}\right]\subset f(A)$. Thus, $f(A)=\left[\frac{q}{2^n},\frac{q+1}{2^n}\right]$. 

\subsection{Part c}
Let $p$ and $q$ be integers such that $0\leq p<q\leq2^n$. Then, there exists a unique expansion of $p$ in terms of lower powers of $2$:
\[
p=\sum_{j=1}^na_j^02^{n-j},\quad a_j^0\in\{0,1\}.
\]
Assuming for now that $p\neq0$, let $k_0$ denote the last index $j$ for which $a_j^0=1$. Then, we have from lecture that $\frac{p}{2^n}$ has exactly two binary expansions:
\[
(a_1^0,\ldots,a_n^0,0,0,\ldots),\quad(a_1^0,\ldots,a_{k_0-1}^0,0,1,1,\ldots).
\]
Furthermore, if we assume that $p+1<2^n$, ${p+1}$ has a unique expansion
\[
p+1=\sum_{j=1}^na_j^12^{n-j},\quad a^1_j\in\{0,1\},
\]
and $\frac{p+1}{2^n}$ has exactly two binary expansions:
\[
(a_1^1,\ldots,a_{ n}^1,0,0,\ldots)=(a_1^0,\ldots,a_{n}^0,0,1,1,\ldots).
\]
Consider the set 
\[
B_0=A_0\cup\{(a_1^0,\ldots,a^0_{k_0-1},0,1,1,\ldots),( a^1_1,\ldots,a^1_n,0,0,\ldots)\},
\]
where 
\[
A_0=\{b~|~b_j=a^0_j,~j=1,\ldots,n\}.
\]
We claim that $f^{-1}\left(\left[\frac{p}{2^n},\frac{p+1}{2^n}\right]\right)=B_0$. To verify this, we first note that part b implies that $A_0\subset f^{-1}\left(\left[\frac{p}{2^n},\frac{p+1}{2^n}\right]\right)$, so we need only verify that the two remaining points are in $f^{-1}\left(\left[\frac{p}{2^n},\frac{p+1}{2^n}\right]\right)$ to show that $B_0\subset f^{-1}\left(\left[\frac{p}{2^n},\frac{p+1}{2^n}\right]\right)$. Observe that 
\[
f\left((a^0_1,\ldots,a^0_{k_0-1},0,1,1,\ldots)\right)=\sum_{j=1}^{k_0-1}\frac{a^0_j}{2^j}+\sum_{j=k_0+1}^\infty\frac{1}{2^j}=\sum_{j=1}^{k_0-1}\frac{a^0_j}{2^j}+\frac{1}{2^{k_0}}=\sum_{j=1}^{k_0}\frac{a^0_j}{2^j}=\sum_{j=1}^{n}\frac{a^0_j}{2^j}=\frac{p}{2^n},
\]
so $(a^0_1,\ldots,a^0_{k_0-1},0,1,1,\ldots)\in f^{-1}\left(\left[\frac{p}{2^n},\frac{p+1}{2^n}\right]\right)$. Similarly, 
\[
f\left(( a^1_1,\ldots,a^1_{n},0,0,\ldots)\right)=\sum_{j=1}^{n}\frac{a^1_j}{2^j}=\frac{p+1}{2^n},
\]
so $(a^1_1,\ldots,a^1_{n},0,0,\ldots)\in f^{-1}\left(\left[\frac{p}{2^n},\frac{p+1}{2^n}\right]\right)$. Thus, $B_0\subset f^{-1}\left(\left[\frac{p}{2^n},\frac{p+1}{2^n}\right]\right)$.

Now, let $x\in\left(\frac{p}{2^n},\frac{p+1}{2^n}\right)$. Then,
\[
x=\frac{p}{2^n}+\frac{c}{2^n}=\frac{p+c}{2^n},
\]
for some $c\in(0,1)$. From class we know that $c$ has a binary expansion which we denote
\[
c=\sum_{j=1}^n\frac{c_j}{2^j}.
\]
Then, the unique binary expansion for $x$ is given by
\[
x=\sum_{j=1}^na_j^02^{n-j}+\frac{1}{2^n}\sum_{j=1}^\infty\frac{c_j}{2^j}=\sum_{j=1}^na_j^02^{n-j}+\sum_{j=n+1}^\infty\frac{c_{j-n}}{2^j},
\]
so $f\left((a_1^0,\ldots,a_n^0,c_{n+1},c_{n+2},\ldots)\right)=x$. Note that $(a_1^0,\ldots,a_n^0,c_{n+1},c_{n+2},\ldots)\in A_0\subset B$. If $x=\frac{p}{2^n}$, then we have already established that $f\left((a_1^0,\ldots,a_n^0,0,0,\ldots)\right)=f\left((a_1^0,\ldots,a_{k_0-1}^0,0,1,1,\ldots)\right)=x$,  $(a_1^0,\ldots,a_n^0,0,0,\ldots)\in A_0\subset B$, $(a_1^0,\ldots,a_{k_0-1}^0,0,1,1,\ldots)\in B$, and these are the only two points that map to $x$. Similarly, if $x=\frac{p+1}{2^n}$, then $f\left((a_1^1,\ldots,a_{ k_1-1}^1,1,0,0,\ldots)\right)=f\left((a_1^0,\ldots,a_{n}^0,0,1,1,\ldots)\right)=x$,  $(a_1^0,\ldots,a_{n}^0,0,1,1,\ldots)\in A_0\subset B$, $(a_1^1,\ldots,a_{n}^1,0,0,\ldots)\in B$, and these are the only two points that map to $x$. Thus, $f^{-1}\left(\left[\frac{p}{2^n},\frac{p+1}{2^n}\right]\right)\subset B_0$, so $f^{-1}\left(\left[\frac{p}{2^n},\frac{p+1}{2^n}\right]\right)=B_0$

Now, observe that by taking $p\leftarrow p+k$ for $\ell=0,\ldots q-p-1$, we get that $f^{-1}\left(\left[\frac{p+\ell}{2^n},\frac{p+\ell+1}{2^n}\right]\right)=B_\ell$, where
\[
B_\ell=A_\ell\cup\{(a_1^\ell,\ldots,a^\ell_{k_\ell-1},0,1,1,\ldots),(a^{\ell+1}_1,\ldots,a^{\ell+1}_n,0,0,\ldots)\},
\]
with
\[
A_\ell=\{b~|~b_j=a^\ell_j,~j=1,\ldots,n\},
\]
and $k_\ell$ denotes the last index $j$ for which $a_{j}^\ell=1$ in the expansion
\[
p+\ell=\sum_{j=1}^na_j^\ell2^{n-j},\quad a^\ell_j\in\{0,1\}.
\]
Then, 
\[
f^{-1}\left(\left[\frac{p}{2^n},\frac{q}{2^n}\right]\right)=f^{-1}\left(\bigcup_{\ell=0}^{q-p-1}\left[\frac{p+\ell}{2^n},\frac{p+\ell+1}{2^n}\right]\right)=\bigcup_{\ell=0}^{q-p-1}B_\ell.
\]
Note that $(a^{\ell+1}_1,\ldots,a^{\ell+1}_n,0,0,\ldots)\in A_{\ell+1}$ and that
\[
(a_1^\ell,\ldots,a^\ell_{k_\ell-1},0,1,1,\ldots)=(a_1^{\ell-1},\ldots,a^{\ell-1}_{n},0,1,1,\ldots),
\]
since these are both binary expansions of $\frac{p+\ell}{2^j}$ distinct from $(a_1^\ell,\ldots,a^\ell_{k_\ell},0,0,\ldots)$ and we know from class that there are exactly two such expansions, so $(a_1^\ell,\ldots,a^\ell_{k_\ell-1},0,1,1,\ldots)\in A_{\ell-1}$. Thus,
\[
f^{-1}\left(\left[\frac{p}{2^n},\frac{q}{2^n}\right]\right)=\left(\bigcup_{\ell=0}^{q-p-1}A_\ell\right)\cup\{(a_1^0,\ldots,a^0_{k_0-1},0,1,1,\ldots),(a^{q-p}_1,\ldots,a^{q-p}_n,0,0,\ldots)\}.
\]
Note that $\bigcup_{\ell=0}^{q-p-1}A_\ell\in\mathcal{A}$ as defined on Homework 1 since $A_\ell=\Pi^{-1}_n(E_\ell)$ for $E_\ell=\{(a_0^\ell,\ldots,a_n^\ell)\}$.

Circling back to the cases $p=0$ and $q=2^n$, the first and last additional points drop out in each case, respectively. This follows from the fact that $0$ and $1$ have unique binary expansions, and the definitions of these points are not valid since these expansions require an infinite number of zeroes or ones, respectively.

To see that the measure of a single point $a\in\mathbb{Z}^\mathbb{N}_2$ is zero, define
\[
D_j=\Pi^{-1}_j\left(\{(a_1,\ldots,a_j)\}\right),\quad j\in\mathbb{N}.
\] 
Then, $\mu(D_1)=\frac{1}{2}<\infty$ and $D_1\supset D_2\supset\ldots$, so by continuity from above,
\[
\mu\left(\{a\}\right)=\mu\left(\bigcup_{j=1}^\infty D_j\right)=\lim_{j\to\infty}\mu(D_j)=\lim_{j\to\infty}\frac{1}{2^j}\card(\{(a_1,\ldots,a_j)\})=0.
\]
Thus, in all cases
\[
\mu\left(f^{-1}\left(\left[\frac{p}{2^n},\frac{q}{2^n}\right]\right)\right)=\mu\left(\bigcup_{\ell=0}^{q-p-1}A_\ell\right)=\mu\left(\Pi_n^{-1}\left(\bigcup_{\ell=0}^{q-p-1}E_\ell\right)\right)=\frac{1}{2^n}\card\left(\bigcup_{\ell=0}^{q-p-1}E_\ell\right)=\frac{q-p}{2^n},
\]
since each $E_\ell$ is distinct due to the fact that each integer has a different binary expansion in powers of 2. 

\subsection{Part d}
Now, consider $0\leq a<b\leq 1$. Then, $[a,b]=\bigcap_{j=1}^\infty F_j$ where
\[
F_j=\left[\frac{\lfloor 2^ja\rfloor}{2^j},\frac{\lceil 2^jb\rceil}{2^j}\right].
\]
Note that $0\leq\lfloor 2^ja\rfloor<\lceil 2^jb\rceil\leq2^n$, so part c can be applied to $F_j$. Observe that $F_1\supset F_2\supset\ldots$ and $F_1\subset[0,1]$, so $\mu(F_1)<\infty$ since
\[
\mu\left(f^{-1}\left([0,1]\right)\right)=\mu\left(f^{-1}\left(\left[\frac{0}{2^n},\frac{2^n}{2^n}\right]\right)\right)=\frac{2^n}{2^n}=1,
\]
for any $n\in\mathbb{N}$. We know from Problem 2 that $\mu(f^{-1}(\cdot))$ defines a Borel measure on $[0,1]$, so continuity from above implies that 
\[
\mu\left(f^{-1}\left([a,b]\right)\right)=\mu\left(f^{-1}\left(\bigcap_{j=1}^\infty F_j\right)\right)=\lim_{j\to\infty}\mu\left(f^{-1}\left(\left[\frac{\lfloor 2^ja\rfloor}{2^j},\frac{\lceil 2^jb\rceil}{2^j}\right]\right)\right)=\lim_{j\to\infty}\frac{\lceil 2^jb\rceil-\lfloor 2^ja\rfloor}{2^j}=b-a.
\]
Now, we note that $f^{-1}(\{a\})$ contains either one or two points since any number in $[0,1]$ has either one or two binary expansions; however, since we have already noted that individual points have measure zero,
\[
\mu\left(f^{-1}\left((a,b]\right)\right)=\mu\left(f^{-1}\left([a,b]\right)\right)-\mu\left(f^{-1}\left(\{a\}\right)\right)=b-a.
\]
We remark that this is also true when $a=b$, since $(a,a]=\{a\}$, and we established that $\mu\left(f^{-1}\left(\{a\}\right)\right)=0$. Now, Theorem 1.16 in Folland gives that there is a unique Borel measure $\mu_F$ corresponding to $F(x)=x$ on $[0,1]$ such that $\mu_F\left((a,b]\right)=F(b)-F(a)=b-a$ for all $0\leq a\leq b\leq 1$. By Problem 2, $\mu(f^{-1}(\cdot))$ is a Borel measure with this property. By definition the Lebesgue measure $m$ is also a Borel measure with this property. Thus, it must hold that
\[
\mu\left(f^{-1}\left(E\right)\right)=m(E),
\]
for all $E\in\mathcal B_{[0,1]}$.

\section{Problem 4 (Folland Problem 26)}
Proposition 1.20 states that if $E\in\mathcal M_\mu$ and $\mu(E)<\infty$, then for every $\epsilon>0$ there is a set $A$ that is a finite union of open intervals such that $\mu(E\triangle A)<\epsilon$. To prove this, fix $\epsilon>0$. Then, by Lemma 1.17, there exists a collection of open intervals $\{I_j\}_{j=1}^\infty$ such that $E\subset\bigcup_{j=1}^\infty I_j$ and 
\[
\sum_{j=1}^\infty\mu(I_j)<\mu(E)+\frac{\epsilon}{2}.
\]
Since $\mu(E)<\infty$, this sum converges, so there exists some $N\in\mathbb{N}$ such that
\[
\sum_{j=N+1}^\infty\mu(I_j)<\frac{\epsilon}{2}.
\]
Let $A=\bigcup_{j=1}^N I_j$. Then, by monotonicity and subadditivity
\[
\mu(E\setminus A)=\mu\left(\bigcup_{j=1}^\infty I_j\setminus\bigcup_{j=1}^N I_j\right)=\mu\left(\bigcup_{j=N+1}^\infty I_j\right)\leq\sum_{j=N+1}^\infty\mu(I_j)<\frac{\epsilon}{2}.
\]
Furthermore, again using monotonicity and subadditivity,
\[
\mu(A\setminus E)\leq\mu\left(\bigcup_{j=1}^\infty I_j\setminus E\right)\leq\mu\left(\bigcup_{j=1}^\infty I_j\right)-\mu(E)<\frac{\epsilon}{2}.
\]
Finally, by disjoint additivity, 
\[
\mu(E\triangle A)=\mu(E\setminus A)+\mu(A\setminus E)<\epsilon,
\]
as desired.

\end{document}
