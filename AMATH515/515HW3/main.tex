\documentclass{article}
\usepackage[utf8]{inputenc}
\usepackage{listings}
\usepackage{multimedia} % to embed movies in the PDF file
\usepackage{graphicx}
\usepackage{comment}
\usepackage[english]{babel}
\usepackage{amsmath}
\usepackage{amsfonts}
\usepackage{wrapfig}
\usepackage{multirow}
\usepackage{verbatim}
\usepackage{float}
\usepackage{cancel}
\usepackage{caption}
\usepackage{subcaption}

\newtheorem{theorem}{Theorem}[section]
\newtheorem{lemma}[theorem]{Lemma}
\newtheorem{corollary}[theorem]{Corollary}
%\newtheorem{algorithm}[theorem]{Algorithm}
\newtheorem{remark}[theorem]{Remark}
\newenvironment{proof}{\noindent {\bf Proof:} }{\hfill $\Box$ \\[2ex] }
\newenvironment{keywords}{\begin{quote} {\bf Key words} }
                         {\end{quote} }
\newenvironment{AMS}{\begin{quote} {\bf AMS subject classifications} }
                         {\end{quote} }


\newcommand{\eref}[1]{\mbox{\rm(\ref{#1})}}
\newcommand{\tref}[1]{\mbox{\rm\ref{#1}}}
\newcommand{\set}[2]{\left\{ #1 \; : \; #2 \right\} }
\newcommand{\deq}{\raisebox{0pt}[1ex][0pt]{$\stackrel{\scriptscriptstyle{\rm def}}{{}={}}$}}

\newcommand {\DS} {\displaystyle}

\newcommand{\real}{\mathbb{R}}
\newcommand{\compl}{\mathbb{C}}



\newcommand {\half} {\mbox{$\frac{1}{2}$}}
\newcommand{\force}{{\mathbf{f}}}
\newcommand{\strain}{{\boldsymbol{\varepsilon}}}
\newcommand{\stress}{{\boldsymbol{\sigma}}}
\renewcommand{\div}{{\boldsymbol{\nabla}}}

\newcommand {\cA} {{\cal A}}
\newcommand {\cB} {{\cal B}}
\newcommand {\cC} {{\cal C}}
\newcommand {\cD} {{\cal D}}
\newcommand {\cE} {{\cal E}}
\newcommand {\cL} {{\cal L}}
\newcommand {\cP} {{\cal P}}
\newcommand {\cQ} {{\cal Q}}
\newcommand {\cR} {{\cal R}}
\newcommand {\cV} {{\cal V}}
\newcommand {\cW} {{\cal W}}
\newcommand {\CH} {{\cal H}}
\newcommand {\CS} {{\cal S}}


\newcommand{\bzero}{\mathbf{0}}
\newcommand{\ba}{\mathbf{a}}
\newcommand{\bb}{\mathbf{b}}
\newcommand{\bc}{\mathbf{c}}
\newcommand{\bd}{\mathbf{d}}
\newcommand{\be}{\mathbf{e}}
\newcommand{\bff}{\mathbf{f}}
\newcommand{\bg}{\mathbf{g}}
\newcommand{\bh}{\mathbf{h}}
\newcommand{\bn}{\mathbf{n}}
\newcommand{\bp}{\mathbf{p}}
\newcommand{\bq}{\mathbf{q}}
\newcommand{\br}{\mathbf{r}}
\newcommand{\bs}{\mathbf{s}}
\newcommand{\bt}{\mathbf{t}}
\newcommand{\bu}{\mathbf{u}}
\newcommand{\bv}{\mathbf{v}}
\newcommand{\bw}{\mathbf{w}}
\newcommand{\bx}{\mathbf{x}}
\newcommand{\by}{\mathbf{y}}
\newcommand{\bz}{\mathbf{z}}
\newcommand{\bA}{\mathbf{A}}
\newcommand{\bB}{\mathbf{B}}
\newcommand{\bC}{\mathbf{C}}
\newcommand{\bD}{\mathbf{D}}
\newcommand{\bE}{\mathbf{E}}
\newcommand{\bF}{\mathbf{F}}
\newcommand{\bG}{\mathbf{G}}
\newcommand{\bH}{\mathbf{H}}
\newcommand{\bI}{\mathbf{I}}
\newcommand{\bJ}{\mathbf{J}}
\newcommand{\bK}{\mathbf{K}}
\newcommand{\bL}{\mathbf{L}}
\newcommand{\bM}{\mathbf{M}}
\newcommand{\bN}{\mathbf{N}}
\newcommand{\bO}{\mathbf{O}}
\newcommand{\bP}{\mathbf{P}}
\newcommand{\bQ}{\mathbf{Q}}
\newcommand{\bR}{\mathbf{R}}
\newcommand{\bS}{\mathbf{S}}
\newcommand{\bU}{\mathbf{U}}
\newcommand{\bV}{\mathbf{V}}
\newcommand{\bW}{\mathbf{W}}
\newcommand{\bX}{\mathbf{X}}
\newcommand{\bY}{\mathbf{Y}}
\newcommand{\bZ}{\mathbf{Z}}

\newcommand{\cO}{ {\cal O} }
\newcommand{\CT}{ {\cal T} }
\newcommand{\IL}{{\mathbb L}}
\newcommand{\sIL}{{{{\mathbb L}_s}}}
\newcommand{\bOmega}{{\boldsymbol{\Omega}}}
\newcommand{\bPsi}{{\boldsymbol{\Psi}}}

\newcommand{\bgamma}{{\boldsymbol{\gamma}}}
\newcommand{\bmu}{{\boldsymbol{\mu}}}
\newcommand{\blambda}{{\boldsymbol{\lambda}}}
\newcommand{\bLambda}{{\boldsymbol{\Lambda}}}
\newcommand{\bpi}{{\boldsymbol{\pi}}}
\newcommand{\bPi}{{\boldsymbol{\Pi}}}
\newcommand{\bphi}{{\boldsymbol{\phi}}}
\newcommand{\bPhi}{{\boldsymbol{\Phi}}}
\newcommand{\bpsi}{{\boldsymbol{\psi}}}
\newcommand{\btheta}{{\boldsymbol{\theta}}}
\newcommand{\bTheta}{{\boldsymbol{\Theta}}}
\newcommand{\bSigma}{{\boldsymbol{\Sigma}}}
\newcommand{\sump}{\sideset{}{^{'}}\sum} 
\DeclareMathOperator*{\Res}{Res}
\DeclareMathOperator{\OO}{O}
\DeclareMathOperator{\oo}{o}
\DeclareMathOperator{\erfc}{erfc}
\def\Xint#1{\mathchoice
   {\XXint\displaystyle\textstyle{#1}}%
   {\XXint\textstyle\scriptstyle{#1}}%
   {\XXint\scriptstyle\scriptscriptstyle{#1}}%
   {\XXint\scriptscriptstyle\scriptscriptstyle{#1}}%
   \!\int}
\def\XXint#1#2#3{{\setbox0=\hbox{$#1{#2#3}{\int}$}
     \vcenter{\hbox{$#2#3$}}\kern-.5\wd0}}
\def\ddashint{\Xint=}
\def\pvint{\Xint-}






\title{AMATH 515 Homework 3}
\author{Cade Ballew}
\date{March 9, 2022}

\begin{document}
	
\maketitle
	
\section{Problem 1}
\subsection{Part a}
Let $f(x) = \delta_{\mathbb{B}_{\infty}}(x)$. Then, 
\[
f^*(z)=\sup_x \left\{ z^Tx - \delta_{\mathbb{B}_{\infty}}(x)\right\}=\sup_{x\in\mathbb{B}_{\infty}}z^Tx.
\]
This supremum is clearly obtained by taking
\[
x_i=\begin{cases}
1, &z_i>0\\
-1, &z_i<0.\\\end{cases}
\]
Thus, 
\[
f^*(z)=\sum_{i=1}^n|z_i|=\|z\|_1.
\]

\subsection{Part b}
Let $f(x) = \delta_{\mathbb{B}_2}(x)$. Then, by the Cauchy-Schwarz inequality,
\[
f^*(z)=\sup_x \left\{ z^Tx - \delta_{\mathbb{B}_{2}}(x)\right\}=\sup_{x\in\mathbb{B}_{2}}z^Tx\leq\sup_{\|x\|<1}\|z\|\|x\|\leq\|z\|.
\]
However, this bound is attained by taking $x=z/\|z\|$ (which is on the boundary of the ball) which gives that $z^Tx=\|z\|$. Thus, this is bound is in fact the supremum, so 
\[
f^*(z)=\|z\|.
\]

\subsection{Part c}
Let $f(x)=\exp(x)$ where $x\in\real$. Then,
\[
f^*(z)=\sup_x \left\{ z^Tx - \exp(x)\right\}=\sup_x \left\{ zx - \exp(x)\right\},
\]
because $z$ and $x$ are scalars. Setting the derivative with respect to $x$ equal to zero to find the $x$ that maximizes this quantity,
\[
0=z-\exp(x),
\]
so $x=\log(z)$ which gives that 
\[
f^*(z)=z\log(z)-\exp(\log(z))=z(\log(z)-1).
\]

\subsection{Part d}
Let $f(x) =  \log(1+\exp(x))$ be a scalar function. Then,
\[
f^*(z)=\sup_x \left\{ zx - \log(1+\exp(x))\right\}.
\]
Setting the derivative equal to 0,
\[
0=z-\frac{e^x}{1+e^x}.
\]
We can write this as 
\[
e^x=\frac{z}{1-z},
\]
so $x=\log{\frac{z}{1-z}}$ and 
\[
f^*(z)=z\log{\frac{z}{1-z}}-\log\left(1+\frac{z}{1-z}\right)=z\log\frac{z}{1-z}-\log\frac{1}{1-z}.
\]

\subsection{Part e}
Let $f(x) = x\log(x)$ be a scalar function. Then,
\[
f^*(z)=\sup_x \left\{ zx - x\log(x)\right\}.
\]
Setting the derivative equal to 0,
\[
0=z-(\log{x}+1),
\]
so $x=\exp(z-1)$ and
\[
f^*(z)=z\exp(z-1)-\exp(z-1)(z-1)=\exp(z-1).
\]

\section{Problem 2}
Let $g$ be a given convex function.
\subsection{Part a}
If $f(x)=\lambda g(x)$, then
\begin{align*}
f^*(z)&=\sup_x \left\{ z^Tx - \lambda g(x)\right\}=\sup_x \left\{ \lambda\left(\frac{1}{\lambda}z^Tx - g(x)\right)\right\}\\&=
\lambda\sup_x \left\{ \left(\frac{z}{\lambda}\right)^Tx - g(x)\right\}=\lambda g^*\left(\frac{z}{\lambda}\right).
\end{align*}

\subsection{Part b}
If $f(x) = g(x-a) + \langle x, b \rangle$, then
\begin{align*}
f^*(z)&=\sup_x \left\{ z^Tx - g(x-a) - b^Tx\right\}=\sup_x \left\{ (z-b)^Tx - g(x-a)\right\}\\&=
\sup_x \left\{ (z-b)^Tx - g(x-a)\right\}=\sup_x \left\{ (z-b)^T(x-a) - g(x-a)+(z-b)^Ta\right\}.
\end{align*}
Now, we rename $x'=x-a$ to get
\begin{align*}
f^*(z)&=\sup_{x'} \left\{ (z-b)^Tx' - g(x')+(z-b)^Ta\right\}\\&=
\sup_{x'} \left\{ (z-b)^Tx' - g(x')\right\}+a^T(z-b)=g^*(z-b)+a^T(z-b).
\end{align*}

\subsection{Part c}
If $f(x) = \inf_z \left\{g(x,z)\right\}$, then
\begin{align*}
f^*(z)&=\sup_x \left\{ z^Tx - \inf_y \left\{g(x,y)\right\}\right\}=\sup_x \left\{ z^Tx + \sup_y \left\{-g(x,y)\right\}\right\}\\&=
\sup_{x,y} \left\{ z^Tx -g(x,y)\right\}.
\end{align*}
Now, note that
\begin{align*}
g^*(z,w)=\sup_{x,y} \left\{ z^Tx+w^Ty -g(x,y)\right\},
\end{align*}
so clearly,
\[
f^*(z)=g^*(z,0).
\]
\subsection{Part d}
If $f(x) = \inf_z \left\{\frac{1}{2}\|x-z\|^2 + g(z)\right\}$, define $h(x,y)=\frac{1}{2}\|x-y\|^2 + g(y)$ and compute
\begin{align*}
h^*(z,w)&=\sup_{x,y} \left\{ z^Tx+w^Ty -\frac{1}{2}\|x-y\|^2 - g(y)\right\}\\&=
\sup_{x,y} \left\{ z^T(x-y)+(w+z)^Ty -\frac{1}{2}\|x-y\|^2 - g(y)\right\}.
\end{align*}
Now, define $x'=x-y$ to get
\begin{align*}
h^*(z,w)&=\sup_{x',y} \left\{ z^Tx'+(w+z)^Ty -\frac{1}{2}\|x'\|^2 - g(y)\right\}\\&=
\sup_{x'} \left\{ z^Tx'-\frac{1}{2}\|x'\|^2\right\} +\sup_{y}\left\{(w+z)^Ty - g(y)\right\}\\&=
p^*(z)+g^*(w+z) 
\end{align*}
where $p(x)=\frac{1}{2}\|x\|^2$. From page 8 of lecture 12, we know that $p^*(z)=p(z)$, so 
\[
h^*(z,w)=\frac{1}{2}\|z\|^2+g^*(w+z).
\]
Now, we can apply part c of this problem to conclude that
\[
f^*(z)=h^*(z,0)=\frac{1}{2}\|z\|^2+g^*(z).
\]
% Now, we find $x$ by computing the gradient with respect to $x$ and setting it equal to 0.
% \[
% 0=z-(x-y),
% \]
% so $x=y+z$. Then,
% \begin{align*}
% f^*(x)&=h^*(z,0)=\sup_{y} \left\{ z^T(y+z) -\frac{1}{2}\|z\|^2 - g(y)\right\}\\&=\sup_{y} \left\{ z^Ty +\frac{1}{2}\|z\|^2 - g(y)\right\}=\sup_{y} \left\{ z^Ty  - g(y)\right\}+\frac{1}{2}\|z\|^2
% \end{align*}

\section{Problem 3}
\subsection{Part a}
Let $f$ be a closed proper convex function. To derive the Moreau identity, we first remark that for a closed proper convex function $g$, $u=\prox_g(v)$ iff $v-u\in\partial g(u)$. To see this, note that
\[
\prox_g(v)\arg\min_x\left\{\frac{1}{2}\|x-v\|^2+g(x)\right\},
\]
so the convexity of $g$ and the differentiability of the 2-norm gives that $u=\prox_g(v)$ iff $u$ minimizes the interior quantity iff $(u-v)+\partial g(u)\ni0$ which occurs iff $v-u\in\partial g(u)$. Using this, let $x=\prox_f(z)$. Then, $z-x\in\partial f(x)$. Now, we apply the Fenchel flip to get that $x\in\partial f^*(z-x)$. This means that 
\[
\underbrace{z}_{v}-\underbrace{(z-x)}_{u}\in\partial f^*(\underbrace{z-x}_u),
\]
so we again apply our property to get that this is true iff
\[
z-x=\prox_{f^*}(z).
\]
Thus,
\[
\mbox{prox}_{f}(z) + \mbox{prox}_{f^*}(z)=x+(z-x)=z.
\]

\subsection{Part b}
From part a and problem 1 part a, we know that
\[
z=\prox_{\mathbb{B}_{\infty}}(z)+\prox_{\mathbb{B}_{\infty}^*}(z)=\prox_{\mathbb{B}_{\infty}}(z)+\prox_{\|\cdot\|_1}(z).
\]
Thus,
\[
\prox_{\|\cdot\|_1}(z)=z-\prox_{\mathbb{B}_{\infty}}(z).
\]
From homework 2, we know that the elements of $\prox_{\mathbb{B}_{\infty}}(z)$ are given by
\[
\prox_{\mathbb{B}_{\infty}}(z)_i=\begin{cases}
1, &z_i>1\\
-1, &z_i<-1\\
z_i, &|z_i|\leq 1.
\end{cases}
\]
Thus, the elements of $\prox_{\|\cdot\|_1}(z)$ are given by
\[
\prox_{\|\cdot\|_1}(z)_i=\begin{cases}
z_i-1, &z_i>1\\
z_i+1, &z_i<-1\\
0, &|z_i|\leq 1
\end{cases}
\]
which matches what we derived on homework 2.\\
Similarly, problem 1 part b tells us that
\[
z=\prox_{\mathbb{B}_{2}}(z)+\prox_{\mathbb{B}_{2}^*}(z)=\prox_{\mathbb{B}_{2}}(z)+\prox_{\|\cdot\|_2}(z),
\]
so 
\[
\prox_{\|\cdot\|_2}(z)=z-\prox_{\mathbb{B}_{2}}(z).
\]
From lecture 9 page 8, we know that 
\[
\prox_{\mathbb{B}_{2}}(z)=\begin{cases}
z, &\|z\|\leq1\\
\frac{z}{\|z\|}, &\|z\|>1.\end{cases}
\]
Thus,
\[
\prox_{\|\cdot\|_2}(z)=\begin{cases}
0, &\|z\|\leq1\\
z-\frac{z}{\|z\|}, &\|z\|>1\end{cases}
\]
which matches what we derived on homework 2. 

\section{Problem 4}
\subsection{Part a}
Consider 
\[
\min_{x} \sum_{i=1}^n g(\langle a_i, x\rangle) - b^TAx + R(x),
\]
where $g$ is convex and $R$ is any regularizer. From lecture 13 page 11, we know that the problem
\[
\min_{x} f(Qx-d)+h(x)+c^Tx
\]
has dual
\[
\sup_z -z^Td-f^*(z)-h^*(-Q^Tz-c).
\]
To write our problem in this form, take $Q=A$, $d=0$, $h(x)=R(x)$, $c=-A^Tb$ ($c^T=-b^TA$), and $f(x)=\sum_{i=1}^n g(x_i)$.
Now, compute
\begin{align*}
f^*(z)&=\sup_x \left\{ z^Tx - \sum_{i=1}^n g(x_i)\right\}=\sup_{x_1,\ldots,x_n} \left\{\sum_{i=1}^n (z_ix_i-g(x_i))\right\}\\&=
\sum_{i=1}^n\sup_{x_i} \left\{z_ix_i-g(x_i)\right\}=\sum_{i=1}^ng^*(z_i).
\end{align*}
Thus, our problem has dual
\[
\sup_z \left\{-\sum_{i=1}^ng^*(z_i)-R^*(-A^Tz+A^Tb)\right\}=\sup_z \left\{-\sum_{i=1}^ng^*(z_i)-R^*(A^T(b-z))\right\}.
\]

\subsection{Part b}
Now, we compute the dual of 
\[
\min_x \sum_{i=1}^n \log(1+\exp(\langle a_i, x \rangle))  - b^TAx  + \frac{\lambda}{2}\|x\|^2. 
\]
by taking $g(x)=\log(1+\exp(x))$ and $R(x)=\frac{\lambda}{2}\|x\|^2$. From problem 1 part d, we know that 
\[
g^*(z)=z\log\frac{z}{1-z}-\log\frac{1}{1-z}.
\]
From problem 2 part a and the conjugate of $\frac{1}{2}\|x\|^2$ that we already established, we find that
\[
R^*(z)=\frac{\lambda}{2}\|z/\lambda\|^2=\frac{1}{2\lambda}\|z\|^2.
\]
Thus, the dual of our problem is given by
\[
\sup_z \left\{-\sum_{i=1}^n\left(z_i\log\frac{z_i}{1-z_i}-\log\frac{1}{1-z_i}\right)-\frac{1}{2\lambda}\|A^T(b-z)\|^2\right\}.
\]
\subsection{Part c}
Now, we compute the dual of 
\[
\min_x \sum_{i=1}^n \exp(\langle a_i, x \rangle) - b^TAx +  \lambda\|x\|_1. 
\]
by taking $g(x)=\exp(x)$ and $R(x)=\lambda\|x\|_1$. From problem 1 part c, we have that 
\[
g^*(z)=z(\log(z)-1).
\]
We also know from problem 1 part a that $\delta^*_{\mathbb{B}_{\infty}}(z)=\|z\|_1$, so using the fact that the 1-norm is closed convex, $\|z\|^*_1=\delta_{\mathbb{B}_{\infty}}(z)$ ($f^{**}=f$ under these conditions from lecture 12 page 13). Applying problem 2 part a,
\[
R^*(z)=\lambda\delta_{\mathbb{B}_{\infty}}(z/\lambda)=\delta_{\mathbb{B}_{\infty}}(z/\lambda)=\delta_{\lambda\mathbb{B}_{\infty}}(z).
\]
Thus, the dual of our problem is given by
\[
\sup_z \left\{-\sum_{i=1}^nz_i(\log(z_i)-1)-\delta_{\lambda\mathbb{B}_{\infty}}(A^T(b-z))\right\}.
\]

\section{Problem 5}
\subsection{Part a}
Consider the Capped Simplex $\Delta_k$
\[
\Delta_k := \left\{x: 1^Tx = k, \quad 0 \leq x_i \leq 1 \quad \forall i. \right\}
\]
and the projection problem is given by 
\[
\mbox{proj}_{\Delta_k}(y) = \arg\min_{x \in \Delta_k} \frac{1}{2}\|x-y\|^2.
\]
From lecture 13 page 13, we know that the dual constraint for the condition $1^Tx = k$ is given by 
\[
\sup_z z(1^Tx - k),
\]
so we can write our dual problem as 
\[
\sup_z\min_{x\in[0,1]^n}\left\{\frac{1}{2}\|x-y\|^2+z(1^Tx - k)\right\}.
\]
Taking the gradient the interior and setting it equal to zero to minimize with respect to x. 
\[
0=(x-y)+z1
\]
which gives $x^*=y-z1$. However, we need to ensure that $x^*\in[0,1]^n$ which we do by projecting onto that space as
\[
x^*_i=\begin{cases}
y_i-z, &0\leq y_i-z\leq1\\
1, &y_i-z>1\\
0, &y_i-z<0.
\end{cases}
\]
Plugging this in for $x$, our dual problem becomes
\[
\sup_z \sum_{i=1}^n f(z)_i
\]
where 
\[
f(z)_i=\begin{cases}
-\frac{z^2}{2}+zy_i, &0\leq y_i-z\leq1\\
\frac{(y_i-1)^2}{2}+z, &y_i-z>1\\
\frac{y_i^2}{2}, &y_i-z<0.
\end{cases}
\]

\subsection{Part b}
To solve this dual, we note that our function is separable, so we set its gradient equal to 0 by computing
\[
f(z)'_i=\begin{cases}
-z+y_i, &0\leq y_i-z\leq1\\
1, &y_i-z>1\\
0, &y_i-z<0.
\end{cases}
\]
and writing
\[
0=\sum_{i=1}^n f'(z)_i.
\]
We find the solution to numerically which yields an optimal $z^*$.

\subsection{Part c}
Once we find an optimal $z^*$, we can then find our optimal $x^*$ as described above, meaning that elementwise,
\[
\mbox{proj}_{\Delta_k}(y)_i=x^*_i=\begin{cases}
y_i-z^*, &0\leq y_i-z\leq1\\
1, &y_i-z>1\\
0, &y_i-z<0.
\end{cases}
\]

\end{document}