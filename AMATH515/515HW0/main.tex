\documentclass{article}
\usepackage[utf8]{inputenc}
\usepackage{listings}
\usepackage{multimedia} % to embed movies in the PDF file
\usepackage{graphicx}
\usepackage{comment}
\usepackage[english]{babel}
\usepackage{amsmath}
\usepackage{amsfonts}
\usepackage{subfigure}
\usepackage{wrapfig}
\usepackage{multirow}
\usepackage{tikz}
\usepackage{verbatim}

\newtheorem{theorem}{Theorem}[section]
\newtheorem{lemma}[theorem]{Lemma}
\newtheorem{corollary}[theorem]{Corollary}
%\newtheorem{algorithm}[theorem]{Algorithm}
\newtheorem{remark}[theorem]{Remark}
\newenvironment{proof}{\noindent {\bf Proof:} }{\hfill $\Box$ \\[2ex] }
\newenvironment{keywords}{\begin{quote} {\bf Key words} }
                         {\end{quote} }
\newenvironment{AMS}{\begin{quote} {\bf AMS subject classifications} }
                         {\end{quote} }


\newcommand{\eref}[1]{\mbox{\rm(\ref{#1})}}
\newcommand{\tref}[1]{\mbox{\rm\ref{#1}}}
\newcommand{\set}[2]{\left\{ #1 \; : \; #2 \right\} }
\newcommand{\deq}{\raisebox{0pt}[1ex][0pt]{$\stackrel{\scriptscriptstyle{\rm def}}{{}={}}$}}

\newcommand {\DS} {\displaystyle}

\newcommand{\real}{\mathbb{R}}
\newcommand{\compl}{\mathbb{C}}



\newcommand {\half} {\mbox{$\frac{1}{2}$}}
\newcommand{\force}{{\mathbf{f}}}
\newcommand{\strain}{{\boldsymbol{\varepsilon}}}
\newcommand{\stress}{{\boldsymbol{\sigma}}}
\renewcommand{\div}{{\boldsymbol{\nabla}}}

\newcommand {\cA} {{\cal A}}
\newcommand {\cB} {{\cal B}}
\newcommand {\cC} {{\cal C}}
\newcommand {\cD} {{\cal D}}
\newcommand {\cE} {{\cal E}}
\newcommand {\cL} {{\cal L}}
\newcommand {\cP} {{\cal P}}
\newcommand {\cQ} {{\cal Q}}
\newcommand {\cR} {{\cal R}}
\newcommand {\cV} {{\cal V}}
\newcommand {\cW} {{\cal W}}
\newcommand {\CH} {{\cal H}}
\newcommand {\CS} {{\cal S}}


\newcommand{\bzero}{\mathbf{0}}
\newcommand{\ba}{\mathbf{a}}
\newcommand{\bb}{\mathbf{b}}
\newcommand{\bc}{\mathbf{c}}
\newcommand{\bd}{\mathbf{d}}
\newcommand{\be}{\mathbf{e}}
\newcommand{\bff}{\mathbf{f}}
\newcommand{\bg}{\mathbf{g}}
\newcommand{\bh}{\mathbf{h}}
\newcommand{\bn}{\mathbf{n}}
\newcommand{\bp}{\mathbf{p}}
\newcommand{\bq}{\mathbf{q}}
\newcommand{\br}{\mathbf{r}}
\newcommand{\bs}{\mathbf{s}}
\newcommand{\bt}{\mathbf{t}}
\newcommand{\bu}{\mathbf{u}}
\newcommand{\bv}{\mathbf{v}}
\newcommand{\bw}{\mathbf{w}}
\newcommand{\bx}{\mathbf{x}}
\newcommand{\by}{\mathbf{y}}
\newcommand{\bz}{\mathbf{z}}
\newcommand{\bA}{\mathbf{A}}
\newcommand{\bB}{\mathbf{B}}
\newcommand{\bC}{\mathbf{C}}
\newcommand{\bD}{\mathbf{D}}
\newcommand{\bE}{\mathbf{E}}
\newcommand{\bF}{\mathbf{F}}
\newcommand{\bG}{\mathbf{G}}
\newcommand{\bH}{\mathbf{H}}
\newcommand{\bI}{\mathbf{I}}
\newcommand{\bJ}{\mathbf{J}}
\newcommand{\bK}{\mathbf{K}}
\newcommand{\bL}{\mathbf{L}}
\newcommand{\bM}{\mathbf{M}}
\newcommand{\bN}{\mathbf{N}}
\newcommand{\bO}{\mathbf{O}}
\newcommand{\bP}{\mathbf{P}}
\newcommand{\bQ}{\mathbf{Q}}
\newcommand{\bR}{\mathbf{R}}
\newcommand{\bS}{\mathbf{S}}
\newcommand{\bU}{\mathbf{U}}
\newcommand{\bV}{\mathbf{V}}
\newcommand{\bW}{\mathbf{W}}
\newcommand{\bX}{\mathbf{X}}
\newcommand{\bY}{\mathbf{Y}}
\newcommand{\bZ}{\mathbf{Z}}

\newcommand{\cO}{ {\cal O} }
\newcommand{\CT}{ {\cal T} }
\newcommand{\IL}{{\mathbb L}}
\newcommand{\sIL}{{{{\mathbb L}_s}}}
\newcommand{\bOmega}{{\boldsymbol{\Omega}}}
\newcommand{\bPsi}{{\boldsymbol{\Psi}}}

\newcommand{\bgamma}{{\boldsymbol{\gamma}}}
\newcommand{\bmu}{{\boldsymbol{\mu}}}
\newcommand{\blambda}{{\boldsymbol{\lambda}}}
\newcommand{\bLambda}{{\boldsymbol{\Lambda}}}
\newcommand{\bpi}{{\boldsymbol{\pi}}}
\newcommand{\bPi}{{\boldsymbol{\Pi}}}
\newcommand{\bphi}{{\boldsymbol{\phi}}}
\newcommand{\bPhi}{{\boldsymbol{\Phi}}}
\newcommand{\bpsi}{{\boldsymbol{\psi}}}
\newcommand{\btheta}{{\boldsymbol{\theta}}}
\newcommand{\bTheta}{{\boldsymbol{\Theta}}}
\newcommand{\bSigma}{{\boldsymbol{\Sigma}}}
\newcommand{\sump}{\sideset{}{^{'}}\sum} 
\DeclareMathOperator*{\Res}{Res}
\DeclareMathOperator{\OO}{O}
\DeclareMathOperator{\oo}{o}
\DeclareMathOperator{\erfc}{erfc}
\def\Xint#1{\mathchoice
   {\XXint\displaystyle\textstyle{#1}}%
   {\XXint\textstyle\scriptstyle{#1}}%
   {\XXint\scriptstyle\scriptscriptstyle{#1}}%
   {\XXint\scriptscriptstyle\scriptscriptstyle{#1}}%
   \!\int}
\def\XXint#1#2#3{{\setbox0=\hbox{$#1{#2#3}{\int}$}
     \vcenter{\hbox{$#2#3$}}\kern-.5\wd0}}
\def\ddashint{\Xint=}
\def\pvint{\Xint-}




\oddsidemargin=10pt


\title{AMATH 515 Homework 0}
\author{Cade Ballew}
\date{January 12, 2022}

\begin{document}

\maketitle

\section{Problem 2}
\subsection{Part a}
For $f(x)=\sin(x_1 + x_2 + x_3 + x_4)$, 
\[
\nabla f(x) = \begin{bmatrix} \cos(x_1 + x_2 + x_3 + x_4) \\ \cos(x_1 + x_2 + x_3 + x_4) \\\cos(x_1 + x_2 + x_3 + x_4) \\\cos(x_1 + x_2 + x_3 + x_4) \end{bmatrix}
\]
and 
\[
\nabla^2 f(x) = 
\begin{bmatrix} -\sin(x_1 + x_2 + x_3 + x_4)  & -\sin(x_1 + x_2 + x_3 + x_4)  &-\sin(x_1 + x_2 + x_3 + x_4)  &-\sin(x_1 + x_2 + x_3 + x_4)  \\
-\sin(x_1 + x_2 + x_3 + x_4)  & -\sin(x_1 + x_2 + x_3 + x_4)  &-\sin(x_1 + x_2 + x_3 + x_4)  &-\sin(x_1 + x_2 + x_3 + x_4)  \\
-\sin(x_1 + x_2 + x_3 + x_4)  & -\sin(x_1 + x_2 + x_3 + x_4)  &-\sin(x_1 + x_2 + x_3 + x_4)  &-\sin(x_1 + x_2 + x_3 + x_4)  \\
-\sin(x_1 + x_2 + x_3 + x_4)  & -\sin(x_1 + x_2 + x_3 + x_4)  &-\sin(x_1 + x_2 + x_3 + x_4)  &-\sin(x_1 + x_2 + x_3 + x_4) 
\end{bmatrix}.
\]

\subsection{Part b}
For $f(x) = \|x\|^2 = x_1^2 + x_2^2 + x_3^2 + x_4^2$,
\[
\nabla f(x) = \begin{bmatrix} 2x_1 \\ 2x_2 \\2x_3 \\2x_4 \end{bmatrix}
\]
and 
\[
\nabla^2 f(x) = 
\begin{bmatrix} 2  & 0  &0  &0  \\
0  & 2  &0  &0 \\
0  & 0  &2  &0  \\
0  & 0  &0  &2
\end{bmatrix}.
\]

\subsection{Part c}
For $f(x) = \ln(x_1x_2x_3x_4)$,
\[
\nabla f(x) = \begin{bmatrix} \frac{1}{x_1} \\ \frac{1}{x_2} \\\frac{1}{x_3} \\\frac{1}{x_4} \end{bmatrix}
\]
and 
\[
\nabla^2 f(x) = 
\begin{bmatrix} -\frac{1}{x_1^2}  & 0  &0  &0  \\
0  & -\frac{1}{x_2^2}  &0  &0 \\
0  & 0  &-\frac{1}{x_3^2}  &0  \\
0  & 0  &0  &-\frac{1}{x_4^2}
\end{bmatrix}.
\]

\section{Problem 3} 
\subsection{Part a}
Because the matrix \[
\begin{bmatrix}
1 & 0 & 0 & 0 \\
\pi & 2 & 0 & 0 \\
64 & -15 & 3 & 0 \\ 
321 & 0 & 0 & 5 
\end{bmatrix}
\]
is lower triangular, its eigenvalues are simply its diagonal entries $1, 2, 3, 5$.

\subsection{Part b}
The matrix 
\[
A = \begin{bmatrix}
1 \\ 0 \\ 1
\end{bmatrix} 
\begin{bmatrix}
1 & 1 & 1
\end{bmatrix} =
\begin{bmatrix}
1&1&1\\
0&0&0\\
1&1&1
\end{bmatrix}
\]
has all linearly dependent columns, so a basis for its range space can be given by just one of them, namely 
\[\begin{bmatrix}
1 \\0\\1
\end{bmatrix}.\]
A vector in the nullspace of A must satisfy 
\[
\begin{bmatrix}
1&1&1\\
0&0&0\\
1&1&1
\end{bmatrix}
\begin{bmatrix}
x_1\\x_2\\x_3
\end{bmatrix}
=\begin{bmatrix}
0\\0\\0
\end{bmatrix}
\],
so satisfy $x_1+x_2+x_3=0$. Thus, it must have the form
\[
\begin{bmatrix}
x_1\\x_2\\-x_1-x_2
\end{bmatrix}
=x_1\begin{bmatrix}
1\\0\\-1
\end{bmatrix}+x_2\begin{bmatrix}
0\\1\\-1
\end{bmatrix},
\]
so a basis for the nullspace of A is given by
\[
\left\{\begin{bmatrix}
1\\0\\-1
\end{bmatrix},\begin{bmatrix}
0\\1\\-1
\end{bmatrix}\right\}.
\]

\subsection{Part c}
If $A$ is a $10 \times 5$ matrix, and $b$ is a vector in $\mathbb{R}^{10}$, then $A^TA$ is $5\times5$ and $A^Tb$ is $5\times1$.\\
The system $Ax=b$ could have zero or infinitely many solutions.\\
The system $A^TAx=A^Tb$ may have one or infinitely many solutions. \\
If the columns of $A$ are linearly independent, the system $Ax=b$ has infinitely many solutions and the system $A^TAx=A^Tb$ has one solution.
\end{document}
